\documentclass{article}

\usepackage[english]{babel}
\usepackage[utf8]{inputenc}
\usepackage{polski}
\usepackage[T1]{fontenc}
 
\usepackage[margin=1.5in]{geometry} 

\usepackage{color} 
\usepackage{amsmath}
\usepackage{amsfonts}                                                                   
\usepackage{graphicx}                                                             
\usepackage{booktabs}
\usepackage{amsthm}
\usepackage{pdfpages}
\usepackage{wrapfig}
\usepackage{hyperref}
\usepackage{etoolbox}

\makeatletter
\newenvironment{definition}[1]{%
    \trivlist
    \item[\hskip\labelsep\textbf{Definition. #1.}]
    \ignorespaces
}{%
    \endtrivlist
}
\newenvironment{fact}[1]{%
    \trivlist
    \item[\hskip\labelsep\textbf{Fact. #1.}]
    \ignorespaces
}{%
    \endtrivlist
}
\newenvironment{theorem}[1]{%
    \trivlist
    \item[\hskip\labelsep\textbf{Theorem. #1.}]
    \ignorespaces
}{%
    \endtrivlist
}
\newenvironment{information}[1]{%
    \trivlist
    \item[\hskip\labelsep\textbf{Information. #1.}]
    \ignorespaces
}{%
    \endtrivlist
}
\newenvironment{identities}[1]{%
    \trivlist
    \item[\hskip\labelsep\textbf{Identities. #1.}]
    \ignorespaces
}{%
    \endtrivlist
}
\makeatother

\title{Abstract algebra and coding}  
\author{Rafał Włodarczyk}
\date{INA 2, 2024}

\begin{document}

\maketitle

\tableofcontents

\section{Definitions}

\subsection{Group}

A group is a set \( G \) along with an operation \( \cdot \) satisfying the following axioms:
\begin{enumerate}
    \item \textbf{Operation is defined}: \( \forall a, b \in G: a \cdot b \in G \)
    \item \textbf{Operation is associative}: \( \forall a, b, c \in G: a \cdot (b \cdot c) = (a \cdot b) \cdot c \)
    \item \textbf{Identity element exists}: \( \exists e \in G: \forall a \in G: a \cdot e = e \cdot a = a \)
    \item \textbf{Inverse element exists}: \( \forall a \in G: \exists a^{-1} \in G: a \cdot a^{-1} = a^{-1} \cdot a = e \)
\end{enumerate}

\subsection{Subgroup}

A subset \( H \) of a group \( G \) is a subgroup if:
\begin{enumerate}
    \item \( H \) is closed under the operation: \( \forall a, b \in H: a \cdot b \in H \)
    \item \( H \) is closed under inverses: \( \forall a \in H: a^{-1} \in H \)
    \item \( H \) contains the identity element: \( e \in H \)
    \item \( H \) is closed under associativity: \( \forall a, b \in H: a \cdot b \in H \)
\end{enumerate}

It suffices to check closure under operation and inverses for \( H \).

\subsection{Normal Subgroup}

A subgroup \( H \) of a group \( G \) is normal in \( G \) if:
\begin{enumerate}
    \item \( H \) is a subgroup of \( G \):
    \begin{itemize}
        \item \( H \) is closed under the operation: \( \forall a, b \in H: a \cdot b \in H \)
        \item \( H \) has an inverse element: \( \forall a \in H: a^{-1} \in H \)
    \end{itemize}
    \item \( H \) is closed under conjugation: \( \forall a \in G: aHa^{-1} = H \)
\end{enumerate}

\subsection{Group Homomorphism}

A group homomorphism is a function \( f: G \to H \) satisfying:
\[ f(a \cdot b) = f(a) \cdot f(b) \]

\subsection{Kernel of a Homomorphism}

The kernel of a homomorphism \( f \) is the set of elements in \( G \) mapped to the identity element in \( H \):
\[ \ker f = \{ a \in G : f(a) = e_H \} \]

\subsection{Image of a Homomorphism}

The image of a homomorphism is the set of elements in \( H \) obtained by applying \( f \) to elements in \( G \):
\[ \text{Im} f = \{ f(a) \in H : a \in G \} \]

\subsection{Order of an Element in a Group}

The order of an element \( a \) in a group \( G \) is defined as:
\[ \text{ord}(a) = \min\{ n \in \mathbb{N} : a^n = e \} \]

If no such \( n \) exists, \( a \) has infinite order.

\subsection{Generator of a Group}

An element \( a \) in a group \( G \) is a generator if:
\[ \forall b \in G: \exists n \in \mathbb{Z}: b = a^n \]

\subsection{Coset of a Group}

The coset of a subgroup \( H \) in a group \( G \) is defined as:
\begin{itemize}
    \item Left coset: \( aH = \{ a \cdot h : h \in H \} \)
    \item Right coset: \( Ha = \{ h \cdot a : h \in H \} \)
    \item Double coset: \( aH = Ha \)
\end{itemize}

\subsection{Cyclic Group}

A group \( G \) is cyclic if there exists an element \( a \in G \) such that:
\[ G = \{ a^n : n \in \mathbb{Z} \} \]

Thus, \( G \) is generated by one element \( a \).

\subsection{Dihedral Group}

The dihedral group \( D_n \) is the group of symmetries of a regular \( n \)-gon.

\subsection{Quotient Group}

The quotient group \( G/H \) of a group \( G \) by a normal subgroup \( H \) is the set of cosets of \( H \) in \( G \) with the operation:
\[ (aH) \cdot (bH) = (a \cdot b)H \]

\subsection{Ring}

A ring \( R \) is a set with two operations \( + \) and \( \cdot \) satisfying:
\begin{enumerate}
    \item \( (R, +) \) is an abelian group
    \item \( \cdot \) is associative: \( \forall a, b, c \in R: a \cdot (b \cdot c) = (a \cdot b) \cdot c \)
    \item Distributivity of multiplication over addition:
    \[ \forall a, b, c \in R: a \cdot (b + c) = a \cdot b + a \cdot c \quad \text{and} \quad (a + b) \cdot c = a \cdot c + b \cdot c \]
\end{enumerate}

\subsection{Invertible Element in a Ring}

An element \( a \) in a ring \( R \) is invertible if there exists an element \( b \in R \) such that:
\[ a \cdot b = b \cdot a = 1 \]

The set of invertible elements is denoted as \( R^* = \{ a \in R : a \text{ is invertible} \} \)

\subsection{Subring}

A subring of a ring \( R \) is a subset \( S \subseteq R \) with operations \( + \) and \( \cdot \) such that:
\begin{enumerate}
    \item \( S \) is closed under addition: \( \forall a, b \in S: a + b \in S \)
    \item \( S \) is closed under multiplication: \( \forall a, b \in S: a \cdot b \in S \)
\end{enumerate}

\subsection{Ring Homomorphism}

A ring homomorphism is a function \( f: R \to S \) satisfying:
\begin{enumerate}
    \item \( f \) is a group homomorphism: \( f(a + b) = f(a) + f(b) \)
    \item \( f \) is a ring homomorphism: \( f(a \cdot b) = f(a) \cdot f(b) \)
\end{enumerate}

\subsection{Ideal}

An ideal of a ring \( R \) is a subset \( I \subseteq R \) satisfying:
\begin{enumerate}
    \item \( (I, +) \) is a subgroup of the abelian group \( (R, +) \)
    \item \( I \) is closed under multiplication: \( \forall a, b \in I: a \cdot b \in I \)
    \item \( I \) is closed under addition: \( \forall a, b \in I: a + b \in I \)
    \item \( I \) is closed under multiplication by ring elements: \( \forall a \in I, r \in R: a \cdot r \in I \) and \( r \cdot a \in I \)
\end{enumerate}

\subsection{Principal Ideal}

A principal ideal generated by an element \( a \in R \) is the set:
\[ \langle a \rangle = \{ a \cdot r : r \in R \} \]

\subsection{Quotient Ring}

The quotient ring \( R/I \) of a ring \( R \) by an ideal \( I \) is the set of cosets of \( I \) in \( R \) with operations:
\[ (a + I) + (b + I) = (a + b) + I \]
\[ (a + I) \cdot (b + I) = (a \cdot b) + I \]

\section{Theorems}

\subsection{Lagrange's Theorem}

If \( G \) is a finite group and \( H \) is a subgroup of \( G \), then the order of \( H \) divides the order of \( G \):
\[ |G| = |H| \cdot [G : H] \]
Or equivalently:
\[ |H| \mid |G| \]

\subsection{Chinese Remainder Theorem}

If \( m_1, m_2, \ldots, m_n \) are pairwise coprime integers, then the system of congruences:
\[ \begin{cases} x \equiv a_1 \pmod{m_1} \\ x \equiv a_2 \pmod{m_2} \\ \vdots \\ x \equiv a_n \pmod{m_n} \end{cases} \]
has exactly one solution modulo \( m_1 \cdot m_2 \cdot \ldots \cdot m_n \).

\subsection{Euler's Theorem}

For any integer \( a \) coprime to \( n \), it holds that:
\[ a^{\varphi(n)} \equiv 1 \pmod{n} \]

\end{document}


\end{document}
