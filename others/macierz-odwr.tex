\documentclass{article}

\usepackage[polish]{babel}
\usepackage[utf8]{inputenc}
\usepackage{polski}
\usepackage[T1]{fontenc}
 
\usepackage[margin=1.5in]{geometry} 

\usepackage{color} 
\usepackage{amsmath}                                                                    
\usepackage{amsfonts}                                                                   
\usepackage{graphicx}                                                             
\usepackage{booktabs}
\usepackage{amsthm}
\usepackage{pdfpages}
\usepackage{wrapfig}

\theoremstyle{definition}
\newtheorem{de}{Definicja}[subsection]

\theoremstyle{definition}
\newtheorem{tw}{Twierdzenie}[subsection]

\theoremstyle{definition}
\newtheorem{pk}{Przykład}[subsection]

\theoremstyle{definition}
\newtheorem*{fakt}{FAKT}

\author{Rafal Wlodarczyk}
\title{Algebra - Wykład VI}  
\date{CBD 1 Sem.}

\begin{document}

\maketitle

\section{Macierz Odwrotna}

Sprowadzamy macierz blokową $[A|I]$ od postaci $[I|B]$, Wtedy $A^{-1}=B$

\begin{center}
    $[A|I]\rightarrow ... \rightarrow [I|A^{-1}]$
\end{center}

Bezwyznacznikowe znajdowanie macierzy odwrotnej:
\begin{enumerate}
    \item zamiana kolejności wierszy
    \item pomnożenie dowolnego wiersza przez liczbę $k\in\mathbb{R}$ różną od zera
    \item dodanie wielokrotności dowolnego wiersza do innego wiersza
\end{enumerate}
Metoda ta nazywana jest metodą elementarną.

\begin{pk}
    Rozważmy następujący przykład:\\\\
    $ A=\left[\begin{array}{ccc}
        2 & 0 & 1 \\
        0 & 1 & 0 \\
        0 & 3 & 3 
        \end{array}\right]  $;
    $ det (A) = 6+0+0-0-0-0=6\neq 0$\\
    $ [A|I]=\left[\begin{array}{ccc|ccc}
        2 & 0 & 1 & 1 & 0 & 0 \\
        0 & 1 & 0 & 0 & 1 & 0 \\
        0 & 3 & 3 & 0 & 0 & 1 
        \end{array}\right] =$
    $ \left[\begin{array}{ccc|ccc}
        2 & 0 & 1 & 1 & 0 & 0 \\
        0 & 1 & 0 & 0 & 1 & 0 \\
        0 & 0 & 3 & 0 & -3 & 1 
        \end{array}\right] =$
    $ \left[\begin{array}{ccc|ccc}
        2 & 0 & 1 & 1 & 0 & 0 \\
        0 & 1 & 0 & 0 & 1 & 0 \\
        0 & 0 & 1 & 0 & -1 & \frac{1}{3} 
        \end{array}\right] =$\\
    $ =\left[\begin{array}{ccc|ccc}
        2 & 0 & 0 & 1 & 1 & \frac{-1}{3} \\
        0 & 1 & 0 & 0 & 1 & 0 \\
        0 & 0 & 1 & 0 & -1 & \frac{1}{3} 
        \end{array}\right]  =$
    $ \left[\begin{array}{ccc|ccc}
        1 & 0 & 0 & \frac{1}{2} & \frac{1}{2} & -\frac{1}{6} \\
        0 & 1 & 0 & 0 & 1 & 0 \\
        0 & 0 & 1 & 0 & -1 & \frac{1}{3} 
        \end{array}\right]  $\\

    Odpowiedź:\\

    $ A^{-1}= \left[\begin{array}{ccc}
        \frac{1}{2} & \frac{1}{2} & -\frac{1}{6} \\
        0 & 1 & 0 \\
        0 & -1 & \frac{1}{3} 
        \end{array}\right]  $
    
\end{pk}

\begin{pk}
    Rozważmy kolejny przykład:\\
    Obliczenia:\\\\
    $ A = \left[\begin{array}{ccc}
        2 & 2 & 3 \\
        1 & -1 & 0 \\
        -1 & 2 & 1 
    \end{array}\right]  $;
    $ det (A) = -2 + 0 + 6 -3 -0 -2 = -1 \neq 0$\\
    $ [A|I]=\left[\begin{array}{ccc|ccc}
        2 & 2 & 3 & 1 & 0 & 0 \\
        1 & -1 & 0 & 0 & 1 & 0 \\
        -1 & 2 & 1 & 0 & 0 & 1
    \end{array}\right] =$
    $ \left[\begin{array}{ccc|ccc}
        1 & -1 & 0 & 0 & 1 & 0 \\
        2 & 2 & 3 & 1 & 0 & 0 \\
        -1 & 2 & 1 & 0 & 0 & 1 
        \end{array}\right] =$
    $ \left[\begin{array}{ccc|ccc}
        1 & -1 & 0 & 0 & 1 & 0 \\
        0 & 4 & 3 & 1 & -2 & 0 \\
        0 & 1 & 1 & 0 & 1 & 1 
        \end{array}\right] =$\\
    $ =\left[\begin{array}{ccc|ccc}
        1 & -1 & 0 & 0 & 1 & 0 \\
        0 & 1 & 1 & 0 & 1 & 1 \\
        0 & 4 & 3 & 1 & -2 & 0 
        \end{array}\right] =$
    $ \left[\begin{array}{ccc|ccc}
        1 & -1 & 0 & 0 & 1 & 0 \\
        0 & 1 & 1 & 0 & 1 & 1 \\
        0 & 0 & -1 & 1 & -6 & -4 
        \end{array}\right]  =$
    $ \left[\begin{array}{ccc|ccc}
        1 & -1 & 0 & 0 & 1 & 0 \\
        0 & 1 & 1 & 0 & 1 & 1 \\
        0 & 0 & 1 & -1 & 6 & 4 
        \end{array}\right]  =$\\
    $ =\left[\begin{array}{ccc|ccc}
        1 & -1 & 0 & 0 & 1 & 0 \\
        0 & 1 & 0 & 1 & -5 & -3 \\
        0 & 0 & 1 & -1 & 6 & 4 
        \end{array}\right] =$
    $ \left[\begin{array}{ccc|ccc}
        1 & 0 & 0 & 1 & -4 & -3 \\
        0 & 1 & 0 & 1 & -5 & -3 \\
        0 & 0 & 1 & -1 & 6 & 4 
        \end{array}\right]  $\\

    Odpowiedź:\\

    $ A^{-1}= \left[\begin{array}{ccc}
        1 & -4 & -3 \\
        1 & -5 & -3 \\
        -1 & 6 & 4 
        \end{array}\right]  $
    
\end{pk}

\begin{pk}
    Policz sam $>\_<$\\
    $A= \left[\begin{array}{ccc}
        1 & 1 & 1 \\
        0 & 1 & 1 \\
        1 & 0 & 1 
        \end{array}\right]  $;
    $det(A) = 1 + 0 + 1 - 1 - 0 - 0 = 1 \neq 0$\\
    $[A|I]= \left[\begin{array}{ccc|ccc}
        1 & 1 & 1 & 1 & 0 & 0 \\
        0 & 1 & 1 & 0 & 1 & 0 \\
        1 & 0 & 1 & 0 & 0 & 1 
        \end{array}\right] =$
    $ \left[\begin{array}{ccc|ccc}
        1 & 1 & 1 & 1 & 0 & 0 \\
        0 & 1 & 1 & 0 & 1 & 0 \\
        0 & -1 & 0 & -1 & 0 & 1 
        \end{array}\right]  =$
    $ \left[\begin{array}{ccc|ccc}
        1 & 1 & 1 & 1 & 0 & 0 \\
        0 & 1 & 1 & 0 & 1 & 0 \\
        0 & 0 & 1 & -1 & 1 & 1 
        \end{array}\right]  =$\\
    $ =\left[\begin{array}{ccc|ccc}
        1 & 1 & 0 & 2 & -1 & -1 \\
        0 & 1 & 0 & 1 & 0 & -1 \\
        0 & 0 & 1 & -1 & 1 & 1 
        \end{array}\right]=$
    $ \left[\begin{array}{ccc|ccc}
        1 & 0 & 0 & 1 & -1 & 0 \\
        0 & 1 & 0 & 1 & 0 & -1 \\
        0 & 0 & 1 & -1 & 1 & 1 
        \end{array}\right]  $\\

    Odpowiedź:\\

    $ A^{-1}= \left[\begin{array}{ccc}
        1 & -1 & 0 \\
        1 & 0 & 1 \\
        -1 & 1 & 1 
        \end{array}\right]  $
    
\end{pk}

\section{Równania macierzowe}
Równanie złożone z macierzy, której niewiadomą jest macierz nazywamy równaniem macierzowym.
Przykładami równań macierzowych mogą być:\\

$A\cdot X = B$
\begin{itemize}
    \item $A, B$ - dane macierzowe
    \item $X$ - szukana macierz
\end{itemize}

\begin{pk}
Rozważmy alternatywne rozwiązanie:\\
$A^{-1}\cdot A\cdot X = A^{-1}\cdot B$\\
$I\cdot X = A^{-1} \cdot B$\\
$X=A^{-1} = B$\\\\
---\\
$X\cdot A = B$\\
$X\cdot A\cdot A^{-1} = B\cdot A^{-1}$\\
$X\cdot I = B\cdot A^{-1}$\\
$X = B\cdot A^{-1}$\\
---\\
$A\cdot X\cdot B =C$\\
$A^{-1}\cdot A \cdot X \cdot B = A^{-1} \cdot C$\\
$I\cdot X \cdot B = A^{-1} \cdot C$\\
$X\cdot B = A^{-1} \cdot C$\\
$X\cdot B\cdot B^{-1} = A^{-1} \cdot C \cdot B^{-1}$\\
$X\cdot I=A^{-1}\cdot C\cdot B^{-1}$\\
$X=A^{-1} \cdot C \cdot B^{-1}$
\end{pk}

\begin{pk}
Rozwiązać równanie macierzowe:\\

$ \begin{bmatrix}
    1 & 0 \\
    2 & 3 
    \end{bmatrix}  \cdot X \cdot $
$ \begin{bmatrix}
    3 & 2 \\
    0 & 1 
    \end{bmatrix} = $
$ \begin{bmatrix}
    1 & 2 \\
    0 & 3 
    \end{bmatrix}  $\\\\

$ [A_1|I]=\begin{bmatrix}
    1 & 0 & 1 & 0 \\
    2 & 3 & 0 & 1 
    \end{bmatrix} =$
$ \begin{bmatrix}
    1 & 0 & 1 & 0 \\
    0 & 3 & -2 & 1 
    \end{bmatrix} =$
$ \begin{bmatrix}
    1 & 0 & 1 & 0 \\
    0 & 1 & -\frac{2}{3} & \frac{1}{3} 
    \end{bmatrix}  $\\

$[A_2|I]=\begin{bmatrix}
    3 & 2 & 1 & 0 \\
    0 & 1 & 0 & 1 
    \end{bmatrix} =$
$ \begin{bmatrix}
    3 & 0 & 1 & -2 \\
    0 & 1 & 0 & 1 
    \end{bmatrix}  =$
$ \begin{bmatrix}
    1 & 0 & \frac{1}{3} & -\frac{2}{3} \\
    0 & 1 & 0 & 1 
    \end{bmatrix}  $\\

$X=\left( \begin{bmatrix}
    1 & 0 \\
    -\frac{2}{3} & \frac{1}{3} 
    \end{bmatrix} 
 \begin{bmatrix}
    1 & 2 \\
    0 & 3 
    \end{bmatrix} \right)$
$ \begin{bmatrix}
    \frac{1}{3} & -\frac{2}{3} \\
    0 & 1 
    \end{bmatrix}  $\\

$X=\begin{bmatrix}
    1 & 2 \\
    -\frac{2}{3} & -\frac{1}{3} 
    \end{bmatrix}  $
$ \begin{bmatrix}
    \frac{1}{3} & -\frac{2}{3} \\
    0 & 1 
    \end{bmatrix}  $\\

$X=\begin{bmatrix}
    \frac{1}{3} & \frac{4}{3} \\
    -\frac{2}{3} & \frac{1}{9} 
    \end{bmatrix}  $

\end{pk}
\begin{pk}
Rozwiązać równanie macierzowe:\\

$ X\cdot \begin{bmatrix}
    1 & 2 \\
    2 & 4 
    \end{bmatrix}  =$
$ \begin{bmatrix}
    0 & 3 \\
    1 & 2 
    \end{bmatrix}  $

$ \begin{bmatrix}
    a & b \\
    c & d 
    \end{bmatrix} \cdot \begin{bmatrix}
        1 & 2 \\
        2 & 4 
        \end{bmatrix}  =$
$ \begin{bmatrix}
    0 & 3 \\
    1 & 2 
    \end{bmatrix}  $\\

$ \begin{bmatrix}
    a+2b & 2a+4b \\
    a+2d & 2c+4d 
    \end{bmatrix} =$
$ \begin{bmatrix}
    0 & 3 \\
    1 & 2 
    \end{bmatrix}  $\\

$\begin{cases}
    a+2b=0\\
    2a+4b=3\\
    c+2d=1\\
    2c+4d=2
\end{cases}=$
$\begin{cases}
    a+2b=0\\
    a+2b=\frac{3}{2}\\
    c+2d=1\\
    c+2d=1
\end{cases}$
$0\neq \frac{3}{2}$ sprzeczność\\
Odp. Równanie macierzowe nie ma rozwiązania.
\end{pk}

\begin{pk}
Rozwiązać równanie macierzowe:\\

$ \begin{bmatrix}
    1 & 2 & 0 \\
    3 & -1 & 1 
    \end{bmatrix}  \cdot X =$
$ \begin{bmatrix}
    10 \\
    20 
    \end{bmatrix}  $\\

$ \begin{bmatrix}
    1 & 2 & 0 \\
    3 & -1 & 1 
    \end{bmatrix}  \cdot$
$ \begin{bmatrix}
    a \\
    b \\
    c \\
    \end{bmatrix}  $
$ =\begin{bmatrix}
    10 \\
    20 
    \end{bmatrix}  $\\

$\begin{cases}
    a+2b=10\\
    3a-b+c=20
\end{cases}=$
$\begin{cases}
    a=10-2b\\
    3(10-2b)-b+c=20
\end{cases}=$
$\begin{cases}
    a=10-b\\
    30-6b-b-c=20
\end{cases}=$\\
$\begin{cases}
    a=10-b\\
    -7b+c=-10
\end{cases}=$
$\begin{cases}
    a=10-b\\
    c=7b-10
\end{cases}$\\\\

Odpowiedź:
$X=\begin{bmatrix}
    10-2b \\
    b \\
    7b-10 
\end{bmatrix}  $\\\\
Równanie ma nieskończenie wiele rozwiązań.
\end{pk}

\end{document}