\documentclass{article}

\usepackage[polish]{babel}
\usepackage[utf8]{inputenc}
\usepackage{polski}
\usepackage[T1]{fontenc}
 
\usepackage[margin=1.5in]{geometry} 

\usepackage{color} 
\usepackage{amsmath}                                                                    
\usepackage{amsfonts}                                                                   
\usepackage{graphicx}                                                             
\usepackage{booktabs}
\usepackage{amsthm}
\usepackage{pdfpages}
\usepackage{wrapfig}
\usepackage{hyperref}

\theoremstyle{definition}
\newtheorem{de}{Definicja}[subsection]

\theoremstyle{definition}
\newtheorem{tw}{Twierdzenie}[subsection]

\theoremstyle{definition}
\newtheorem{pk}{Przykład}[subsection]

\theoremstyle{definition}
\newtheorem*{fakt}{FAKT}

\author{Rafal Wlodarczyk}
\title{Analiza Matematyczna}  
\date{INA 1 Sem. 2023}

\begin{document}

\maketitle

\section{Wykład pierwszy}

Liczby naturalne $\mathbb{N}=\{1,2,3,\dots\}$

\begin{de}
    Zasada indukcji matematycznej. Niech będzie dana własność liczb naturalnych, która czyni zadość warunkom:
    \begin{enumerate}
        \item Liczba 1 posiada tę własność.
        \item Jeżeli liczba $n$ posiada tę własność, to posiada ją również liczba $n+1$.
    \end{enumerate}
    Zasada indukcji matematycznej mówi, że przy tych założeniach każda liczba naturalna posiada tę własność.
\end{de}

\begin{pk}
    $1+2+\dots+n=\frac{n(n+1)}{2}$.
    \begin{enumerate}
        \item $n=1$ $L=1$ $P=\frac{1(1+1)}{2}$
        \item $\forall_{n\geq 1} 1+2+\dots+n=\frac{n(n+1)}{2} \implies 1+2+\dots+n+n+1=\frac{(n+1)(n+2)}{2}$.\\
        Z założenia indukcyjnego mamy:\\
        $1+2+...+n+(n+1)=\frac{n(n+1)}{2}+n+1=(n+1)(\frac{n}{2}+1)=\frac{(n+1)(n+2)}{2}$\\
        Na mocy zasady indukcji matematycznej teza zachodzi $\square$.
    \end{enumerate}
\end{pk}

\begin{pk}
    Nierówność Bernoulli'ego. Niech $a\geq 1$, wówczas dla dowolnego $n$ naturalnego zachodzi nierówność:
    $(1+a)^n \geq 1 + na$
    \begin{enumerate}
        \item $n=1$, $L=(1+a)^1 = 1+a$, $P=1+1\cdot a = 1+ a$, $L=P$, własność zachodzi
        \item $\forall_{n>1} (1+a)^n \geq 1 + na \implies (1+a)^{n+1} \geq 1 + (n+1)\cdot a$\\
        $(1+a)^{n+1}=(1+a)^n\cdot(1+a)\geq^{ind.} (1+na)(1+a)$\\
        $(1+a)^{n+1} \geq 1 + a + na + na^2 = 1 + (n+1)a + na^2 \geq 1 + (n+1)\cdot a$\\
        Na mocy zasady indukcji matematycznej nierówność jest prawdziwa.
    \end{enumerate}
\end{pk}

Liczby Całkowite $\mathbb{Z}=\{0,1,-1,2,-2,3,-3,...\}$

\begin{de}
    Liczby wymierne $\mathbb{Q}$ to liczby postaci:\\
    \begin{center}
        $\frac{p}{q}$, gdzie $p,q\in\mathbb{Z}$ oraz $q\neq0$
    \end{center}
    Zbiór liczb wymiernych jest liniowo uporządkowany, to znaczy każde dwie liczby wymierne można połączyć jednym ze znaków:\\
    $a<b, a>b, a=b$.\\
    Dodawanie $\mathbb{Q}$\\
    $\frac{p_1}{q_1}+\frac{p_2}{q_2}=\frac{p_1q_2 + p_2q_1}{q_1q_2}$\\
    Mnożenie $\mathbb{Q}$\\
    $\frac{p_1}{q_1} \cdot \frac{p_2}{q_2} = \frac{p_1p_2}{q_1q_2}$\\
    Własności:\\
    \begin{enumerate}
        \item Przemienność $a+b=b+a$
        \item Łączność $a+(b+c)=(a+b)+c$
        \item Rozdizelność $(a+b)c=ac+bc$
    \end{enumerate}
\end{de}

Uwaga. Jeżeli ($a<c \land c<b) \iff a<c<b$. Mówimy wtedy, że c leży między liczbami a i b.\\
Z twierdzenia Pitagorasa $1^2+1^2=x^2 \implies x=\sqrt{2}$. D-d niewymierności $\sqrt{2}$ jako ćwiczenie.\\

Własność - zbiór $\mathbb{Q}$ jest zbiorem gęstym.\\
Niech $a,b$ będą dowolnymi liczbami wymiernymi, takimi że $a<b$. Wówczas istnieje liczba $c$ leżąca między liczbami $a$ i $b$.\\
np.: $c=\frac{a+b}{2}$\\

Liczby rzeczywiste $\mathbb{R}$

\begin{de}
    Mówimy, że \underline{zbiór jest ograniczony} jeżeli istnieją takie dwie liczby $m, M$, że:
    \begin{center}
        $\forall_{x\in X} m\leq x\leq M, X\in[m,M]$
    \end{center}
    Uwaga analogicznie ograniczoność z dołu i góry osobno.
\end{de}

\begin{de}
    Kres górny zbioru. Niech $X$ będzie zbiorem ograniczonym z góry.
    \begin{center}
        $\forall_{x\in X} \exists_M x\leq M$
    \end{center}
    Kresem górnym zbioru nazywamy najmniejszą liczbę ograniczającą zbiór $X$ z góry.\\
    $(-\infty, 1)$: kres $1$\\
    $(-\infty, 1) \cup (1,2]$: kres $2$\\
\end{de}

\subsection{Aksjomat Zupełności}
Każdy ograniczony z góry podzbiór liczb rzeczywistych ma kres górny.\\

\begin{de}
    Kres dolny zbioru nazywamy największą liczbą ograniczjącą zbiór $X$ z dołu.
    \begin{center}
        $\forall_{x\in X} \exists_m m\leq X$
    \end{center}
    $(-1, +\infty)$: kres $-1$\\
    $(2, +\infty)$: kres $2$\\
    Kres górny zbioru i kres dolny zbioru to pojęcia dualne.
\end{de}

\subsection{Wartość bezwzględna}

$|a|=\begin{cases}
    a, a\geq 0\\
    -a, a<0
\end{cases}$

\begin{pk}
    Własności:
    \begin{itemize}
        \item $|a|=|-a|$
        \item $|ab|=|a|\cdot|b|$
        \item $|a+b|\leq|a|+|b|$
        \item $|a-b|\leq|a|+|b|$
        \item $|a|-|b| \leq |a-b|$
    \end{itemize}
\end{pk}

\begin{de}
    Współczynnik Newtona. Zakładamy że $n,k$ są liczbami naturalnymi,
    takimi że $n\geq k$. Współczynnik Newtona określam wzorem:
    \begin{center}
        $\binom{n}{k}=\frac{n!}{k!(n-k)!}$
    \end{center}
    Własności:
    \begin{enumerate}
        \item $\binom{n}{0}=\binom{n}{n}=1$
        \item $\binom{n}{k}=\binom{n}{n-k}$
        \item $\binom{n+1}{k}=\binom{n}{k-1}+\binom{n}{k}$
        \item $\binom{n}{k+1}=\binom{n}{k}\cdot \frac{n-k}{k+1}$
    \end{enumerate}
\end{de}

\begin{itemize}
    \item Symbol sumy $\sum$
    \item Symbol iloczynu $\Pi$
\end{itemize}

\begin{de}
    Nierówność Cauchy'ego - Schwarza. Niech $a_1, a_2, \dots, a_n$ oraz $b_1, b_2, \dots, b_n$ będą dowolnymi liczbami rzeczywistymi.
    Wówczas zachodzi nierówność:
    \begin{center}
        $(a_1^2+a_2^2+\dots+a_n^2)(b_1^2+b_2^2+\dots+b_n^2)\geq(a_1b_1+a_2b_2+\dots+a_nb_n)^2$
    \end{center}
    lub równoważnie:
    \begin{center}
        $\sum_{i=1}^{n} a_i^2 \cdot \sum_{i=1}^{n} b_i^2 \geq \left(\sum_{i=1}^{n} a_ib_i\right)^2$
    \end{center}
\end{de}

\section{Wykład drugi}

\begin{de}
    Ciąg liczbowy to funkcja z $\mathbb{N}$ w $\mathbb{R}$. Stosujemy zapis $a_1,a_2,\dots,a_n$. Przykłady:
    \begin{itemize}
        \item $a_n=c+(n-1)d$ - arytmetyczny
        \item $b_n=cq^{n-1}$ - geometryczny
        \item $c_n=n!$
        \item $d_{n+1}=2^{d_n}$ - rekurencyjny
    \end{itemize}
\end{de}

\begin{de}
    Ciąg monotoniczny. 
    \begin{enumerate}
        \item $a_n$ jest rosnący $\iff \forall_{n\in\mathbb{N}} a_n<a_{n+1}$
        \item $a_n$ jest malejący $\iff \forall_{n\in\mathbb{N}} a_n>a_{n+1}$
        \item $a_n$ jest niemalejący $\iff \forall_{n\in\mathbb{N}} a\leq a_{n+1}$
        \item $a_n$ jest nierosnący $\iff \forall_{n\in\mathbb{N}} a\geq a_{n+1}$
    \end{enumerate}
    Analogicznie definiujemy ciąg monotoniczny od pewnego miejsca:
    \begin{enumerate}
        \item $a_n$ jest rosnący od $n_0 \iff \forall_{n>n_0} a_n<a_{n+1}$
    \end{enumerate}
\end{de}

\begin{de}
    Liczbą graniczną ciągu $a_n$ nazywamy liczbę $g$, taką że:
    \begin{center}
        $\forall_{\varepsilon>0}\exists_{n_0}\forall_{n>n_0} |a_n-g|<\varepsilon$
    \end{center}
    Piszemy wtedy: $lim_{n\rightarrow \infty} a_n = g$ lub $a_n\rightarrow g$.\\
    $|a_n-g|<\varepsilon \iff -\varepsilon < a_n -g < \varepsilon \iff g-\varepsilon < a_n < g+\varepsilon$
\end{de}

\section{Wykład trzeci}

\begin{tw}
Twierdzenie (o ciągu monotonicznym i ograniczonym)\\
a) Ciąg rosnący i ograniczony z góry jest zbieżny.\\
$\forall_{n>n_0} a_n\leq a_{n+1}$ i $\forall_{n\in \mathbb{N} a_n< M}$ $\implies \exists \lim_{n\rightarrow \infty} a_n$\\
b) Ciąg malejący i ograniczony z dołu jest zbieżny.\\
$\forall_{n>n_0} a_n\geq a_{n+1}$ i $\forall_{n\in \mathbb{N} a_n> m}$ $\implies \exists \lim_{n\rightarrow \infty} a_n$\\
Idea dowodu:\\
$A=\{a_{n_0+1},a_{n_0+2},\dots,a_n,\dots\} \in \mathbb{R}$\\
A - ograniczony, istnieje kres górny zbioru A\\
Każdy ograniczony podzbiór liczb rzeczywistych ma kres\\
czyli $sup(A)$ (??) $sup(A)=lim_{n\rightarrow \infty a_n}$
\end{tw}

\begin{pk}
Rozważmy następujący ciąg rekurencyjny:
$a_1=\sqrt{2}$ $a_{n+1}=\sqrt{2+a_n}$\\
Idea dowodu indukcyjnego:\\
1. $a_n\leq 2$, indukcja po $n$\\
2. $a_n\leq a_{n+1}$, indukcja po $n$. $a_n\leq a_{n+1}\implies a_{n+1}\leq a_{n+2}$\\
3. $\sqrt{2+a_n}\leq \sqrt{2+a_{n+1}}$ kwadrat stronami rozwiązuje krok indukcyjny\\

$\forall_{n\geq 1} a_n \leq 2 \implies a_{n+1}\leq 2$\\
$a_{n+1}=\sqrt{2+a_n}\leq_{z. ind} \sqrt{2+2}=2$\\

Na mocy twierdzenia o ciągu monotonicznym i ograniczonym istnieje:\\
$lim_{n\rightarrow \infty} a_n = g$\\

\begin{center}
$a_{n+1}=\sqrt{2+a_n}$, $lim_{n\rightarrow \infty} a_n = g = lim_{n\rightarrow \infty} a_{n+1} = g$\\
$g=\sqrt{2+g}$\\
$g^2-g-2=0$\\
$\Delta=9=3^2$\\
$g_1=\frac{1+3}{2}=2$ lub $g_2=\frac{1-3}{2}=-1$, które nie zachodzi, zatem $lim a_n=g_1$
\end{center}
\end{pk}

\begin{de}
Podciąg ciągu\\
Niech $a_n$ będzie dowolnym ciągiem. Niech $n_1, n_2, ... n_k$ będzie pewnym rosnącym ciągiem liczb naturalnych.
Wówczas ciąg $a_{nk}=(a_{n1}, a_{n2}, a_{n3}, ...)$ Nazywamy podciągiem ciągu.
\end{de}

\begin{pk}
Rozważmy następujące przykłady ($\mathbb{N}=\{1,2,3,4,\dots\}$):\\
a) $a_n=(-1)^{n}$, $n\in\mathbb{N}$\\
$a_{2k}=(-1)^{n}=1$, $k\in\mathbb{N}$\\
$(a_2, a_4, a_6, ...)$ - podciąg o wyrazach parzystych.\\
b) $a_{2k-1}=(-1)^{2k-1}=-1, n\in\mathbb{N}$\\
$(a_1, a_3, a_5, ...)$ - podciąg o wyrazach nieparzystych.\\
$S=\{1,-1\}$\\
c) $(1, \frac{1}{2}, 3, \frac{1}{4}, 5, \frac{1}{6},\dots)$\\
$a_{2k-1}=2k-1$ - podciąg o wyr. nieparzystych.\\
$a_{2k}=\frac{1}{2k}$ - podciąg o wyr. parzystych.\\
$S=\{0, \infty\}$
d) $sin(\frac{n\pi}{3})$ - $plot(sin(\frac{n\pi}{3}),(n,1,17)) \leftarrow$ wolframalpha
\end{pk}

\begin{de}
Liczba $s$ jest punktem skupienia ciągu $a_n\iff s$ jest granicą właściwą lub niewłaściwą pewnego podciągu.
Oznaczenie $S$ - zbiór punktów skupienia.\\

Jeśli $lim_{n\rightarrow \infty} a_n = \infty \implies a_n$ ma granicę niewłaściwą $+\infty$
\end{de}

\begin{itemize}
\item $sup()$ - superior - kres górny
\item $inf()$ - inferior - kres dolny
\end{itemize}

\begin{de}
Granica górna ciągu $a_n$ to kres górny granic podciągu $a_n$.\\
$\lim_{n\rightarrow \infty} sup(a_n) = \lim_{n\rightarrow\infty} a_n$
\end{de}

\begin{de}
Granica dolna ciągu $a_n$ to kres dolny granic podciągu $a_n$.\\
$\lim_{n\rightarrow \infty} inf(a_n) = \lim_{n\rightarrow\infty} a_n$
\end{de}

$\lim inf(a_n)\leq \lim sup(a_n)$, równość dla granicy ciągu.

\begin{tw}
Twierdzenie (Bolzano - Weierstrassa). Każdy ciąg ograniczony ma podciąg zbieżny.
\href{https://en.wikipedia.org/wiki/Bolzano%E2%80%93Weierstrass_theorem}{(English Wikipedia)}\\
D-d. $\forall_{n\in\mathbb{N}} m \leq a_n \leq M$ Dzielimy przedział $[m_1,M_1]$ na dwa podprzedziały:
$[m_1, \frac{m_1+M_1}{2}]$, $[\frac{m_1+M_1}{2},M_1]$. Przynajmniej w jednym z przedziałów jest nieskończenie wiele wyrazów ciągu.
Oznaczmy tę połówkę przez $[m_2, M_2]$. Postępujemy tak dalej i mamy:\\
$\forall_{k\in\mathbb{N}} m_1\leq m_k\leq a_{nk} \leq M_k \leq M_1$\\
$M_k$ malejący i ograniczony $\implies$ zbieżny $g_1$\\
$m_k$ rosnący i ograniczony $\implies$ zbieżny $g_2$\\
$g_1=g_2=g$\\
$M_k-m_k=\frac{M_1-m_1}{2}$\\
$M_k\rightarrow g_1; m_k\rightarrow g_2$, ponieważ $\frac{M_1-m_1}{2^k}\rightarrow 0$

\end{tw}

\begin{de}
Ciąg $a_n$ nazywamy ciągiem Cauchy'ego, wtedy i tylko wtedy, gdy:\\
$\forall_{\varepsilon > 0}\exists_{n_0}\forall_{n,m>n_0} |a_n-a_m|<\varepsilon$.
\end{de}

\begin{tw}
Ciąg liczb rzeczywistych jest zbieżny $\iff$ jest ciągiem Cauchy'ego.
\end{tw}

\begin{pk}
$x_n = \frac{1}{0!} + \frac{1}{1!} + \frac{1}{2!} + \frac{1}{3!} + \frac{1}{4!} + ...$\\
$x_1 = 1, x_2 = 2, x_3 = 2 + 1/2$.
\begin{enumerate}
    \item $x_n$ jest rosnący $x_{n+1}-x_n=\frac{1}{(n+1)!}>0 \iff x_{n_1}>x_n$
    \item $x_n$ jest ograniczony (pamiętając, że $\forall_{n>3} 2^n\leq n!$
    czyli $\frac{1}{4!} < \frac{1}{2^4}$, $\frac{1}{5!} < \frac{1}{2^5}$)...\\
    Dla $n>3$ $x_n=\frac{1}{0!} + \frac{1}{1!} + \frac{1}{2!} + \frac{1}{3!} + \frac{1}{4!} + ...\leq$\\
    $2+\frac{1}{2}+\frac{1}{6}+\frac{1}{2^4}+\frac{1}{2^5}+...+\frac{1}{2^n}$\\
    $\frac{1}{2^4}\cdot\frac{1}{1-\frac{1}{2}}=\frac{1}{2^3}$\\
    Istnieje $lim_{n\rightarrow \infty} x_n = e = 2.7182...$\\
    $sum(1/k!, (k,0,300))\leftarrow$ wolframalpha
\end{enumerate}
\end{pk}

\begin{tw}
    Liczba eulera wyraża się wzorem:
    \begin{center}
    $\lim_{n\rightarrow \infty} (1+\frac{1}{n})^n = e$
    \end{center}
\end{tw}

\begin{tw}
    Niech $a_n$ będzie dowolnym ciągiem takim, że:
    $lim_{n\implies \infty} a_n = \infty$. Wówczas:\\
    \begin{center}
    $lim_{n\implies \infty} (1+\frac{1}{a_n})^{a_n} = e, (1-\frac{1}{a_n})^{a_n} = \frac{1}{e}$
    \end{center}
\end{tw}

\begin{pk}
    $\lim ((1+\frac{1}{2n})^{2n})^{\frac{1}{2}}=e^{\frac{1}{2}}=\sqrt{e}$
\end{pk}

Własność: $lim_{n\rightarrow \infty} a_n = g_1 \land lim_{n\rightarrow \infty} b_n = g_2 \implies lim_{n\rightarrow \infty} (a_n^{b_n})=g_1^{g_2}$

\begin{pk}
    $lim (1-\frac{1}{n})^{n/2}=lim \left((1-\frac{1}{n})^n\right)^{\frac{\frac{n}{2}}{n}}=(\frac{1}{e})^{\frac{1}{2}}=\frac{1}{\sqrt{e}}$
\end{pk}

Wskazówka: $limit\left((1+\frac{1}{2^n})^{n+1},n\rightarrow infty\right)$

\begin{de}
Szereg o wyrazach nieujemnych. Dla dowolnego ciągu $a_1, a_2, \dots, a_n$ o wyrazach nieujemnych,
tworzymy ciąg sum częściowych:
\begin{center}
$S_1=a_1, S_2=a_1+a_2, S_3=a_1+a_2+a_3, \dots, S_N=a_1+a+2+...+a_N$
\end{center}

Przykładowo dla $e$ $S_0=\frac{1}{0!}, S_1=\frac{1}{0!}+\frac{1}{1!}\dots$.\\
Jeżeli ciąg $S_n$ jest zbieżny to piszemy, że:
\begin{center}
$\sum_{n=1}^{\infty} a_n = \lim_{n\rightarrow \infty} S_N$
\end{center}
(granica to suma szeregu)
\begin{center}
    $S_1\leq S_2\leq S_3\leq S_N < M$
\end{center}
\end{de}

\begin{pk}
    $apart(1/(n\cdot(n+1)),n)\leftarrow$ wolframalpha
    \begin{center}
    $\frac{1}{1\cdot 2} + \frac{1}{2\cdot 3}+ \dots + \frac{1}{n(n+1)}=S_N$\\
    $S_1=\frac{1}{2}, S_2=\frac{1}{1\cdot 2}+\frac{1}{2\cdot 3}$, zatem:\\
    $\frac{1}{1\cdot 2} + \frac{1}{2\cdot 3}+ \dots + \frac{1}{n(n+1)}=$\\
    $=\frac{1}{1} - \frac{1}{2} + \frac{1}{2} - \frac{1}{3} + ... + \frac{1}{n} - \frac{1}{n+1}=$\\
    $=1-\frac{1}{n+1}$, finalnie:\\
    $\sum_{n=1}^{\infty} \frac{1}{n(n+1)} = lim_{n\rightarrow \infty} S_N = lim_{n\rightarrow \infty} (1 - \frac{1}{n+1}) = 1$
    \end{center}
\end{pk}

\begin{pk}
    $a+aq+...+aq^n=a\cdot \frac{1-q^{n+1}}{1-q}$, dla $|q|<1$:
    \begin{center}
    $\sum_{n=0}^{\infty} aq^n = lim_{n\rightarrow \infty} a\cdot \frac{1-q^{n+1}}{1-q} = \frac{a}{1-q}$
    \end{center}
\end{pk}

\begin{pk}
    $\sum_{n=0}^{\infty} \frac{1}{n!} = e$
\end{pk}

\begin{pk}
Szereg harmoniczny.
$H_N = \sum_{n=1}^{N} \frac{1}{n}$, $\lim_{N\rightarrow \infty}=\infty$, wolny wzrost do $\infty$\\
$H_{2^{n+1}}=\frac{1}{1} + \frac{1}{2} + \frac{1}{2+1} + \frac{1}{2+2} + \frac{1}{2^2 + 1} + \frac{1}{2^2+2} + \frac{1}{2^2+3} + \frac{1}{2^3} + \frac{1}{2^3+1} + \frac{1}{2^3 + 2} + \frac{1}{2^3 + 3} + \dots + \frac{1}{2^3 + 2^3} + \frac{1}{2^n + 1} + \frac{1}{2^n + 2} + \dots + \frac{1}{2^n+2^n}$\\\\
$\frac{1}{1}+\frac{1}{2}=\frac{3}{2}$\\
$\frac{1}{2+1}+\frac{1}{2+2}\geq 2\cdot \frac{1}{2+2}=\frac{1}{2}$\\
$\frac{1}{2^2+1}+\frac{1}{2^2+2} + ... \geq 4\cdot \frac{1}{2^2 + 2^2}=\frac{1}{2}$\\
$\frac{1}{2^3+1}+\frac{1}{2^3+2} + ... \geq 8\cdot \frac{1}{2^3 + 2^3}=\frac{1}{2}$\\
$\frac{1}{2^n+1}+\frac{1}{2^n+2} + ... \geq 2^{n}\cdot \frac{1}{2^n + 2^n}=\frac{1}{2}$\\
$H_{2^{n+1}}\geq \frac{3}{2} + \frac{1}{2} \cdot n = 1 + \frac{1}{2} (n+1)$\\
$H_{2^{n+1}}\geq 1 + \frac{n+1}{2}$\\
$H_{2^n}\geq 1 + \frac{n}{2}$\\\\
Założmy, że $2^N=k \implies N=log_2()$\\
$H_{k}\geq 1 + \frac{\log_2(k)}{2} \rightarrow \infty$\\
Na mocy twierdenia o dwóch ciągach $H_k \rightarrow \infty$
\end{pk}

Następny wykład - kryteria zbieżności szeregów: kryterium kondensacyjne.

\begin{de}
    Warunek konieczny zbieżności szeregów. Jeżeli $\sum_{n=1}^{\infty} a_n$ jest zbieżny, to $lim_{n\rightarrow \infty} a_n = 0$. (dla $\sum_{n=1}^{\infty} a_n < \infty$).
\end{de}

Szereg $\sum_{n=1}^{\infty} \frac{n}{n+1}$ jest rozbieżny, bo nie jest spełniony warunek konieczny\\ $lim_{n\rightarrow \infty} \frac{n}{n+1}=1$\\
Warunek konieczny nie jest wystarczający.

\section{Wykład czwarty}

\subsection{Kryteria zbieżności szeregów}

\begin{tw}
    Szereg o wyrazach dodatnich jest zbieżny $\iff$ jest ograniczony.\\
    \begin{center}
        $\sum_{n=1}^{\infty} a_n, a_n>0$, czy $S_n = \sum_{n=1}^{\infty} a_n$
    \end{center}
    $S_{N+1}-S_{N}=a_{n+1} > 0, S_N$ - rosnący. Jeżeli $S_N$ jest ograniczony to jest zbieżny.\\
    Wniosek:\\
    $\sum_{n=1}^{\infty} \frac{1}{n^2}$, dla $n\geq 2$:\\
    $S_N=\sum_{n=1}^{N} \frac{1}{n^2}=1+\sum_{n=2}^{N} \frac{1}{n^2}\leq 1 + \sum_{n=2}^{N} \frac{1}{n(n-1)} = 1 + \sum_{n=2}^{N} (\frac{1}{n-1} - \frac{1}{n}) = 2 - \frac{1}{N} \leq 2$
\end{tw}

\begin{tw}
    Kryterium porównawcze. $\sum_{n=1}^{\infty} a_n$, $\sum_{n=1}^{\infty} b_n$ oraz $a_n, b_n > 0$:
    \begin{center}
    Jeżeli $\exists_{n_0}\forall_{n>n_0} a_n\leq b_n$ i $\sum_{n=1}^{\infty} b_n$ jest zbieżny, to $\sum_{n=1}^{\infty} a_n$ jest zbieżny.\\
    \end{center}
\end{tw}

\begin{tw}
    Jeżeli $\sum_{n>n_0}^{\infty} \leq \sum_{n>n_0}^{\infty}$ i $\sum_{n>n_0}^{\infty} a_n = \infty$ ($a_n$ rozbieżny),
    to wówczas $\sum_{n>n_0}^{\infty} b_n = \infty$ ($b_n$ rozbieżny).
\end{tw}

Wniosek: $\sum_{n=1}^{\infty} \frac{1}{n^p} = \infty$ dla $p\leq 1$, bo $\frac{1}{n^p} \geq \frac{1}{n}, \sum_{n=1}^{\infty} \frac{1}{n} = \infty$\\

\begin{tw}
    \begin{center}
    Twierdzenie o zagęszczaniu. Zakładamy, że $a_n\geq 0$ i $a_{n+1} \leq a_n$. Wówczas $\sum_{n=1}^{\infty} a_n$ jest zbieżny $\iff \sum_{n=1}^{\infty} 2^n a_{2^n}$ jest zbieżny.
    \end{center}
\end{tw}

\begin{pk}
    Rozważmy poniższy przykład ciągu:\\
    $a_1+a_2+a_3+a_4+a_5+a_6+a_7+a_8 \implies^{tw. zag} $\\
    $2\cdot a_2 + 4\cdot a_4 + 8\cdot a_8$
\end{pk}

\begin{pk}
    Zastosowanie Tw. o zagęszczaniu.\\
    $\sum_{n=1}^{\infty} \frac{1}{n^p}, p\in\mathbb{R}$ zbieżny $\iff \sum_{n=1}^{\infty} 2^n (\frac{1}{2^n})^p$ jest zbieżny.\\
    $\sum_{n=1}^{\infty} 2^n (\frac{1}{2^n})^p = \sum_{n=1}^{\infty} 2^n \frac{1}{2^{np}} = \sum_{n=1}^{\infty} \frac{1}{2^{np-n}} = \sum_{n=1}^{\infty} \frac{1}{2^{n(p-1)}}$\\
    $\sum_{n=1}^{\infty} \frac{1}{2^{n(p-1)}}$ jest zbieżny dla $p>1$\\
    Wniosek 1: $\sum_{n=1}^{\infty} \frac{1}{n^p}$ jest zbieżny dla $p>1$ i rozbieżny dla $p\leq 1$\\
\end{pk}

\begin{de}
    Kryterium d'Alemberta:
    $\sum_{n=1}^{\infty} a_n, a_n\geq 0$:
    \begin{itemize}
        \item Jeżeli $\exists_{n_0}\forall_{n>n_0} \frac{a_{n+1}}{a_n} \leq q < 1$, to $\sum_{n=1}^{\infty} a_n$ jest zbieżny.
        \item Jeżeli $\exists_{n_0}\forall_{n>n_0} \frac{a_{n+1}}{a_n} \geq 1$, to $\sum_{n=1}^{\infty} a_n$ jest rozbieżny.
        \item Jeżeli $\exists_{n_0}\forall_{n>n_0} \frac{a_{n+1}}{a_n} = 1$, to kryterium d'Alemberta nie rozstrzyga zbieżności.
    \end{itemize}
    Idea d-d: \\
    $lim_{n\rightarrow \infty} sup |\frac{a_{n+1}}{a_n}| = q, q < 1$\\
    $|\frac{a_{n+1}}{a_n}|<q$\\
    $a_{n+1} < a_n q$\\
    $a_n < a_0 q^{n-1}$\\
    $\sum_{n=1}^{\infty} a_n < \sum_{n=1}^{\infty} a_0 q^{n-1}$ - zbieżne
\end{de}

\begin{pk}
    Przykład: $\sum_{n=1}^{\infty}, a_n=\frac{n!}{n^n}$:\\
    $\frac{a_{n+1}}{a_n}=\frac{\frac{(n+1)!}{(n+1)^{n+1}}}{\frac{n!}{n^n}}=\frac{n^n}{(n+1)^n}=\frac{1}{\frac{(n+1)^n}{n^n}}=\frac{1}{(1+\frac{1}{n})^n}$\\
    $\lim_{n\rightarrow \infty} \frac{a_{n+1}}{a_n} = \lim_{n\rightarrow\infty} \frac{1}{(1+\frac{1}{n})^n}=\frac{1}{e}<1$\\
    Z kryterium d'Alemberta szereg jest zbieżny.
\end{pk}

\begin{pk}
    Przykład: $\sum_{n=1}^{\infty} \frac{n!}{2^n}$ jest rozbieżny. ($\sum_{n=1}^{\infty} \frac{n!}{2^n}=\infty$)
\end{pk}

\begin{pk}
    Przykład:
    $\sum_{n=1}^{\infty} \frac{1}{n} = \infty$\\
    $\frac{a_{n+1}}{a_n} = \frac{\frac{1}{n+1}}{\frac{1}{n}}=\frac{n}{n+1}\rightarrow 1$ Kryterium d'Alamberta nic nie powie.
\end{pk}

\begin{pk}
    Przykład: $\sum_{n=1}^{\infty} \frac{1}{n(n+1)}$\\
    $\sum_{n=1}^{N} \frac{1}{n(n+1)} = \sum_{n=1}^{N} (\frac{1}{n}-\frac{1}{n+1})=1-\frac{1}{N+1}\rightarrow 1$\\
    $\frac{a_{n+1}}{a_n} = \dots = \frac{n^2+n}{n^2+3n+2}=1$ Kryterium d'Alamberta nic nie powie.
\end{pk}

Simplify (wolframalpha):\\
$\frac{(n+1)!}{(n+1)^{n+1}}\cdot\frac{n^n}{n!}=\frac{n^n}{(n+1)^n}$\\

discreteplot$(n^2,(n,1,20))$ (wolframalpha)\\

discreteplot$(n^2,{n,1,20})$ (mathematica)

\begin{de}
    Kryterium Cauchy'ego. $\sum_{n=1}^{\infty}, a_n\geq 0$
    \begin{enumerate}
        \item Jeżeli $\exists_{n_0}\forall_{n>n_0} \sqrt[n]{a_n} \leq q < 1$, to $\sum_{n=1}^{\infty} a_n$ jest zbieżny.
        \item Jeżeli $\exists_{n_0}\forall_{n>n_0} \sqrt[n]{a_n} \geq 1$, to $\sum_{n=1}^{\infty} a_n$ jest rozbieżny.
        \item Jeżeli $\exists_{n_0}\forall_{n>n_0} \sqrt[n]{a_n} = 1$, to kryterium Cauchy'ego nie rozstrzyga zbieżności.
    \end{enumerate}
    Idea:\\
    $\sqrt[n]{|a_n|}<q$, $0<q<1$ czyli
    $|a_n|<q^n$ więc
    $a_n<q^n$ zatem
    $\sum_{n=1}^{\infty} q^r$ zbieżny.
\end{de}

\begin{pk}
    $\sum_{n=1}^{\infty} \frac{n^2}{2^n}, a_n=\frac{n^2}{2^n}$\\
    z kryterium Cauchy'ego: $\sqrt[n]{a_n} = \frac{\sqrt[n]{n}^2}{2} = \frac{1}{2} < 1$ - zbieżny
\end{pk}

\begin{pk}
    $\sum_{n=1}^{\infty} \frac{7^n}{2^n+5^n}, a_n=\frac{7^n}{2^n+5^n}$\\
    z kryterium Cauchy'ego: $\sqrt[n]{a_n} = \frac{7}{\sqrt[n]{2^n+5^n}} = \frac{7}{5} > 1$ - rozbieżny
\end{pk}

\begin{pk}
    $\sum_{n=1}^{\infty} \frac{5^n}{5^n+3^n}$\\
    kryterium Cauchy'ego nie działa: $\sqrt[n]{a_n}=\frac{5}{\sqrt[n]{5^n+3^n}} \implies 1$\\
    $a_n=\frac{5^n}{5^n+3^n}$, sprawdźmy warunek konieczny zbieżności:\\
    $lim_{n\rightarrow \infty} a_n = lim_{n\rightarrow \infty} \frac{5^n}{5^n+3^n} = 1 \neq 0$\\
    Ciąg jest rozbieżny.
\end{pk}

\begin{de}
    Zbieżność bezwzględna. Rozważmy szereg $\sum_{n=1}^{\infty} a_n$ o wyrazach dowolnych. 
    Mówimy, że szereg $\sum_{n=1}^{\infty} a_n$ jest zbieżny bezwzględnie jeśli:
    $\sum_{n=1}^{\infty} |a_n|$ jest zbieżny.
\end{de}

\begin{pk}
    $\sum_{n=1}^{\infty} \frac{(-1)^n n^2}{2^n}$ jest zbieżny bezwzględnie.\\
    $\sum_{n=1}^{\infty} |\frac{(-1)^n n^2}{2^n}| = \sum_{n=1}^{\infty} \frac{n^2}{2^n} < \infty$\\
    $\sum_{n=1}^{\infty} \frac{n^2}{2^n}$ jest zbieżny (kryterium d'Alemberta)
\end{pk}

\begin{fakt}
    Badanie zbieżności bezwzględnej szeregu sprowadza się do badania zbieżności szeregu o wyrazach nieujemnych.
\end{fakt}

\begin{tw}
    Zbieżność bezwzględna implikuje zwykłą zbieżność.
    \begin{center}
        $\sum_{n=1}^{\infty} |a_n|$ jest zbieżny $\implies \sum_{n=1}^{\infty} a_n$ zbieżny.
    \end{center}
    \underline{Uwaga}: twierdzenie w drugą stronę nie działa.
\end{tw}

\begin{pk}
    $\sum_{n=1}^{\infty} \frac{(-1)^n}{n}$ nie jest zbieżny bezwzględnie, bo:\\
    $\sum_{n=1}^{\infty} \left|\frac{(-1)^n}{n}\right| = \sum_{n=1}^{\infty} \frac{1}{n} = \infty$\\
\end{pk}

\begin{tw}
    Kryterium Abela (Dirichleta). Niech zachodzą następujące warunki:
    \begin{enumerate}
        \item $a_n\geq 0$
        \item $a_1\geq a_2\geq ... \geq a_n \geq ...$
        \item $lim_{n\rightarrow\infty} a_n = 0$
    \end{enumerate}
    Wówczas:
    \begin{center}
        $\sum_{n=1}^{\infty} a_n(-1)^n$ jest zbieżny
    \end{center}
\end{tw}

\begin{pk}
    Pokażmy, że szereg $\sum_{n=1}^{\infty} \frac{(-1)^n}{n}$ jest zbieżny. Z kryterium Abela:\\
    $a_n=\frac{1}{n}\geq 0 \land a_1\geq a_2\geq ... \geq a_n \geq ... \land lim_{n\rightarrow\infty} a_n = 0$
    Szereg jest zatem zbieżny.
\end{pk}

\begin{pk}
    $\sum_{n=1}^{\infty} a_n (-1)^{n+1}= \sum_{n=1}^{\infty} (-1)(-1)^n a_n = -1 \cdot \sum_{n=1}^{\infty} (-1)^n a_n$\\
    Dalej z kryterium Abela...
\end{pk}

Ciągi to funkcje $\mathbb{N}\rightarrow\mathbb{R}$

\subsection{Funkcje}

Analizujemy funkcje $\mathbb{R}\rightarrow\mathbb{R}$

\begin{de}
    Dziedzina funkcji (domain): $dom(f)$ - zbiór wszystkich $x$ dla których funkcja jest określona.
\end{de}

\begin{de}
    Zbiór wartości (range): $rng(f) = \{f(x): x\in dom(f)\}$
\end{de}

\begin{de}
    Wykres funkcji (graph): $G(f) = \{(x,f(x)): x\in dom(f)\}$
\end{de}

\begin{de}
    Funkcja różnowartościowa (one-to-one function):
    \begin{center}
         $\forall_{x,y\in A} x\neq y \implies f(x)\neq f(y)$
    \end{center}
    Uwaga. Jeśli $f: A\rightarrow B$ jest różnowartościowa, to istnieje dokładnie jedna funkcja\\ $f^{-1}: rng(f)\rightarrow A$, taka że:
    $\forall_{x\in A} f^{-1}(f(x))=x$ oraz $\forall_{y\in rng(f)} f(f^{-1}(y))=y$.
\end{de}

\begin{de}
    Funkcje monotoniczne $f: A\rightarrow B$:
    \begin{enumerate}
        \item $\forall{x,y\in A} (x<y \implies f(x)<f(y))$ - rosnąca
        \item $\forall{x,y\in A} (x<y \implies f(x)>f(y))$ - malejąca
        \item $\forall{x,y\in A} (x<y \implies f(x)\leq f(y))$ - niemalejąca (słabo rosnąca)
        \item $\forall{x,y\in A} (x<y \implies f(x)\geq f(y))$ - nierosnąca (słabo malejąca)
    \end{enumerate}
\end{de}

\begin{de}
    Złożenie funkcji $f: A\rightarrow B$, $g: B\rightarrow C$ wówczas:
    \begin{center}
    $g\circ f: A\rightarrow C$\\
    $(g\circ f)(x)=g(f(x))$
    \end{center}
\end{de}

\begin{pk}
    Rozważmy następujące funkcje i ich złożenia:\\
    $f: \mathbb{R}\rightarrow\mathbb{R}, g: \mathbb{R}\rightarrow[-1,1]$\\
    $f(x) = x^3 + 1, g(y) = sin(y)$\\
    $g\circ f(x) = g(f(x)) = sin(f(x)) = sin(x^3+1)$\\
    Przykład drugi:\\
    $g: \mathbb{R}\rightarrow[-1,1], f: [-1,1]\rightarrow\mathbb{R}$\\
    $f\circ g(x) = f(g(x)) = f(sin(x)) = sin^3(x)+1$
\end{pk}


\end{document}
