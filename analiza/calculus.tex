\documentclass{article}

\usepackage[polish]{babel}
\usepackage[utf8]{inputenc}
\usepackage{polski}
\usepackage[T1]{fontenc}
 
\usepackage[margin=1.5in]{geometry} 

\usepackage{color} 
\usepackage{amsmath}                                                                    
\usepackage{amsfonts}                                                                   
\usepackage{graphicx}                                                             
\usepackage{booktabs}
\usepackage{amsthm}
\usepackage{pdfpages}
\usepackage{wrapfig}
\usepackage{hyperref}

\theoremstyle{definition}
\newtheorem{de}{Definicja}[subsection]

\theoremstyle{definition}
\newtheorem{tw}{Twierdzenie}[subsection]

\theoremstyle{definition}
\newtheorem{pk}{Przykład}[subsection]

\theoremstyle{definition}
\newtheorem*{fakt}{FAKT}

\author{Rafal Wlodarczyk}
\title{Analiza Matematyczna}  
\date{INA 1 Sem. 2023}

\begin{document}

\maketitle

\section{Wykład pierwszy}

Liczby naturalne $\mathbb{N}=\{1,2,3,\dots\}$

\begin{de}
    Zasada indukcji matematycznej. Niech będzie dana własność liczb naturalnych, która czyni zadość warunkom:
    \begin{enumerate}
        \item Liczba 1 posiada tę własność.
        \item Jeżeli liczba $n$ posiada tę własność, to posiada ją również liczba $n+1$.
    \end{enumerate}
    Zasada indukcji matematycznej mówi, że przy tych założeniach każda liczba naturalna posiada tę własność.
\end{de}

\begin{pk}
    $1+2+\dots+n=\frac{n(n+1)}{2}$.
    \begin{enumerate}
        \item $n=1$ $L=1$ $P=\frac{1(1+1)}{2}$
        \item $\forall_{n\geq 1} 1+2+\dots+n=\frac{n(n+1)}{2} \implies 1+2+\dots+n+n+1=\frac{(n+1)(n+2)}{2}$.\\
        Z założenia indukcyjnego mamy:\\
        $1+2+...+n+(n+1)=\frac{n(n+1)}{2}+n+1=(n+1)(\frac{n}{2}+1)=\frac{(n+1)(n+2)}{2}$\\
        Na mocy zasady indukcji matematycznej teza zachodzi $\square$.
    \end{enumerate}
\end{pk}

\begin{pk}
    Nierówność Bernoulli'ego. Niech $a\geq 1$, wówczas dla dowolnego $n$ naturalnego zachodzi nierówność:
    $(1+a)^n \geq 1 + na$
    \begin{enumerate}
        \item $n=1$, $L=(1+a)^1 = 1+a$, $P=1+1\cdot a = 1+ a$, $L=P$, własność zachodzi
        \item $\forall_{n>1} (1+a)^n \geq 1 + na \implies (1+a)^{n+1} \geq 1 + (n+1)\cdot a$\\
        $(1+a)^{n+1}=(1+a)^n\cdot(1+a)\geq^{ind.} (1+na)(1+a)$\\
        $(1+a)^{n+1} \geq 1 + a + na + na^2 = 1 + (n+1)a + na^2 \geq 1 + (n+1)\cdot a$\\
        Na mocy zasady indukcji matematycznej nierówność jest prawdziwa.
    \end{enumerate}
\end{pk}

Liczby Całkowite $\mathbb{Z}=\{0,1,-1,2,-2,3,-3,...\}$

\begin{de}
    Liczby wymierne $\mathbb{Q}$ to liczby postaci:\\
    \begin{center}
        $\frac{p}{q}$, gdzie $p,q\in\mathbb{Z}$ oraz $q\neq0$
    \end{center}
    Zbiór liczb wymiernych jest liniowo uporządkowany, to znaczy każde dwie liczby wymierne można połączyć jednym ze znaków:\\
    $a<b, a>b, a=b$.\\
    Dodawanie $\mathbb{Q}$\\
    $\frac{p_1}{q_1}+\frac{p_2}{q_2}=\frac{p_1q_2 + p_2q_1}{q_1q_2}$\\
    Mnożenie $\mathbb{Q}$\\
    $\frac{p_1}{q_1} \cdot \frac{p_2}{q_2} = \frac{p_1p_2}{q_1q_2}$\\
    Własności:\\
    \begin{enumerate}
        \item Przemienność $a+b=b+a$
        \item Łączność $a+(b+c)=(a+b)+c$
        \item Rozdizelność $(a+b)c=ac+bc$
    \end{enumerate}
\end{de}

Uwaga. Jeżeli ($a<c \land c<b) \iff a<c<b$. Mówimy wtedy, że c leży między liczbami a i b.\\
Z twierdzenia Pitagorasa $1^2+1^2=x^2 \implies x=\sqrt{2}$. D-d niewymierności $\sqrt{2}$ jako ćwiczenie.\\

Własność - zbiór $\mathbb{Q}$ jest zbiorem gęstym.\\
Niech $a,b$ będą dowolnymi liczbami wymiernymi, takimi że $a<b$. Wówczas istnieje liczba $c$ leżąca między liczbami $a$ i $b$.\\
np.: $c=\frac{a+b}{2}$\\

Liczby rzeczywiste $\mathbb{R}$

\begin{de}
    Mówimy, że \underline{zbiór jest ograniczony} jeżeli istnieją takie dwie liczby $m, M$, że:
    \begin{center}
        $\forall_{x\in X} m\leq x\leq M, X\in[m,M]$
    \end{center}
    Uwaga analogicznie ograniczoność z dołu i góry osobno.
\end{de}

\begin{de}
    Kres górny zbioru. Niech $X$ będzie zbiorem ograniczonym z góry.
    \begin{center}
        $\forall_{x\in X} \exists_M x\leq M$
    \end{center}
    Kresem górnym zbioru nazywamy najmniejszą liczbę ograniczającą zbiór $X$ z góry.\\
    $(-\infty, 1)$: kres $1$\\
    $(-\infty, 1) \cup (1,2]$: kres $2$\\
\end{de}

\subsection{Aksjomat Zupełności}
Każdy ograniczony z góry podzbiór liczb rzeczywistych ma kres górny.\\

\begin{de}
    Kres dolny zbioru nazywamy największą liczbą ograniczjącą zbiór $X$ z dołu.
    \begin{center}
        $\forall_{x\in X} \exists_m m\leq X$
    \end{center}
    $(-1, +\infty)$: kres $-1$\\
    $(2, +\infty)$: kres $2$\\
    Kres górny zbioru i kres dolny zbioru to pojęcia dualne.
\end{de}

\subsection{Wartość bezwzględna}

$|a|=\begin{cases}
    a, a\geq 0\\
    -a, a<0
\end{cases}$

\begin{pk}
    Własności:
    \begin{itemize}
        \item $|a|=|-a|$
        \item $|ab|=|a|\cdot|b|$
        \item $|a+b|\leq|a|+|b|$
        \item $|a-b|\leq|a|+|b|$
        \item $|a|-|b| \leq |a-b|$
    \end{itemize}
\end{pk}

\begin{de}
    Współczynnik Newtona. Zakładamy że $n,k$ są liczbami naturalnymi,
    takimi że $n\geq k$. Współczynnik Newtona określam wzorem:
    \begin{center}
        $\binom{n}{k}=\frac{n!}{k!(n-k)!}$
    \end{center}
    Własności:
    \begin{enumerate}
        \item $\binom{n}{0}=\binom{n}{n}=1$
        \item $\binom{n}{k}=\binom{n}{n-k}$
        \item $\binom{n+1}{k}=\binom{n}{k-1}+\binom{n}{k}$
        \item $\binom{n}{k+1}=\binom{n}{k}\cdot \frac{n-k}{k+1}$
    \end{enumerate}
\end{de}

\begin{itemize}
    \item Symbol sumy $\sum$
    \item Symbol iloczynu $\Pi$
\end{itemize}

\begin{de}
    Nierówność Cauchy'ego - Schwarza. Niech $a_1, a_2, \dots, a_n$ oraz $b_1, b_2, \dots, b_n$ będą dowolnymi liczbami rzeczywistymi.
    Wówczas zachodzi nierówność:
    \begin{center}
        $(a_1^2+a_2^2+\dots+a_n^2)(b_1^2+b_2^2+\dots+b_n^2)\geq(a_1b_1+a_2b_2+\dots+a_nb_n)^2$
    \end{center}
    lub równoważnie:
    \begin{center}
        $\sum_{i=1}^{n} a_i^2 \cdot \sum_{i=1}^{n} b_i^2 \geq \left(\sum_{i=1}^{n} a_ib_i\right)^2$
    \end{center}
\end{de}

\section{Wykład drugi}

\begin{de}
    Ciąg liczbowy to funkcja z $\mathbb{N}$ w $\mathbb{R}$. Stosujemy zapis $a_1,a_2,\dots,a_n$. Przykłady:
    \begin{itemize}
        \item $a_n=c+(n-1)d$ - arytmetyczny
        \item $b_n=cq^{n-1}$ - geometryczny
        \item $c_n=n!$
        \item $d_{n+1}=2^{d_n}$ - rekurencyjny
    \end{itemize}
\end{de}

\begin{de}
    Ciąg monotoniczny. 
    \begin{enumerate}
        \item $a_n$ jest rosnący $\iff \forall_{n\in\mathbb{N}} a_n<a_{n+1}$
        \item $a_n$ jest malejący $\iff \forall_{n\in\mathbb{N}} a_n>a_{n+1}$
        \item $a_n$ jest niemalejący $\iff \forall_{n\in\mathbb{N}} a\leq a_{n+1}$
        \item $a_n$ jest nierosnący $\iff \forall_{n\in\mathbb{N}} a\geq a_{n+1}$
    \end{enumerate}
    Analogicznie definiujemy ciąg monotoniczny od pewnego miejsca:
    \begin{enumerate}
        \item $a_n$ jest rosnący od $n_0 \iff \forall_{n>n_0} a_n<a_{n+1}$
    \end{enumerate}
\end{de}

\begin{de}
    Liczbą graniczną ciągu $a_n$ nazywamy liczbę $g$, taką że:
    \begin{center}
        $\forall_{\varepsilon>0}\exists_{n_0}\forall_{n>n_0} |a_n-g|<\varepsilon$
    \end{center}
    Piszemy wtedy: $lim_{n\rightarrow \infty} a_n = g$ lub $a_n\rightarrow g$.\\
    $|a_n-g|<\varepsilon \iff -\varepsilon < a_n -g < \varepsilon \iff g-\varepsilon < a_n < g+\varepsilon$
\end{de}

\section{Wykład trzeci}

\begin{tw}
Twierdzenie (o ciągu monotonicznym i ograniczonym)\\
a) Ciąg rosnący i ograniczony z góry jest zbieżny.\\
$\forall_{n>n_0} a_n\leq a_{n+1}$ i $\forall_{n\in \mathbb{N} a_n< M}$ $\implies \exists \lim_{n\rightarrow \infty} a_n$\\
b) Ciąg malejący i ograniczony z dołu jest zbieżny.\\
$\forall_{n>n_0} a_n\geq a_{n+1}$ i $\forall_{n\in \mathbb{N} a_n> m}$ $\implies \exists \lim_{n\rightarrow \infty} a_n$\\
Idea dowodu:\\
$A=\{a_{n_0+1},a_{n_0+2},\dots,a_n,\dots\} \in \mathbb{R}$\\
A - ograniczony, istnieje kres górny zbioru A\\
Każdy ograniczony podzbiór liczb rzeczywistych ma kres\\
czyli $sup(A)$ (??) $sup(A)=lim_{n\rightarrow \infty a_n}$
\end{tw}

\begin{pk}
Rozważmy następujący ciąg rekurencyjny:
$a_1=\sqrt{2}$ $a_{n+1}=\sqrt{2+a_n}$\\
Idea dowodu indukcyjnego:\\
1. $a_n\leq 2$, indukcja po $n$\\
2. $a_n\leq a_{n+1}$, indukcja po $n$. $a_n\leq a_{n+1}\implies a_{n+1}\leq a_{n+2}$\\
3. $\sqrt{2+a_n}\leq \sqrt{2+a_{n+1}}$ kwadrat stronami rozwiązuje krok indukcyjny\\

$\forall_{n\geq 1} a_n \leq 2 \implies a_{n+1}\leq 2$\\
$a_{n+1}=\sqrt{2+a_n}\leq_{z. ind} \sqrt{2+2}=2$\\

Na mocy twierdzenia o ciągu monotonicznym i ograniczonym istnieje:\\
$lim_{n\rightarrow \infty} a_n = g$\\

\begin{center}
$a_{n+1}=\sqrt{2+a_n}$, $lim_{n\rightarrow \infty} a_n = g = lim_{n\rightarrow \infty} a_{n+1} = g$\\
$g=\sqrt{2+g}$\\
$g^2-g-2=0$\\
$\Delta=9=3^2$\\
$g_1=\frac{1+3}{2}=2$ lub $g_2=\frac{1-3}{2}=-1$, które nie zachodzi, zatem $lim a_n=g_1$
\end{center}
\end{pk}

\begin{de}
Podciąg ciągu\\
Niech $a_n$ będzie dowolnym ciągiem. Niech $n_1, n_2, ... n_k$ będzie pewnym rosnącym ciągiem liczb naturalnych.
Wówczas ciąg $a_{nk}=(a_{n1}, a_{n2}, a_{n3}, ...)$ Nazywamy podciągiem ciągu.
\end{de}

\begin{pk}
Rozważmy następujące przykłady ($\mathbb{N}=\{1,2,3,4,\dots\}$):\\
a) $a_n=(-1)^{n}$, $n\in\mathbb{N}$\\
$a_{2k}=(-1)^{n}=1$, $k\in\mathbb{N}$\\
$(a_2, a_4, a_6, ...)$ - podciąg o wyrazach parzystych.\\
b) $a_{2k-1}=(-1)^{2k-1}=-1, n\in\mathbb{N}$\\
$(a_1, a_3, a_5, ...)$ - podciąg o wyrazach nieparzystych.\\
$S=\{1,-1\}$\\
c) $(1, \frac{1}{2}, 3, \frac{1}{4}, 5, \frac{1}{6},\dots)$\\
$a_{2k-1}=2k-1$ - podciąg o wyr. nieparzystych.\\
$a_{2k}=\frac{1}{2k}$ - podciąg o wyr. parzystych.\\
$S=\{0, \infty\}$
d) $sin(\frac{n\pi}{3})$ - $plot(sin(\frac{n\pi}{3}),(n,1,17)) \leftarrow$ wolframalpha
\end{pk}

\begin{de}
Liczba $s$ jest punktem skupienia ciągu $a_n\iff s$ jest granicą właściwą lub niewłaściwą pewnego podciągu.
Oznaczenie $S$ - zbiór punktów skupienia.\\

Jeśli $lim_{n\rightarrow \infty} a_n = \infty \implies a_n$ ma granicę niewłaściwą $+\infty$
\end{de}

\begin{itemize}
\item $sup()$ - superior - kres górny
\item $inf()$ - inferior - kres dolny
\end{itemize}

\begin{de}
Granica górna ciągu $a_n$ to kres górny granic podciągu $a_n$.\\
$\lim_{n\rightarrow \infty} sup(a_n) = \lim_{n\rightarrow\infty} a_n$
\end{de}

\begin{de}
Granica dolna ciągu $a_n$ to kres dolny granic podciągu $a_n$.\\
$\lim_{n\rightarrow \infty} inf(a_n) = \lim_{n\rightarrow\infty} a_n$
\end{de}

$\lim inf(a_n)\leq \lim sup(a_n)$, równość dla granicy ciągu.

\begin{tw}
Twierdzenie (Bolzano - Weierstrassa). Każdy ciąg ograniczony ma podciąg zbieżny.
\href{https://en.wikipedia.org/wiki/Bolzano%E2%80%93Weierstrass_theorem}{(English Wikipedia)}\\
D-d. $\forall_{n\in\mathbb{N}} m \leq a_n \leq M$ Dzielimy przedział $[m_1,M_1]$ na dwa podprzedziały:
$[m_1, \frac{m_1+M_1}{2}]$, $[\frac{m_1+M_1}{2},M_1]$. Przynajmniej w jednym z przedziałów jest nieskończenie wiele wyrazów ciągu.
Oznaczmy tę połówkę przez $[m_2, M_2]$. Postępujemy tak dalej i mamy:\\
$\forall_{k\in\mathbb{N}} m_1\leq m_k\leq a_{nk} \leq M_k \leq M_1$\\
$M_k$ malejący i ograniczony $\implies$ zbieżny $g_1$\\
$m_k$ rosnący i ograniczony $\implies$ zbieżny $g_2$\\
$g_1=g_2=g$\\
$M_k-m_k=\frac{M_1-m_1}{2}$\\
$M_k\rightarrow g_1; m_k\rightarrow g_2$, ponieważ $\frac{M_1-m_1}{2^k}\rightarrow 0$

\end{tw}

\begin{de}
Ciąg $a_n$ nazywamy ciągiem Cauchy'ego, wtedy i tylko wtedy, gdy:\\
$\forall_{\varepsilon > 0}\exists_{n_0}\forall_{n,m>n_0} |a_n-a_m|<\varepsilon$.
\end{de}

\begin{tw}
Ciąg liczb rzeczywistych jest zbieżny $\iff$ jest ciągiem Cauchy'ego.
\end{tw}

\begin{pk}
$x_n = \frac{1}{0!} + \frac{1}{1!} + \frac{1}{2!} + \frac{1}{3!} + \frac{1}{4!} + ...$\\
$x_1 = 1, x_2 = 2, x_3 = 2 + 1/2$.
\begin{enumerate}
    \item $x_n$ jest rosnący $x_{n+1}-x_n=\frac{1}{(n+1)!}>0 \iff x_{n_1}>x_n$
    \item $x_n$ jest ograniczony (pamiętając, że $\forall_{n>3} 2^n\leq n!$
    czyli $\frac{1}{4!} < \frac{1}{2^4}$, $\frac{1}{5!} < \frac{1}{2^5}$)...\\
    Dla $n>3$ $x_n=\frac{1}{0!} + \frac{1}{1!} + \frac{1}{2!} + \frac{1}{3!} + \frac{1}{4!} + ...\leq$\\
    $2+\frac{1}{2}+\frac{1}{6}+\frac{1}{2^4}+\frac{1}{2^5}+...+\frac{1}{2^n}$\\
    $\frac{1}{2^4}\cdot\frac{1}{1-\frac{1}{2}}=\frac{1}{2^3}$\\
    Istnieje $lim_{n\rightarrow \infty} x_n = e = 2.7182...$\\
    $sum(1/k!, (k,0,300))\leftarrow$ wolframalpha
\end{enumerate}
\end{pk}

\begin{tw}
    Liczba eulera wyraża się wzorem:
    \begin{center}
    $\lim_{n\rightarrow \infty} (1+\frac{1}{n})^n = e$
    \end{center}
\end{tw}

\begin{tw}
    Niech $a_n$ będzie dowolnym ciągiem takim, że:
    $lim_{n\implies \infty} a_n = \infty$. Wówczas:\\
    \begin{center}
    $lim_{n\implies \infty} (1+\frac{1}{a_n})^{a_n} = e, (1-\frac{1}{a_n})^{a_n} = \frac{1}{e}$
    \end{center}
\end{tw}

\begin{pk}
    $\lim ((1+\frac{1}{2n})^{2n})^{\frac{1}{2}}=e^{\frac{1}{2}}=\sqrt{e}$
\end{pk}

Własność: $lim_{n\rightarrow \infty} a_n = g_1 \land lim_{n\rightarrow \infty} b_n = g_2 \implies lim_{n\rightarrow \infty} (a_n^{b_n})=g_1^{g_2}$

\begin{pk}
    $lim (1-\frac{1}{n})^{n/2}=lim \left((1-\frac{1}{n})^n\right)^{\frac{\frac{n}{2}}{n}}=(\frac{1}{e})^{\frac{1}{2}}=\frac{1}{\sqrt{e}}$
\end{pk}

Wskazówka: $limit\left((1+\frac{1}{2^n})^{n+1},n\rightarrow infty\right)$

\begin{de}
Szereg o wyrazach nieujemnych. Dla dowolnego ciągu $a_1, a_2, \dots, a_n$ o wyrazach nieujemnych,
tworzymy ciąg sum częściowych:
\begin{center}
$S_1=a_1, S_2=a_1+a_2, S_3=a_1+a_2+a_3, \dots, S_N=a_1+a+2+...+a_N$
\end{center}

Przykładowo dla $e$ $S_0=\frac{1}{0!}, S_1=\frac{1}{0!}+\frac{1}{1!}\dots$.\\
Jeżeli ciąg $S_n$ jest zbieżny to piszemy, że:
\begin{center}
$\sum_{n=1}^{\infty} a_n = \lim_{n\rightarrow \infty} S_N$
\end{center}
(granica to suma szeregu)
\begin{center}
    $S_1\leq S_2\leq S_3\leq S_N < M$
\end{center}
\end{de}

\begin{pk}
    $apart(1/(n\cdot(n+1)),n)\leftarrow$ wolframalpha
    \begin{center}
    $\frac{1}{1\cdot 2} + \frac{1}{2\cdot 3}+ \dots + \frac{1}{n(n+1)}=S_N$\\
    $S_1=\frac{1}{2}, S_2=\frac{1}{1\cdot 2}+\frac{1}{2\cdot 3}$, zatem:\\
    $\frac{1}{1\cdot 2} + \frac{1}{2\cdot 3}+ \dots + \frac{1}{n(n+1)}=$\\
    $=\frac{1}{1} - \frac{1}{2} + \frac{1}{2} - \frac{1}{3} + ... + \frac{1}{n} - \frac{1}{n+1}=$\\
    $=1-\frac{1}{n+1}$, finalnie:\\
    $\sum_{n=1}^{\infty} \frac{1}{n(n+1)} = lim_{n\rightarrow \infty} S_N = lim_{n\rightarrow \infty} (1 - \frac{1}{n+1}) = 1$
    \end{center}
\end{pk}

\begin{pk}
    $a+aq+...+aq^n=a\cdot \frac{1-q^{n+1}}{1-q}$, dla $|q|<1$:
    \begin{center}
    $\sum_{n=0}^{\infty} aq^n = lim_{n\rightarrow \infty} a\cdot \frac{1-q^{n+1}}{1-q} = \frac{a}{1-q}$
    \end{center}
\end{pk}

\begin{pk}
    $\sum_{n=0}^{\infty} \frac{1}{n!} = e$
\end{pk}

\begin{pk}
Szereg harmoniczny.
$H_N = \sum_{n=1}^{N} \frac{1}{n}$, $\lim_{N\rightarrow \infty}=\infty$, wolny wzrost do $\infty$\\
$H_{2^{n+1}}=\frac{1}{1} + \frac{1}{2} + \frac{1}{2+1} + \frac{1}{2+2} + \frac{1}{2^2 + 1} + \frac{1}{2^2+2} + \frac{1}{2^2+3} + \frac{1}{2^3} + \frac{1}{2^3+1} + \frac{1}{2^3 + 2} + \frac{1}{2^3 + 3} + \dots + \frac{1}{2^3 + 2^3} + \frac{1}{2^n + 1} + \frac{1}{2^n + 2} + \dots + \frac{1}{2^n+2^n}$\\\\
$\frac{1}{1}+\frac{1}{2}=\frac{3}{2}$\\
$\frac{1}{2+1}+\frac{1}{2+2}\geq 2\cdot \frac{1}{2+2}=\frac{1}{2}$\\
$\frac{1}{2^2+1}+\frac{1}{2^2+2} + ... \geq 4\cdot \frac{1}{2^2 + 2^2}=\frac{1}{2}$\\
$\frac{1}{2^3+1}+\frac{1}{2^3+2} + ... \geq 8\cdot \frac{1}{2^3 + 2^3}=\frac{1}{2}$\\
$\frac{1}{2^n+1}+\frac{1}{2^n+2} + ... \geq 2^{n}\cdot \frac{1}{2^n + 2^n}=\frac{1}{2}$\\
$H_{2^{n+1}}\geq \frac{3}{2} + \frac{1}{2} \cdot n = 1 + \frac{1}{2} (n+1)$\\
$H_{2^{n+1}}\geq 1 + \frac{n+1}{2}$\\
$H_{2^n}\geq 1 + \frac{n}{2}$\\\\
Założmy, że $2^N=k \implies N=log_2()$\\
$H_{k}\geq 1 + \frac{\log_2(k)}{2} \rightarrow \infty$\\
Na mocy twierdenia o dwóch ciągach $H_k \rightarrow \infty$
\end{pk}

Następny wykład - kryteria zbieżności szeregów: kryterium kondensacyjne.

\begin{de}
    Warunek konieczny zbieżności szeregów. Jeżeli $\sum_{n=1}^{\infty} a_n$ jest zbieżny, to $lim_{n\rightarrow \infty} a_n = 0$. (dla $\sum_{n=1}^{\infty} a_n < \infty$).
\end{de}

Szereg $\sum_{n=1}^{\infty} \frac{n}{n+1}$ jest rozbieżny, bo nie jest spełniony warunek konieczny\\ $lim_{n\rightarrow \infty} \frac{n}{n+1}=1$\\
Warunek konieczny nie jest wystarczający.

\section{Wykład czwarty}

\subsection{Kryteria zbieżności szeregów}

\begin{tw}
    Szereg o wyrazach dodatnich jest zbieżny $\iff$ jest ograniczony.\\
    \begin{center}
        $\sum_{n=1}^{\infty} a_n, a_n>0$, czy $S_n = \sum_{n=1}^{\infty} a_n$
    \end{center}
    $S_{N+1}-S_{N}=a_{n+1} > 0, S_N$ - rosnący. Jeżeli $S_N$ jest ograniczony to jest zbieżny.\\
    Wniosek:\\
    $\sum_{n=1}^{\infty} \frac{1}{n^2}$, dla $n\geq 2$:\\
    $S_N=\sum_{n=1}^{N} \frac{1}{n^2}=1+\sum_{n=2}^{N} \frac{1}{n^2}\leq 1 + \sum_{n=2}^{N} \frac{1}{n(n-1)} = 1 + \sum_{n=2}^{N} (\frac{1}{n-1} - \frac{1}{n}) = 2 - \frac{1}{N} \leq 2$
\end{tw}

\begin{tw}
    Kryterium porównawcze. $\sum_{n=1}^{\infty} a_n$, $\sum_{n=1}^{\infty} b_n$ oraz $a_n, b_n > 0$:
    \begin{center}
    Jeżeli $\exists_{n_0}\forall_{n>n_0} a_n\leq b_n$ i $\sum_{n=1}^{\infty} b_n$ jest zbieżny, to $\sum_{n=1}^{\infty} a_n$ jest zbieżny.\\
    \end{center}
\end{tw}

\begin{tw}
    Jeżeli $\sum_{n>n_0}^{\infty} \leq \sum_{n>n_0}^{\infty}$ i $\sum_{n>n_0}^{\infty} a_n = \infty$ ($a_n$ rozbieżny),
    to wówczas $\sum_{n>n_0}^{\infty} b_n = \infty$ ($b_n$ rozbieżny).
\end{tw}

Wniosek: $\sum_{n=1}^{\infty} \frac{1}{n^p} = \infty$ dla $p\leq 1$, bo $\frac{1}{n^p} \geq \frac{1}{n}, \sum_{n=1}^{\infty} \frac{1}{n} = \infty$\\

\begin{tw}
    \begin{center}
    Twierdzenie o zagęszczaniu. Zakładamy, że $a_n\geq 0$ i $a_{n+1} \leq a_n$. Wówczas $\sum_{n=1}^{\infty} a_n$ jest zbieżny $\iff \sum_{n=1}^{\infty} 2^n a_{2^n}$ jest zbieżny.
    \end{center}
\end{tw}

\begin{pk}
    Rozważmy poniższy przykład ciągu:\\
    $a_1+a_2+a_3+a_4+a_5+a_6+a_7+a_8 \implies^{tw. zag} $\\
    $2\cdot a_2 + 4\cdot a_4 + 8\cdot a_8$
\end{pk}

\begin{pk}
    Zastosowanie Tw. o zagęszczaniu.\\
    $\sum_{n=1}^{\infty} \frac{1}{n^p}, p\in\mathbb{R}$ zbieżny $\iff \sum_{n=1}^{\infty} 2^n (\frac{1}{2^n})^p$ jest zbieżny.\\
    $\sum_{n=1}^{\infty} 2^n (\frac{1}{2^n})^p = \sum_{n=1}^{\infty} 2^n \frac{1}{2^{np}} = \sum_{n=1}^{\infty} \frac{1}{2^{np-n}} = \sum_{n=1}^{\infty} \frac{1}{2^{n(p-1)}}$\\
    $\sum_{n=1}^{\infty} \frac{1}{2^{n(p-1)}}$ jest zbieżny dla $p>1$\\
    Wniosek 1: $\sum_{n=1}^{\infty} \frac{1}{n^p}$ jest zbieżny dla $p>1$ i rozbieżny dla $p\leq 1$\\
\end{pk}

\begin{de}
    Kryterium d'Alemberta:
    $\sum_{n=1}^{\infty} a_n, a_n\geq 0$:
    \begin{itemize}
        \item Jeżeli $\exists_{n_0}\forall_{n>n_0} \frac{a_{n+1}}{a_n} \leq q < 1$, to $\sum_{n=1}^{\infty} a_n$ jest zbieżny.
        \item Jeżeli $\exists_{n_0}\forall_{n>n_0} \frac{a_{n+1}}{a_n} \geq 1$, to $\sum_{n=1}^{\infty} a_n$ jest rozbieżny.
        \item Jeżeli $\exists_{n_0}\forall_{n>n_0} \frac{a_{n+1}}{a_n} = 1$, to kryterium d'Alemberta nie rozstrzyga zbieżności.
    \end{itemize}
    Idea d-d: \\
    $lim_{n\rightarrow \infty} sup |\frac{a_{n+1}}{a_n}| = q, q < 1$\\
    $|\frac{a_{n+1}}{a_n}|<q$\\
    $a_{n+1} < a_n q$\\
    $a_n < a_0 q^{n-1}$\\
    $\sum_{n=1}^{\infty} a_n < \sum_{n=1}^{\infty} a_0 q^{n-1}$ - zbieżne
\end{de}

\begin{pk}
    Przykład: $\sum_{n=1}^{\infty}, a_n=\frac{n!}{n^n}$:\\
    $\frac{a_{n+1}}{a_n}=\frac{\frac{(n+1)!}{(n+1)^{n+1}}}{\frac{n!}{n^n}}=\frac{n^n}{(n+1)^n}=\frac{1}{\frac{(n+1)^n}{n^n}}=\frac{1}{(1+\frac{1}{n})^n}$\\
    $\lim_{n\rightarrow \infty} \frac{a_{n+1}}{a_n} = \lim_{n\rightarrow\infty} \frac{1}{(1+\frac{1}{n})^n}=\frac{1}{e}<1$\\
    Z kryterium d'Alemberta szereg jest zbieżny.
\end{pk}

\begin{pk}
    Przykład: $\sum_{n=1}^{\infty} \frac{n!}{2^n}$ jest rozbieżny. ($\sum_{n=1}^{\infty} \frac{n!}{2^n}=\infty$)
\end{pk}

\begin{pk}
    Przykład:
    $\sum_{n=1}^{\infty} \frac{1}{n} = \infty$\\
    $\frac{a_{n+1}}{a_n} = \frac{\frac{1}{n+1}}{\frac{1}{n}}=\frac{n}{n+1}\rightarrow 1$ Kryterium d'Alamberta nic nie powie.
\end{pk}

\begin{pk}
    Przykład: $\sum_{n=1}^{\infty} \frac{1}{n(n+1)}$\\
    $\sum_{n=1}^{N} \frac{1}{n(n+1)} = \sum_{n=1}^{N} (\frac{1}{n}-\frac{1}{n+1})=1-\frac{1}{N+1}\rightarrow 1$\\
    $\frac{a_{n+1}}{a_n} = \dots = \frac{n^2+n}{n^2+3n+2}=1$ Kryterium d'Alamberta nic nie powie.
\end{pk}

Simplify (wolframalpha):\\
$\frac{(n+1)!}{(n+1)^{n+1}}\cdot\frac{n^n}{n!}=\frac{n^n}{(n+1)^n}$\\

discreteplot$(n^2,(n,1,20))$ (wolframalpha)\\

discreteplot$(n^2,{n,1,20})$ (mathematica)

\begin{de}
    Kryterium Cauchy'ego. $\sum_{n=1}^{\infty}, a_n\geq 0$
    \begin{enumerate}
        \item Jeżeli $\exists_{n_0}\forall_{n>n_0} \sqrt[n]{a_n} \leq q < 1$, to $\sum_{n=1}^{\infty} a_n$ jest zbieżny.
        \item Jeżeli $\exists_{n_0}\forall_{n>n_0} \sqrt[n]{a_n} \geq 1$, to $\sum_{n=1}^{\infty} a_n$ jest rozbieżny.
        \item Jeżeli $\exists_{n_0}\forall_{n>n_0} \sqrt[n]{a_n} = 1$, to kryterium Cauchy'ego nie rozstrzyga zbieżności.
    \end{enumerate}
    Idea:\\
    $\sqrt[n]{|a_n|}<q$, $0<q<1$ czyli
    $|a_n|<q^n$ więc
    $a_n<q^n$ zatem
    $\sum_{n=1}^{\infty} q^r$ zbieżny.
\end{de}

\begin{pk}
    $\sum_{n=1}^{\infty} \frac{n^2}{2^n}, a_n=\frac{n^2}{2^n}$\\
    z kryterium Cauchy'ego: $\sqrt[n]{a_n} = \frac{\sqrt[n]{n}^2}{2} = \frac{1}{2} < 1$ - zbieżny
\end{pk}

\begin{pk}
    $\sum_{n=1}^{\infty} \frac{7^n}{2^n+5^n}, a_n=\frac{7^n}{2^n+5^n}$\\
    z kryterium Cauchy'ego: $\sqrt[n]{a_n} = \frac{7}{\sqrt[n]{2^n+5^n}} = \frac{7}{5} > 1$ - rozbieżny
\end{pk}

\begin{pk}
    $\sum_{n=1}^{\infty} \frac{5^n}{5^n+3^n}$\\
    kryterium Cauchy'ego nie działa: $\sqrt[n]{a_n}=\frac{5}{\sqrt[n]{5^n+3^n}} \implies 1$\\
    $a_n=\frac{5^n}{5^n+3^n}$, sprawdźmy warunek konieczny zbieżności:\\
    $lim_{n\rightarrow \infty} a_n = lim_{n\rightarrow \infty} \frac{5^n}{5^n+3^n} = 1 \neq 0$\\
    Ciąg jest rozbieżny.
\end{pk}

\begin{de}
    Zbieżność bezwzględna. Rozważmy szereg $\sum_{n=1}^{\infty} a_n$ o wyrazach dowolnych. 
    Mówimy, że szereg $\sum_{n=1}^{\infty} a_n$ jest zbieżny bezwzględnie jeśli:
    $\sum_{n=1}^{\infty} |a_n|$ jest zbieżny.
\end{de}

\begin{pk}
    $\sum_{n=1}^{\infty} \frac{(-1)^n n^2}{2^n}$ jest zbieżny bezwzględnie.\\
    $\sum_{n=1}^{\infty} |\frac{(-1)^n n^2}{2^n}| = \sum_{n=1}^{\infty} \frac{n^2}{2^n} < \infty$\\
    $\sum_{n=1}^{\infty} \frac{n^2}{2^n}$ jest zbieżny (kryterium d'Alemberta)
\end{pk}

\begin{fakt}
    Badanie zbieżności bezwzględnej szeregu sprowadza się do badania zbieżności szeregu o wyrazach nieujemnych.
\end{fakt}

\begin{tw}
    Zbieżność bezwzględna implikuje zwykłą zbieżność.
    \begin{center}
        $\sum_{n=1}^{\infty} |a_n|$ jest zbieżny $\implies \sum_{n=1}^{\infty} a_n$ zbieżny.
    \end{center}
    \underline{Uwaga}: twierdzenie w drugą stronę nie działa.
\end{tw}

\begin{pk}
    $\sum_{n=1}^{\infty} \frac{(-1)^n}{n}$ nie jest zbieżny bezwzględnie, bo:\\
    $\sum_{n=1}^{\infty} \left|\frac{(-1)^n}{n}\right| = \sum_{n=1}^{\infty} \frac{1}{n} = \infty$\\
\end{pk}

\begin{tw}
    Kryterium Abela (Dirichleta). Niech zachodzą następujące warunki:
    \begin{enumerate}
        \item $a_n\geq 0$
        \item $a_1\geq a_2\geq ... \geq a_n \geq ...$
        \item $lim_{n\rightarrow\infty} a_n = 0$
    \end{enumerate}
    Wówczas:
    \begin{center}
        $\sum_{n=1}^{\infty} a_n(-1)^n$ jest zbieżny
    \end{center}
\end{tw}

\begin{pk}
    Pokażmy, że szereg $\sum_{n=1}^{\infty} \frac{(-1)^n}{n}$ jest zbieżny. Z kryterium Abela:\\
    $a_n=\frac{1}{n}\geq 0 \land a_1\geq a_2\geq ... \geq a_n \geq ... \land lim_{n\rightarrow\infty} a_n = 0$
    Szereg jest zatem zbieżny.
\end{pk}

\begin{pk}
    $\sum_{n=1}^{\infty} a_n (-1)^{n+1}= \sum_{n=1}^{\infty} (-1)(-1)^n a_n = -1 \cdot \sum_{n=1}^{\infty} (-1)^n a_n$\\
    Dalej z kryterium Abela...
\end{pk}

Ciągi to funkcje $\mathbb{N}\rightarrow\mathbb{R}$

\subsection{Funkcje}

Analizujemy funkcje $\mathbb{R}\rightarrow\mathbb{R}$

\begin{de}
    Dziedzina funkcji (domain): $dom(f)$ - zbiór wszystkich $x$ dla których funkcja jest określona.
\end{de}

\begin{de}
    Zbiór wartości (range): $rng(f) = \{f(x): x\in dom(f)\}$
\end{de}

\begin{de}
    Wykres funkcji (graph): $G(f) = \{(x,f(x)): x\in dom(f)\}$
\end{de}

\begin{de}
    Funkcja różnowartościowa (one-to-one function):
    \begin{center}
         $\forall_{x,y\in A} x\neq y \implies f(x)\neq f(y)$
    \end{center}
    Uwaga. Jeśli $f: A\rightarrow B$ jest różnowartościowa, to istnieje dokładnie jedna funkcja\\ $f^{-1}: rng(f)\rightarrow A$, taka że:
    $\forall_{x\in A} f^{-1}(f(x))=x$ oraz $\forall_{y\in rng(f)} f(f^{-1}(y))=y$.
\end{de}

\begin{de}
    Funkcje monotoniczne $f: A\rightarrow B$:
    \begin{enumerate}
        \item $\forall{x,y\in A} (x<y \implies f(x)<f(y))$ - rosnąca
        \item $\forall{x,y\in A} (x<y \implies f(x)>f(y))$ - malejąca
        \item $\forall{x,y\in A} (x<y \implies f(x)\leq f(y))$ - niemalejąca (słabo rosnąca)
        \item $\forall{x,y\in A} (x<y \implies f(x)\geq f(y))$ - nierosnąca (słabo malejąca)
    \end{enumerate}
\end{de}

\begin{de}
    Złożenie funkcji $f: A\rightarrow B$, $g: B\rightarrow C$ wówczas:
    \begin{center}
    $g\circ f: A\rightarrow C$\\
    $(g\circ f)(x)=g(f(x))$
    \end{center}
\end{de}

\begin{pk}
    Rozważmy następujące funkcje i ich złożenia:\\
    $f: \mathbb{R}\rightarrow\mathbb{R}, g: \mathbb{R}\rightarrow[-1,1]$\\
    $f(x) = x^3 + 1, g(y) = sin(y)$\\
    $g\circ f(x) = g(f(x)) = sin(f(x)) = sin(x^3+1)$\\
    Przykład drugi:\\
    $g: \mathbb{R}\rightarrow[-1,1], f: [-1,1]\rightarrow\mathbb{R}$\\
    $f\circ g(x) = f(g(x)) = f(sin(x)) = sin^3(x)+1$
\end{pk}

\section{Wykład piąty}

Funkcje elementarne.
\begin{enumerate}
    \item $f(x)=ax+b$ - funkcja liniowa
    \item $f(x)=ax^2+bx+c$ - funkcja kwadratowa
    \item $W(x)$ - wielomian (wymierna)
    \item $f(x)=a^x$, $a>0$ - funkcja wykładnicza\\
    $a^b\cdot a^c=a^{b+c};'' (a^b)^c=a^{b\cdot c}$
    \item $f(x)=log_a(x)$, $a>0$ - funkcja logarytmiczna, odwrotna do $f(x)=a^x$\\
    $log_a(x\cdot y)=log_a(x)+log_a(y);\\ log_a(x^y)=ylog_(x)$\\
    Wzór na zamianę podstawy logarytmu:\\
    $log_a(x)=\frac{log_b(x)}{log_b(a)}$
    \item $e$ - Liczba Eulera - $e\approx 2.7172$\\
    $log_e(x)=ln(x)$\\
    $log_a(x)=ln(x)\cdot log_a(e)$
\end{enumerate}

\subsection{Trygonometria}
\begin{center}
$e^{it}=cost + isint$\\
$cost = Re(e^{it})$\\
$sint = Im(e^{it})$
\end{center}

Szeregi liczby Eulera:
\begin{itemize}
    \item $\sum_{k=0}^{\infty} \frac{(i-t)^k}{k!}=e^{it}$
    \item $e^x = \sum_{k=0}^{\infty} \frac{x^k}{k!}$
    \item $\sum_{k=0}^{\infty} \frac{1}{k!} [sum(1\/k!, (k,0,1000))]$
\end{itemize}

Zobaczmy wzór:

$cos(x+y)=cos(x)cos(y)-sin(x)sin(y)$

$e^{ix}\cdot e^{iy} = e^{ix + iy}=e^{i(x+y)}=_{def}=cos(x+y)+isin(x+y)$

$(cosx + isinx)(cosy + isiny)=(cosx) (cosy) + (cosx) (sinx) i + (sinx) (cosy) + i^2 sinx siny$

$((cosx)+(cosy)-(sinx)(siny))+i((cosx)(siny)+(sinx)(cosy))$

$Re = Re, Im=Im$, a zatem d-d.\\

Funkcje $tg(x)$, $ctg(x)$, $tg(x)=\frac{sin(x)}{cos(x)}, ctg(x)=\frac{cos(x)}{sin(x)}$

$plot(tan(x),(x,-20,20))$\\

\subsection{Funkcje odwrotne do trygonometrycznych}

$sin(x)$ w $x\in[\frac{-\pi}{2},\frac{\pi}{2}]$ jest bijekcją, dzięki czemu można zdefiniować funkcję odwrotną.

\begin{itemize}
    \item $arcsin(x): [-1,1]\rightarrow[\frac{-\pi}{2},\frac{\pi}{2}]$
    \item $arccos(x): [-1,1]\rightarrow[0,\pi]$
    \item $arctg(x): \mathbb{R}\rightarrow (\frac{-\pi}{2},\frac{\pi}{2})$
    \item $arcctg(x): \mathbb{R}\rightarrow (0,\pi)$
\end{itemize}

\subsection{Funkcje hiperboliczne}

$sinh(x)=\frac{e^x-e^{-x}}{2}$, $cosh(x)=\frac{e^x+e^{-x}}{2}$\\

Jedynka hiperboliczna:

\begin{center}
    $cos^2h-sin^2h(x)=1$
\end{center}

D-d:
$cosh^2{x}-sinh^2{x}=(\frac{e^x+e^{-x}}{2})^2-(\frac{e^x-e^{-x}}{2})^2=$

$\frac{e^x\cdot e^x + e^{-x}\cdot e^{-x} + 2e^x\cdot e^{-x}}{4}-\frac{e^x\cdot e^x + e^{-x}\cdot e^{-x} - 2e^x\cdot e^{-x}}{4}=$

$\frac{4 e^x \cdot e^{-x}}{4}=1$\\

Definicja $tgh, ctgh$:
\begin{itemize}
    \item $tgh(x)=\frac{sinh(x)}{cosh(x)}$
    \item $ctgh(x)=\frac{cosh(x)}{sinh(x)}$
\end{itemize}

\subsection{Funkcje sigmoidalne}

\begin{enumerate}
    \item funkcja logistyczna $\sigma(x)=\frac{1}{1+e^{-x}}$, $\sigma(x): \mathbb{R}\rightarrow[0,1]$\\
    Uogólniona:

    \begin{center}
        $f(x)=\frac{1}{(1+e^x)^\alpha}, \alpha>0$
    \end{center}    
    \item tangens hiperboliczny $f(x)=tgh(x)$
    \item arcus tangens hiperboliczny $f(x)=arctg(x)$
    \item error function - funkcja błędu
\end{enumerate}

$\sigma(x)=\frac{1}{2} + \frac{1}{2}\cdot tgh(\frac{x}{2})$

\subsection{Funkcje okresowe}

Definicja $A$ - dziedzina $f$: $\exists_T$ takie, że $\forall_{x\in A} f(x+T)=f(x)$\\

\subsection{Funkcje egzotyczne}

\begin{enumerate}
\item $[x]= \max \{k\in\mathbb{Z}, k\leq x\}$ - cz. całkowita $x$. Najw. całkowita nieprzekraczająca $x$.

\item $\lfloor x \rfloor = \max \{k\in\mathbb{Z}, k\leq x\}$ - podłoga liczby $x$ (to samo co część całkowita)

\item $\lceil x \rceil = \min \{k\in\mathbb{Z}, k\leq x\}$ - sufit liczby $x$

\item $sgn(x)=\begin{cases}
    1, x>0\\
    0, x=0\\
    -1, x<0
\end{cases}
$
\end{enumerate}

$[5.5]=5$, $[4.7]=4$, $[-3.4]=-4$

\subsection{(Heine) Granica funkcji}

Zakładamy, że istnieje $\Delta>0$ taka, że $f$ jest określona na $(a-\Delta,a)\cup(a,a+\Delta)$ (sąsiedztwie punktu $a$).
\begin{center}
    $\lim_{x\rightarrow a} f(x) = g \iff \forall_{x_n\implies a, x_n\neq a} lim_{n\rightarrow \infty} f(x_n) = g$
\end{center}

\begin{pk}
    Policzmy granicę następującej funkcji:\\
    $\lim_{x\rightarrow 0} x^2 = 0$\\
    Weźmy $x_n \rightarrow 0; x_n \neq 0$:
    $\lim_{n\rightarrow \infty} x_n^{2} = lim_{n\rightarrow \infty}\cdot lim_{n\rightarrow \infty} = 0\cdot 0 = 0$
\end{pk}

\begin{pk}
    Policzmy granicę następującej funkcji:\\
    $\lim_{x\rightarrow 1} \frac{x^2-1}{x-1} = \left[\frac{0}{0}\right] =$
    $=\lim_{x\rightarrow 1} \frac{(x-1)(x+1)}{x-1} = \lim_{x\rightarrow 1} \frac{x+1}{1} = 2$
\end{pk}

\begin{pk}
    $\lim_{x\rightarrow 0} \frac{sin(x)}{x} = 1$ - dowód z tw. o trzech funkcjach.
\end{pk}

\subsection{(Cauchy) Granica funkcji}

\begin{center}
    $\lim_{x\rightarrow a} f(x)=g \iff \forall_{\varepsilon>0} \exists_{\delta>0}\forall_{x\in A}$
    $ 0<|x-a|<\delta \implies |f(x)-g|< \varepsilon$
\end{center}

Symbolu $\lim_{x\rightarrow a} f(x)$ - używamy również na oznaczenie granicy niewłaściwej.\\\\
Przykłady:
\begin{itemize}
    \item $\lim_{x\rightarrow 0} \frac{0}{x^2} = \left[\frac{1}{0^{+}}\right] = \infty$
    \item $\lim_{x\rightarrow 0} \frac{1}{x}$ nie istnieje
    $x'_n = \frac{1}{n} \rightarrow 0, x''_n = \frac{-1}{n} \rightarrow 0$, ale
    $f(x'_n) \rightarrow \infty, f(x''_n) \rightarrow -\infty$
    \item $\lim_{x\rightarrow 0} sin(\frac{1}{x})$ nie istnieje
    $x'_n = \frac{1}{2\pi n} \rightarrow 0, x''_n = \frac{1}{2\pi n + \frac{\pi}{2}} \rightarrow 0$, ale
    $f(x'_n) = 0, f(x''_n) = 1$
\end{itemize}

\subsection{Granice Jednostronne}

Granica prawostronna:
\begin{center}
    $\lim_{x\rightarrow a^{+}} f(x)=g \iff \forall_{x_n\rightarrow a, x_n\neq a, x_n>a} \lim_{n\rightarrow \infty} f(x_n)=g$
\end{center}
Granica lewostronna:
\begin{center}
    $\lim_{x\rightarrow a^{-}} f(x)=g \iff \forall_{x_n\rightarrow a, x_n\neq a, x_n<a} \lim_{n\rightarrow \infty} f(x_n)=g$
\end{center}
$\lim_{x\rightarrow 1^{+}} |x^2-x|$ - $limit(abs(x^2-x),x\rightarrow 1,assumptions\rightarrow rightarrow x>1)\leftarrow$ wolfram\\
$Limit\{Abs[x^2-x],x\rightarrow 1, assumptions\rightarrow x>1\}\leftarrow$ mathematica

\subsection{Granica w nieskończoności}

$x_n\rightarrow \infty \lim_{n\rightarrow \infty} f(x_n)$\\
$\lim_{x\rightarrow \infty} f(x)=g \iff \forall_{n\rightarrow \infty} f(x_n)=g$

\begin{pk}
    Zobaczmy granice w nieskończoności:
    \begin{itemize}
        \item $\lim_{x\rightarrow \infty} e^x = e^{\infty} = \infty$
        \item $\lim_{x\rightarrow \infty} a^x = \infty$, o ile $a>1$
        \item $\lim_{x\rightarrow \infty} ln(x) = \infty$
    \end{itemize}
\end{pk}

\begin{pk}
    $\lim_{x\rightarrow -\infty} e^x = \begin{cases}
        x=-t\\
        t\rightarrow \infty
    \end{cases}=
    \lim_{t\rightarrow \infty} e^{-t} = \lim_{t\rightarrow \infty} \frac{1}{e^t} = 0
    $
\end{pk}

\subsection{Twierdzenie o arytmetyce granic}

\begin{enumerate}
    \item $\lim_{x\rightarrow a} (f(x)+g(x))=\lim_{x\rightarrow a} f(x) + \lim_{x\rightarrow a} g(x)$
    \item $\lim_{x\rightarrow a} (f(x)\cdot g(x))=\lim_{x\rightarrow a} f(x) \cdot \lim_{x\rightarrow a} g(x)$
\end{enumerate}

\subsection{Twierdzenie o trzech funkcjach}
Zakładamy, że $f(x)\leq g(x)\leq h(x)$ oraz $\lim_{x\rightarrow a} f(x) = \lim_{x\rightarrow a} h(x) = g$, wtedy:
\begin{center}
    $\lim_{x\rightarrow a} g(x) = g$
\end{center}

\begin{pk}
    $\lim \left(x\cdot sin\left(\frac{1}{x}\right)\right)=0$\\
    $0\leq |x\sin\left(\frac{1}{x}\right)|\leq|x|\cdot 1$
\end{pk}

\begin{tw}
Definicja z granic ciągów $\lim_{n\rightarrow \infty} a_n = 0 \iff \lim_{n\rightarrow \infty} |a_n| = 0$,
przenosi się na granice funkcji:
\begin{center}
    $\lim_{x\rightarrow a} f(x) = 0 \iff \lim_{x\rightarrow a} |f(x)| = 0$
\end{center}
\end{tw}

\subsection{Notacja duże O, notacja asymptotyczna}

Mamy dwa ciągi $a(n)$, $b(n)$. Mówimy że:
\begin{center}
     $a(n)=O\left(b\left(n\right)\right)\iff \exists_C \exists_{n_0} \forall_{n>n_0} \left|a(n)\right|\leq C\left|b(n)\right|$
\end{center}

\begin{pk}
    Przykłady notacji big O:\\
    \begin{itemize}
        \item $a(n)=n^2 - \frac{1}{2} n = O(n^2)$
        \item $b(n)=\left(\frac{1}{2}\right) n^2 + n = O(n^2)$\\
        $\forall_{n\geq 1} \frac{1}{2} n^2 + n \leq \frac{1}{2} n^2 + n^2 = \frac{3}{2} n^2, C=\frac{3}{2}$
        \item $c(n)=a_2\cdot n^2 + a_1\cdot n + a_0$, $a_0,a_1,a_2 \in \mathbb{R}, a_2\neq 0$\\
        $c(n)=O(n^2), c=|a_2|+|a_1|+|a_0|$   
    \end{itemize}
\end{pk}

\begin{tw}
    $f(n) = O(g(n)) \iff \lim \sup_{n\rightarrow \infty} \left|\frac{f(n)}{g(n)}\right|<\infty$
\end{tw}
Zastosowania:\\
$n^3-n^2+1=O(n^3)$, ponieważ:\\
$\lim \sup_{n\rightarrow \infty} \left|\frac{n^3-n^2+1}{n^3}\right|=1<\infty$

\begin{pk}
    Przykład ambitny:\\
    Ustalmy $k$ - stała:
    $\binom{n}{k}=O(n^k)$ - dowód jako zadanie z (*).
\end{pk}

Kolejno:\\
$n^2+n=O(n^2), n^2+n=O(n^3)$ - na interesuje najmniejsze O

\begin{de}
    Mówimy, że:
    \begin{center}
        $f(n) = \Theta(g(n))\iff f(n)=O(g(n)) \land g(n)=O(f(n))$
    \end{center}
\end{de}

\begin{pk}
    $\frac{1}{2} n^2 + n = \Theta(n^2)$
\end{pk}

\begin{tw}
$\left(\lim \left|\frac{f(n)}{g(n)}\right|=g \land 0<g<\infty \right)\implies f(n)=\Theta(g(n))$
\end{tw}

Oraz kolejno (zaawansowane):\\
$\binom{n}{k}=\Theta (n^k)$\\
$\binom{n}{k}=\frac{1}{k!} n^k + \Theta(n^{k-1})$

\section{Wykład szósty}

\begin{pk}
    Policzmy granicę:\\\\
    $\lim_{x\rightarrow 0^+} \frac{1}{x}$, podstawiając $\frac{1}{x}=t, t\rightarrow\infty$ zatem $\lim_{t\rightarrow \infty} t $\\\\
    $\lim_{x\rightarrow 0^-} \frac{1}{x}$, podstawiając $\frac{1}{x}=t, t\rightarrow -\infty$ zatem $\lim_{t\rightarrow -\infty} t $
\end{pk}

\begin{pk}
    Pokażmy, że $\lim_{x\rightarrow 0} (1+x)^{\frac{1}{x}} = e$\\
    Granica prawostronna:\\
    $\lim_{x\rightarrow 0+} (1+x)^{\frac{1}{x}}$, podstawienie $x=\frac{1}{t}$,
    $\lim_{t\rightarrow \infty} (1+\frac{1}{t})^{t}=e$\\
    Granica lewostronna: \\
    $\lim_{x\rightarrow 0-} (1+x)^{\frac{1}{x}}$, podstawienie $x=\frac{1}{t}$,
    $\lim_{t\rightarrow -\infty} (1+\frac{1}{t})^{t}$, podstawienie $t=-s$,\\
    $\lim_{s\rightarrow \infty} (1-\frac{1}{s})^{-s}=\frac{1}{e^{-1}}=e$
\end{pk}

\subsection{Asymptoty}

\begin{de}
    Prosta $x=a$ jest asymptotą pionową lewostronną funkcji $f$ w punkcie $a$,
    jeżeli zajdzie jeden z warunków:
    \begin{center}
        $\lim_{x\rightarrow a^-} f(x)=-\infty$ lub $\lim_{x\rightarrow a^-} f(x)=+\infty$
    \end{center}
    Analogicznie definiujemy asymptotę pionową prawostronną dla $x\rightarrow a^+$.
\end{de}

\begin{de}
    Prosta jest asymptotą pionową jeżeli jest jednocześnie asymptotą pionową lewostronną i prawostronną.
\end{de}

\begin{pk}
    Asymptoty pionowe mogą wystąpić w punktach poza dziedziną funkcji:\\
    $f(x)=\frac{1}{x+1}, \lim_{x\rightarrow -1^+} f(x) = \infty, \lim_{x\rightarrow -1^-} f(x) = -\infty \implies$ prosta $x=-1$ jest asymptotą pionową obustronną funkcji $f(x)$.
\end{pk}

\begin{de}
    Prosta $y=ax+b$ jest asymptotą ukośną funkcji $f$ w $\infty$ wtedy i tylko wtedy, gdy:
    \begin{center}
        $\lim_{x\rightarrow \infty} (f(x)-(ax+b))=0$. 
    \end{center}
\end{de}

\begin{tw}
    Prosta $y=ax+b$ jest asymptotą ukośną funkcji $f$ w $\infty$ wtedy i tylko wtedy, gdy:
    \begin{center}
        $a=\lim_{x\rightarrow \infty} \frac{f(x)}{x}$\\
        $b=\lim_{x\rightarrow \infty} f(x)-ax$
    \end{center}
    Jeżeli te granice nie istnieją to funkcja nie posiada asymptoty ukośnej w $\infty$.\\\\
    Idea dowodu:\\
    \begin{center}
    $\lim_{x\rightarrow \infty} (f(x)-(ax+b))=0 \implies \lim_{x\rightarrow \infty} \frac{(f(x)-(ax+b))}{x}=\frac{0}{\infty}=0$\\
    $\lim_{x\rightarrow \infty} (\frac{f(x)}{x}-a-\frac{b}{x})\implies a = \lim_{f\rightarrow \infty} \frac{f(x)}{x}$\\
    $b=\lim_{x\rightarrow \infty} (f(x)-(ax))$
    \end{center}
\end{tw}

\begin{tw}
    Prosta $y=ax+b$ jest asymptotą ukośną funkcji $f$ w $- \infty$ wtedy i tylko wtedy, gdy:
    \begin{center}
        $a=\lim_{x\rightarrow -\infty} \frac{f(x)}{x}$\\
        $b=\lim_{x\rightarrow -\infty} f(x)-ax$
    \end{center}
\end{tw}

\begin{pk}
    Narysujmy wyres funkcji: $f(x)=\frac{x^2+1}{x-1}$, pokażmy, że prosta $y=x+1$ jest asymptotą ukośną $f(x)$ w $\pm \infty$.\\
    $a_{+}= \lim_{x\rightarrow \infty} \frac{f(x)}{x}=\lim_{x\rightarrow \infty} \frac{x+\frac{1}{x}}{x-1}=1$\\
    $a_{-}= \lim_{x\rightarrow -\infty} \frac{f(x)}{x}=\lim_{x\rightarrow -\infty} \frac{x+\frac{1}{x}}{x-1}=1$\\
    $b_{+}= \lim_{x\rightarrow \infty} (f(x)-ax) = \lim_{x\rightarrow \infty} \frac{x^2+1}{x-1} - x = \lim_{x\rightarrow \infty} \frac{x^2+1-x^2+x}{x-1} = 1$\\
    $b_{-}= \lim_{x\rightarrow -\infty} (f(x)-ax) = \lim_{x\rightarrow -\infty} \frac{x^2+1}{x-1} - x = \lim_{x\rightarrow -\infty} \frac{x^2+1-x^2+x}{x-1} = 1$\\
\end{pk}

\subsection{Ciągłość funkcji}

\begin{de}
    Ciągłość funkcji (Heinego). Zakładamy, że $f$ jest określona na pewnym otoczeniu punktu $a$,
    tzn. na przedziale $(a-\Delta, a+\Delta)$ dla pewnego ustalonego $\Delta>0$. Mówimy, że $f$ jest ciągła w punkcie $a$ wtedy i tylko wtedy, gdy:
    \begin{center}
        $f(a)=lim_{x\rightarrow a} f(x)$
    \end{center}
\end{de}

\begin{pk}
    Zobaczmy jak w praktyce można zastosować te definicje:\\
    $\lim_{x\rightarrow a} f(x) = f(\lim_{x\rightarrow a} x) = f(a)$\\
    $\lim_{x\rightarrow 1} (x+1)^2 = (\lim_{x\rightarrow 1} (x)+1)^2 = (1+1)^2=2^2=4$
\end{pk}

\begin{pk}
    Przykłady:
    \begin{itemize}
        \item $f(x)=x^2$, $dom(f)=\mathbb{R}$ jest ciągła w każdym punkcie dziedziny.
        \item $f(x)=sin(1/x)$, $dom(f)=\mathbb{R}-\{0\}$ jest ciągła w każdym punkcie dziedziny.
    \end{itemize}
\end{pk}

\begin{tw}
    Ciągłość prawostronna (Heinego). Zakładamy, że $f$ jest określona na pewnym prawostronnym otoczeniu punktu $a$,
    tzn. na przedziale $(a,a+\Delta)$ dla pewnego ustalonego $\Delta>0$. Mówimy, że $f$ jest prawostronnie ciągła w punkcie $a$,
    wtedy i tylko wtedy, gdy:
    \begin{center}
        $f(a) = \lim_{x\rightarrow a^{+}} f(x)$
    \end{center}
    Analogicznie definiujemy ciągłość lewostronną dla $x\rightarrow a^{-}, [a-\Delta,a]$
\end{tw}

\begin{tw}
    Funkcja jest ciągła w punkcie $a$ jeżeli jest jednocześnie ciągła prawostronnie i lewostronnie.
\end{tw}

\begin{pk}
    Zbadaj ciągłość podanej funkcji w $0$ w zależności od parametru $a$:\\
    $f(x)=
    \begin{cases}
        x+a, \text{dla } x\geq 0\\
        x^2+1, \text{dla } x < 0
    \end{cases}$\\\\
    Liczymy:
    $\lim_{x\rightarrow 0^{-}} f(x) = \lim_{x\rightarrow 0^{-}} (x^2+1) = 1$\\
    $\lim_{x\rightarrow 0^{+}} f(x) = \lim_{x\rightarrow 0^{+}} (x+a) = a$\\
    $a=1$
\end{pk}

\begin{de}
    Ciągłość funkcji (Cauchy'ego). Funkcja $f$ jest ciągła w punkcie $a$ $\iff$
    \begin{center}
        $\forall_{\varepsilon>0} \exists_{\delta>0} \forall_{x} |x-a|<\delta \implies |f(x)-f(a)|<\varepsilon$
    \end{center}
\end{de}

\begin{tw}
    Jeżeli $f, g$ są ciągłe w punkcie $x_0=a$, to wówczas:
    \begin{enumerate}
        \item $f(x)\pm g(x)$
        \item $f(x) \cdot g(x)$
        \item $\frac{f(x)}{g(x)}$, o ile $g(a)\neq 0$
    \end{enumerate}
    są ciągłe w punkcie $x_0=a$.
\end{tw}
Wniosek. Wielomiany i funkcje wymierne są ciągłe w swojej dziedzine.
Funkcje trygonometryczne są ciągłe w swojej dziedzine.

\begin{tw}
    Złożenie funkcji ciągłych $f,g$ jest funkcją ciągłą.
    \begin{center}
        $f$ ciągła $\land$ $g$ ciągła $\implies$ $f\circ g$ ciągła
    \end{center}
\end{tw}

Zobaczmy przykład:
$\lim_{x\rightarrow a} \left(cos\left(\frac{1}{x}\right)\right)=cos\left(\lim_{x_\rightarrow a} \frac{1}{x}\right) = cos\left(\frac{1}{a}\right)$

\subsection{Mnożenie szeregów}

$(1+2x+x^2)(-1+3x+x^2+x^3)=(1+2\cdot 1 + 1\cdot 3)x^3 + (1\cdot 1 + 2\cdot 1) x^4 + (1\cdot 1) x^5$
$(a_0+a_1x+a_2x^2)(b_0+b_1x+b_2x^2+b_3x^3)=a_0b_01 + (a_0b_1 + a_1b_0) x + (a_0b_2+a_1b_1 + a_2b_0) x^2 + ...$
$=c_0\cdot 1 + c_1\cdot x + c_2\cdot x^2$\\

$c_0= \sum_{k=0}^0 a_k b_{0-k}$; $c_1= \sum_{k=0}^1 a_k b_{1-k}$; $c_2= \sum_{k=0}^2 a_k b_{2-k}$

\begin{tw}
    Twierdzenie Cauchy'ego o mnożeniu szeregów. \\
    Zakładamy, że $\sum_{n=0}^{\infty} |a_n| < \infty, \sum_{n=0}^{\infty} |b_n | < \infty$, wtedy$c_0 \sum_{k=0}^0 a_k b_{0-k}$:
    \begin{center}
        $\sum_{n=0}^{\infty} a_n \cdot \sum_{n=0}^{\infty} b_n = \sum_{n=0}^{\infty} c_n$, gdzie
        $c_n=\sum_{k=0}{n} a_k b_{n-k}$ - dyskretny splot
    \end{center}
\end{tw}

\subsection{Funkcja exp(x)}

\begin{de}
    Niech $x\in\mathbb{R}$:
    \begin{center}
        $\exp(x)=\sum_{n=0}^{\infty} \frac{x^n}{n!}$
    \end{center}
    $\exp(x)$ jest poprawnie zdefiniowna.
    $a_n=\frac{x^n}{n!}, \lim_{n\rightarrow \infty} | \frac{a_{n+1}}{a_n} | =$
    $\lim_{n\rightarrow \infty} | \frac{x^{n+1}}{(n+1)!} \cdot \frac{(n)!}{x^n} = \lim_{n\rightarrow \infty} |\frac{x}{n+1} | = 0\rightarrow$ jest zbieżność bezwzględna $x\in\mathbb{R}$.
\end{de}

\begin{pk}
Dyskretny splot $\exp$:\\
$\exp(x)+\exp(y)=\exp(x+y)$\\
$\sum_{n=0}^{\infty} \frac{x^n}{n!} + \sum_{n=0}^{\infty} \frac{y^n}{n!} = \sum_{n=0}^{\infty} c_n$\\\\
$c_n=\sum_{k=0}^{n} a_k \cdot b_{n-k}$ dyskretny splot szeregów $a_n, b_n$\\
zatem:\\\\
$c_n = \sum_{n=0}^{\infty} \frac{x^k}{k!} \cdot \frac{y^{n-k}}{(n-k)!} = \sum_{n=0}^{\infty} \binom{n}{k} \frac{1}{n!} x^k y^{n-k}=$
$=\frac{1}{n!} \sum_{k=0}^{n} \binom{n}{k} x^k y^{n-k} = \frac{1}{n!} (x+y)^n$\\
\end{pk}

\begin{pk}
    $\forall_{x\in\mathbb{R}}  exp(x) > 0$
    \begin{enumerate}
        \item $x>0$ $\exp(x)=\sum_{n=0}^{\infty} \frac{x^n}{n!} > 0$, poniweaż  $(1+\sum_{n=1}^{\infty} \frac{x^n}{n!}=1)$
        \item $\exp(x)\cdot \exp(-x) = \exp(x+(-x)) = \exp(0) = 1$\\
        $\exp(-x) =\frac{1}{\exp(x)} > 0$
    \end{enumerate}
\end{pk}

\begin{pk}
    Funkcja $\exp(x)$ jest ciągła:\\
    $\lim_{x\rightarrow x_0}=\lim_{h\rightarrow 0} \exp(x_0+h) = \lim_{h\rightarrow 0} \exp(x_0)\exp(h)$\\
    $= \exp(x_0)\lim_{h\rightarrow 0} \exp(h) = \exp(x_0) \cdot 1$
\end{pk}

\section{Wykład VII}

Konfa ... 

\section{Wykład VIII}

\subsection{Suma i iloczyn pochodnych}

\begin{tw}
    Pochodna sumy jest sumą pochodnych:
    \begin{center}
        $(f(x)+g(x))'=f'(x)+g'(x)$
    \end{center}
    D-d.
    $(f(x)+g(x))'=$
    $lim_{h\rightarrow 0} \frac{f(x+h)+g(x+h)-f(x)-g(x)}{h}=$
    $lim_{h\rightarrow 0} \frac{f(x+h)-f(x)}{h} - \frac{g(x+h)-g(x)}{h}=^*$
    $f'(x)+g'(x)$ ($*$ granica sumy jest sumą granic)
\end{tw}

\begin{tw}
    CHAIN RULE. Pochodna iloczynu funkcji wyraża się wzorem:
    \begin{center}
        $(f(x)\cdot g(x))'=f'(x)\cdot g(x) + f(x)\cdot g'(x)$
    \end{center}
    D-d.
    $(f(x)\cdot g(x))'=$
    $lim_{h\rightarrow 0} \frac{f(x+h)\cdot g(x+h)-f(x)\cdot g(x)}{h}=$
    $lim_{h\rightarrow 0} \frac{(f(x+h)-f(x)+f(x))\cdot g(x+h)-f(x)\cdot g(x)}{h}=$
    $lim_{h\rightarrow 0} \frac{(f(x+h)-f(x))g(x+h)}{h} + \frac{f(x)g(x+h)-f(x)g(x)}{h}=$
    $lim_{h\rightarrow 0} \frac{f(x+h)-f(x)}{h}\cdot g(x+h) + f(x)\cdot \frac{g(x+h)-g(x)}{h}=$
    $f'(x)\cdot g(x) + f(x)\cdot g'(x)$ (g jest różniczkowalna $\implies$ g ciągła)
\end{tw}
Wniosek: $(c\cdot f(x))' = (c)'\cdot f(x) + c\cdot f'(x) = 0\cdot f(x) + c\cdot f'(x) = c\cdot f'(x)$ (Liniowość pochodnej)

\subsection{Odwołanie - odwzorowanie liniowe}

$A: X\rightarrow Y$\\
$A(x_1+x_2)=A(x_1)+A(x_2)$\\
$A(c\cdot x)=c\cdot A(x)$

\subsection{Odwrotność pochodnej}

Pokaż, że: $\left(\frac{1}{g(x)}\right)'=\frac{-g'(x)}{g(x)^2}$ (lista zadań)\\
$\left(\frac{1}{g(x)}\right)' = \frac{\frac{1}{g(x+h)}-\frac{1}{g(x)}}{h}$

\subsection{Pochodna ilorazu}

\begin{tw}
    Pochodna ilorazu dwóch funkcji $f(x), g(x)$ wynosi:
    \begin{center}
        $\left(\frac{f(x)}{g(x)}\right)'=\frac{f'(x)\cdot g(x)-f(x)\cdot g'(x)}{\left(g(x)\right)^2}$
    \end{center}
    D-d. $\left(\frac{f(x)}{g(x)}\right)'=\left(f(x)\cdot \frac{1}{g(x)}\right)'=$
    $f'(x)\cdot\frac{1}{g(x)} + f(x)\left(\frac{1}{g(x)}\right)'=$
    $f'(x)\cdot\frac{1}{g(x)} + f(x)\left(\frac{-g'(x)}{g(x)^2}\right)=$
    $\frac{f'(x)g(x)-f(x)g'(x)}{\left(g(x)\right)^2}$
\end{tw}

\begin{pk}
    Rozważmy poniższy wzór:
    \begin{center}
        $\forall_{n\in\mathbb{N}-\{0\}} (x^n)' = n\cdot x^{n-1}$
    \end{center}
    D-d. Wykorzystujemy wzór dwumianowy Newtona:\\
    $(x^n)'=\lim_{h\rightarrow 0} \frac{(x+h)^n-x^n}{h}=$
    $\lim_{h\rightarrow 0} \frac{\left(\sum_{k=0}^{n} \binom{n}{k} h^k x^{n-k}\right) - x^{n}}{h}$
    , Założmy, że $n\geq 2$:\\
    $\lim_{h\rightarrow 0} \frac{\left(\sum_{k=0}^{n} \binom{n}{k} h^k x^{n-k}\right) - x^{n}}{h}=$
    $\lim_{h\rightarrow 0} \frac{h^0 x^n \binom{n}{0}+h^1 x^{n-1} \binom{n}{1} + \left(\sum_{k=2}^{n} \binom{n}{k} h^k x^{n-k}\right) - x^n}{h}=$\\
    $\lim_{h\rightarrow 0} \frac{x^n + h x^{n-1} n + \left(\sum_{k=2}^{n} \binom{n}{k} h^k x^{n-k}\right) - x^n}{h}=$
    $\lim_{h\rightarrow 0} n x^{n-1} + \left(\sum_{k=2}^{n} \binom{n}{k} h^k x^{n-k}\right) = n x^{n-1}$
\end{pk}

\begin{pk}
    $\frac{d}{dx} \left(a_0 + a_1 x + \dots + a_n x^n\right)=a_1 + 2a_2 x + 3a_3 x^2 + \dots + n a_n x^{n-1}$
\end{pk}

\begin{pk}
    Sinus, cosinus:\\
    $sin'(x)=cos(x)$ tydzień temu\\
    $cos'(x)=-sin(x)$ ćw\\
\end{pk}

\begin{pk}
    Policzmy $tan'(x)$:\\
    $tan'(x)=(\frac{sinx}{cosx})'=\frac{sin'(x)cos(x)-sin(x)cos'(x)}{cos(x)^2}=$
    $\frac{cos^2(x)+sin^2(x)}{cos^2(x)}=\frac{1}{(cos(x))^2}$
\end{pk}

\begin{pk}
    Policzmy $cot'(x)$:\\
    $cot'(x)=(\frac{cosx}{sinx})'=\frac{-sin^2(x)-cos^2(x)}{\left(sin(x)\right)^2}=\frac{-1}{\left(sin(x)\right)^2}$
\end{pk}

\subsection{Pochodne [e]}

\begin{pk}
    Lemat techniczny:
    $\forall_{n\in\mathbb{R}} |e^n-1-h| \leq \frac{|h^2|}{2}e^{|h|} = \frac{h^2}{2}e^{|h|}$\\
    $e^n=\sum_{k=0}^{\infty} \frac{h^k}{k!}= 1+h+\sum_{k=2}^{\infty} \frac{h^k}{k!}$\\
    $e^n-1-h=\sum_{k=2}^{\infty} \frac{h^k}{k!} = \sum_{k=0}^{\infty} \frac{h^{k+2}}{(k+2)!}=$
    $\frac{h^2}{2} \sum_{k=0}^{\infty} \frac{2h^k}{(k+2)!\cdot 2}$\\
    $|e^n-1-h| \leq \frac{|h|^2}{2} \sum_{k=0}^{\infty} \frac{2|h|^k}{(k+2)!}=$
    $\frac{h^2}{2}\sum_{k=0}^{\infty} \frac{|h|^k\cdot 2}{(k+2)!}=$
    $\frac{h^2}{2}\sum_{k=0}^{\infty} \frac{|h|^k}{k!} \cdot \frac{2}{(k+1)(k+2)}<$\\
    $\frac{h^2}{2}\sum_{k=0}^{\infty} \frac{|h|^k}{k!}=\frac{h^2}{2}e^{|h|}$
\end{pk}

\begin{tw}
    $\lim_{h\rightarrow 0} \frac{e^h-1}{h}=1 \iff \lim_{h\rightarrow 0} \left(\frac{e^h-1}{h}-1\right)=0$\\
    $\lim_{h\rightarrow 0} \frac{e^h-1-h}{h}=0 \iff \lim \left|\frac{e^h - 1 -h}{h}\right|$\\
    $0<\frac{|e^h-1-h|}{|h|}<\frac{\frac{|h|^2}{2}e^{|h|}}{h}=0$
    Z twierdzenia o trzech ciągach mamy dowód.
\end{tw}

\begin{pk}
    $(e^x)'=\lim_{h\rightarrow 0} \frac{e^{x+h}-e^x}{h} = \lim_{h\rightarrow 0} e^x\cdot \frac{e^h-1}{h} = e^x$   
\end{pk}

\begin{pk}
    $(a^x)'=(e^{ln(a)\cdot x})'=\lim_{h\rightarrow 0} \frac{e^{\ln(a)\cdot(x+h)}-e^{\ln(a)\cdot x}}{h}=$
    $\lim_{h\rightarrow 0} e^{\ln(a)\cdot x} \frac{e^{\ln(a) h}- e^{1}}{h}=$\\
    $=\lim_{h\rightarrow 0} a^x \frac{e^{\ln(a)h}-1}{\ln(a) h} ln(a) = a^x \cdot \ln(a)$
\end{pk}

\subsection{Pochodna funkcji odwrotnej}

\begin{tw}
    Twierdzenie o pochodnej funkcji odwrotnej\\
    $y=f(x)$, różniczkowalna i rosnąca (lub majlejąca) na $[a,b]$.
    Niech $f(x_0)=y_0$. Wówczas istnieje funkcja odwrotna $x=f^{-1}(y)$
    oraz zachodzi wzór:
    \begin{center}
        $\left[f^{-1}(y)\right]'_{y=y_0} = \frac{1}{\left[f(x)\right]'}_{x=x_0}$
    \end{center}
\end{tw}
D-d.\\
Weźmy: $f(x_0 + k) = y_0 + h$\\
$\left[f^{-1}(x)\right]_{y=y_0}=\lim_{h\rightarrow 0} \frac{f^{-1}(y_0+h) - f^{-1}(y_0)}{h}=$
$\lim_{h\rightarrow 0} \frac{f^{-1}(f(x_0+k))-f^{-1}(f(x_0))}{y_0+h-y_0}=$\\
$\lim_{k\rightarrow 0} \frac{x_0 + k - x_0}{f(x_0 +k)-f(x_0)}=$
$\lim_{k\rightarrow 0} \frac{1}{\frac{f(x_0+k)-f(x_0)}{k}}=\frac{1}{\frac{1}{f'(x_0)}}=\frac{1}{\left[f(x)\right]'}_{x=x_0}$

\begin{pk}
    $f(x)=e^x$, $e^{x_0}=y_0, f^{-1}(y)=ln(y)$\\
    $\left[ln(y)\right]'_{y=y_0}=\frac{1}{e^x}'_{x=x_0} = \frac{1}{e^{x_0}} = \frac{1}{y_0}$
    $\left(ln(y)\right)' = \frac{1}{y}$
\end{pk}

\begin{pk}
    $\log_a(y)'=\left(\frac{ln(y)}{ln(a)}\right)'=\frac{1}{ln(a)} \cdot \frac{1}{y}$
\end{pk}

\begin{pk}
    Sprawdźmy zrozumienie tw. o pochodnej funkcji odwrotnej:\\
    $f(x)=sin(x), x\in[-\frac{\pi}{2},\frac{pi}{2}], y=sin(x)$\\
    $\left(arcsin(y)\right)'=\frac{1}{(sin(x))'} = \frac{1}{cos(x)}=\frac{1}{\sqrt{1-sin^2(x)}}=\frac{1}{\sqrt{1-y^2}}$\\
    Następnie:\\
    $f(x)=cos(x), x\in[0,\pi], y=cos(x)$\\
    $\left(arccos(y)\right)'=\frac{1}{(cos(x))'} = \frac{1}{-sin(x)} = \frac{-1}{\sqrt{1-cos^2(y)}}=\frac{-1}{\sqrt{1-y^2}}$\\
    Kolejno:\\
    $f(x)=tan(x), x\in[-\frac{\pi}{2},\frac{pi}{2}], y=tan(x)$\\
    $\left(arctan(y)\right)'=\frac{1}{(tan(x))'}=cos^2(x)=\frac{1}{1+tan^2(x)}=\frac{1}{1+y^2}$
\end{pk}

\end{document}
