\documentclass{article}

\usepackage[polish]{babel}
\usepackage[utf8]{inputenc}
\usepackage{polski}
\usepackage[T1]{fontenc}
 
\usepackage[margin=1.5in]{geometry} 

\usepackage{color} 
\usepackage{amsmath}                                                                    
\usepackage{amsfonts}                                                                   
\usepackage{graphicx}                                                             
\usepackage{booktabs}
\usepackage{amsthm}
\usepackage{pdfpages}
\usepackage{wrapfig}

\theoremstyle{definition}
\newtheorem{de}{Definicja}[subsection]

\theoremstyle{definition}
\newtheorem{tw}{Twierdzenie}[subsection]

\theoremstyle{definition}
\newtheorem{pk}{Przykład}[subsection]

\theoremstyle{definition}
\newtheorem*{fakt}{FAKT}

\author{Rafal Wlodarczyk}
\title{Algebra - Notatki z wykladu}  
\date{INA 1 Sem.}

\begin{document}

\maketitle

\section{Wyklad Pierwszy}

\subsection{Symbole}

\paragraph{Logika}

$\neg, \land, \lor, \implies, \iff$

\paragraph{Zbiory}

$x\in A, A \cap B, A \cup B, A - B, A\backslash B, A^C, B^C, A\subseteq B, A\times B$

\paragraph{Funkcje}

$f: X\rightarrow Y, f: X\times Y\rightarrow A$ funkcja dwuargumentowa

\paragraph{Własność}

$\\\text{Dla } \mathbb{N}$
$W(n) \forall_{x} (x|n)\implies x=1 \lor x=n$
\\Jest to definicja liczb pierwszych.

\subsection{Definicje}

\begin{de}
Niech $X$ - Zbiór. Działaniem na $X$ nazywamy każdą funkcję $f:X\cdot X\rightarrow X$
\end{de}

\begin{pk}
\begin{itemize}
\item[]
\item $f(x,y)=x\cdot y$ Jest działaniem na $\mathbb{R}$ - tak
\item $f(x,y)=x-y$ Jest działaniem na $\mathbb{N}$? - nie, ponieważ $\exists_{x,y} f(x,y)\notin \mathbb{N}$
\end{itemize}
\end{pk}

\paragraph{Oznaczenie}

$f(x,y)\iff x+y, x\cdot y, x\circ y$ - Działanie ogólne

\begin{de}
Niech $X$ - Zbiór. Działanie $\circ$ nazywamy łącznym, gdy:
$\forall_{x,y,z\in X} (x\circ y)\circ z = x\circ (y\circ z)$
Działanie $\circ$ nazywamy przemiennym, gdy:
$\forall_{x,y\in X} x\circ y = y\circ x$
\end{de}

\begin{pk}
\begin{itemize}
\item[]
\item $+$ na $\mathbb{R}$ jest łączne i przemienne
\item $-$ na $\mathbb{R}$ nie jest ani łączne, ani nieprzemienne
\end{itemize}
\end{pk}

\begin{de}
Niech $\circ$ - działanie na zbiorze $X$. Element $e\in X$ nazywamy elementem neutralnym (dla $\circ$), gdy:
$\forall_{x\in X} e\circ x = x\circ e = x$
\end{de}

\begin{pk}
\begin{itemize}
\item[]
\item $0$ jest elementem neutralnym dla $+$ na $\mathbb{N}$
\item $1$ jest elementem neutralnym dla $\cdot$ na $\mathbb{R}$
\end{itemize}
\end{pk}

\begin{fakt}
Niech $\circ$ - działanie na zbiorze $X$. Jeżeli $\circ$ ma element neutralny, to jest on jedyny.
D-d. Niech $a,b$ oznaczają elementy neutralne. Działanie $\circ$ na $X$:
\begin{itemize}
\item $a\circ b = b$
\item $a\circ b = a$
\end{itemize}
Zatem: $a=b\qed$
\end{fakt}

\begin{de}
Niech $\circ$ - działanie na zbiorze $X$. Element $a\in X$ nazywamy elementem odwrotnym (dla $\circ$), gdy:
$\forall_{x\in X} a\circ x = x\circ a = e$
\end{de}

\begin{pk}
\begin{itemize}
    \item[]
    \item $-x$ jest elementem odwrotnym dla $+$ na $\mathbb{R}$
    \item $\frac{1}{x}$ jest elementem odwrotnym dla $\cdot$ na $\mathbb{R}$
    \item $x^2$ nie ma elementu odwrotnego dla $\cdot$ na $\mathbb{R}$
    \item $x^2$ ma element odwrotny dla $\cdot$ na $\mathbb{R}^+$
\end{itemize}
\end{pk}

\begin{fakt}
Niech $\circ$ - działanie na $X, e\in X$ - element neutralny $x\in X$ - dowolny $x$. W działaniu łącznym liczba odwrotna może być co najwyżej jedna.
Istnieje maksymalnie jeden element odwrotny do $x$.\\
D-d. Niech $a,b$ ozn. el. odwrotne do $x$\\
$(a\circ x)\circ b = a\circ (x\circ b)$ (z łączności)\\
$e\circ b=a\circ e$\\
$b=a\qed$
\end{fakt}

\begin{de}
Grupą nazywamy parę elementów $(G,\circ)$, gdzie $G$ - zbiór. $\circ$ działanie na $G$, takie że:
\begin{enumerate}
\item $\circ$ jest działaniem na $G$
\item $\forall_{a,b,c\in G} (a\circ b)\circ c = a\circ (b\circ c)$ - Łączność
\item $\exists_{e\in G} \forall_{a\in G} a\circ e = e\circ a = a$ - Element neutralny
\item $\forall_{a\in G} \exists_{b\in G} a\circ b = b\circ a = e$ - Element odwrotny
\end{enumerate}
\end{de}

\section{Wykład drugi}
... tbd
\section{Wykład trzeci}
... tbd
\section{Wykład czwarty - Pierścienie}

\begin{pk}
$(\mathbb{Z},+,-)$ - rozszerza grupę
\end{pk}

\begin{de}
Pierścieniem nazywamy trójkę $(P,\oplus, \odot)$, gdzie $P$ - zbiór, $\oplus, \odot$ - działania na $P$, takie że:
\begin{enumerate}
\item $(P,\oplus)$ - grupa przemienna (abelowa)
\item działanie na $\odot$ jest łączne na $P$
\item $\forall_{x} \forall_{a,b} x\odot(a\oplus b) = (x\odot a)\oplus (x\odot b)$ oraz
$(a\oplus b)\odot x = a\odot x \oplus b\odot x$
\end{enumerate}
\end{de}

\begin{pk}
$(\mathbb{Z},+,-)$ jest pierścieniem, ponieważ:
\begin{itemize}
\item $(\mathbb{Z},+)$ - grupa przemienna
\item $\cdot$ jest łączne na $\mathbb{Z}$
\item Rozdzielność mnożenia względem dodawania
\end{itemize}
Rozważmy inne przykłady:
\begin{itemize}
\item $(\mathbb{R},+, \cdot), (\mathbb{Q},+,\cdot)$ - pierścienie
\item $(\mathbb{N},+, \cdot)$ - nie jest pierścieniem
\item $(\mathbb{R}[x],+,\cdot)$ - (el neu. $W(x)=0$, el odw. $-W(x)$) - zbiór wielomianów o współczynnikach $\mathbb{R}$
\end{itemize}
\end{pk}

\subsection{Oznaczenia}
Działania na $P$ oznaczamy $+, \cdot$, nazywamy dodawaniem i mnożeniem.
\begin{itemize}
\item Element neutralny $+$ oznaczamy $0$ i nazywamy zerem.
\item Element przeciwny (odwrotny) do $a$ to $-a$, bo $(a+(-a))=0$
\item Element neutralny $\cdot$ (nie musi istnieć) oznaczamy $1$ i nazywamy jedynką
\item Element odwrotny do $a$ to $a^{-1}$
\end{itemize}
Analogicznie do $(\mathbb{Z},+,\cdot)$

\begin{pk}
\begin{itemize}
\item[]
\item Pierścień bez 1 to np. $(2\mathbb{Z},+,\cdot)$
\item Istnieje pierścień z nieprzemiennym $\cdot$ - mnożeniem (pierścień macierzy) 
\end{itemize}
\end{pk}

\subsection{Własności}

\begin{fakt}
Niech $(P,+,\cdot)$ - pierścień. Wtedy:
\begin{enumerate}
\item $\forall_{a\in P} a\cdot 0=0\cdot a = a$\\
D-d. $a\cdot 0=a\cdot(0+0)=^{rd} a\cdot 0 + a\cdot 0 | +(-(a\cdot 0))$\\
$a\cdot 0 - (a\cdot 0) = a\cdot 0 + a\cdot 0 - a\cdot 0$\\
$0=a\cdot 0 \qed$
\item $\forall_{a,b\in P} (-a)\cdot b=-(a\cdot b)$\\
D-d. $(-a)\cdot b=-(a\cdot b) | + (a\cdot b)$\\
$(-a)\cdot b + a\cdot b =^{rd} (-a + a)\cdot b = 0\cdot b=^{1} 0$\\
$(-a)\cdot b+ a\cdot b=0 |+(-(ab))$\\
$(-a)\cdot b = -(ab)$
\item $\forall_{a,b\in P} (-a)\cdot (-b) = a\cdot b$\\
D-d. ćwiczenie
\item $\forall_{a,-a\in P} (-1)\cdot a = -a$\\
D-d. ćwiczenie
\end{enumerate}
\end{fakt}

\begin{de}
Niech $(P,+,\cdot)$ - pierścień oraz niech $A\subseteq P$. Zbiór niepusty $A$ nazywamy podpierścieniem, gdy:
\begin{enumerate}
\item $\forall_{a,b\in A} (a+b)\in A \land (-a)\in A$
\item $\forall_{a,b\in A} (a\cdot b)\in A$ (odwrotność mnożenia nie jest wymagana)
\end{enumerate}
\end{de}

\begin{pk}
Niech $P=(\mathbb{R}, +, \cdot), A=\mathbb{Z}$. $\mathbb{Z}$ jest podpieścieniem, ponieważ:
\begin{enumerate}
\item $a,b \in \mathbb{Z} \implies a+b\in \mathbb{Z} \land (-a)\in \mathbb{Z}$
\item $a,b \in \mathbb{Z} \implies a\cdot b\in \mathbb{Z}$
\end{enumerate}
$2\mathbb{Z}$ jest podpierścieniem $(P,+,\cdot)$\\
$((0,\infty),+,\cdot)$ nie jest podpierścieniem bo $7\in (0,\infty), -7\notin (0,\infty)$
\end{pk}

\paragraph{Oznaczenie}
$A\leq P$ oznaczamy, że $A$ jest podpierścieniem $P$

\paragraph{Własności} Jeśli $(P,+,\cdot)$ - pierścień oraz $A\leq P$
to $(A,+,\cdot)$ jest pierścieniem.\\

D-d. $\leftarrow$ ćwiczenie.
$(P,+,\cdot)$ posiada dwa podpierścienie $P\leq P$ i $\{0\}\leq P$

\subsection{Produkt}
Niech $(P,+,\cdot), (R,\oplus,\odot)$ - pierścienie\\
Na zbiorze $P\times R$ definiujemy działania:
\begin{enumerate}
\item $(p_1, r_1) +_b (p_2, r_2) = (p_1+p_2,r_1 \oplus r_2)$ 
\item $(p_1, r_1) \cdot_b (p_2,r_2) = (p_1\cdot p_2, r_1\odot r_2)$
\end{enumerate}
D-d. Sprawdzić listę własności z definicji pierścienia.

\begin{pk}
$\mathbb{Z}\times \mathbb{Z}$\\
$(3,5) + (7,8) = (3+7, 5+8) = (10,13)$\\
Element neutralny $(0,0)\in \mathbb{Z}\times \mathbb{Z}$\\
Element przeciwny $-(a,b) = (-a,-b)$
\end{pk}

\begin{de}
Niech $n\in \mathbb(N)^{+}$ $\mathbb{Z}_n=(\{0,1,2,\dots ,n-1\}, +_n, \cdot_n)$. 
\end{de}

\begin{fakt}
$\mathbb{Z}_n$ jest pierścieniem skończonym.\\
D-d.
\begin{enumerate}
\item $(\{0,1,\dots ,n-1\},+_n)$ - grupa przeciwna $(\mathbb{C}_n)$
\item Łączność - $(a\cdot_n(b\cdot_n c))=(a\cdot b\cdot c) mod (n)$
\item Rozdzielność - $L= a\cdot_n (x +_n y)=(a(x+y))mod (n)$\\
$P=a\cdot_n x +_n a \cdot_n y =(ax+ay)mod (n)$\\
$L=P$ z rodzielności dodatania w $(\mathbb{Z},+,\cdot)$
\end{enumerate}
\end{fakt}

\begin{de}
Niech $(P,+,\cdot), (R, \oplus, \odot)$ - pierścienie. Funkcję $f_{P\rightarrow R}$ nazywamy homomorfizmem, gdy:
\begin{enumerate}
\item $\forall_{a,b\in P} f(a+b)=f(a)\oplus f(b)$
\item $\forall_{a,b\in P} f(a\cdot b)=f(a)\odot f(b)$
\end{enumerate}
\end{de}

\begin{pk}
\begin{enumerate}
\item[]
\item $(P,+,\cdot)$ oraz $f(a)=a :P\rightarrow P$ to $f$ jest homomorfizmem.
\item $(P,+,\cdot)$ oraz $g(a)=- :P\rightarrow P$ to $g$ jest homomorfizmem.\\\\
D-d. $g(a+b)=0=0+0=g(a)+g(b) \land g(a\cdot b)=0\cdot 0 = g(a)\cdot g(b)$
\end{enumerate}
\end{pk}

\begin{pk}
$\varphi_n(k)=k (mod(n)) : \mathbb(Z) \rightarrow {0,1,\dots, n-1}$\\
$\varphi_n$ jest homomorficzna dla pierścieni $(\mathbb{Z},+,\cdot), (\mathbb(Z)_n,+_n,\cdot_n)$, ponieważ:
\begin{enumerate}
\item $\varphi_n (a+b)=(a+b) mod (n)$\\
$(a (mod (n)) + b (mod (n))) mod (n)=\varphi_n(a) +_n \varphi_n(b)$
\item ćwiczenie (dla mnożenia)
\end{enumerate}
\end{pk}

\begin{fakt}
Niech $(P,+,\cdot), (R, \oplus, \odot)$ - pierścienie. Oraz $f_{P\rightarrow R}$ homomorfizm. Wtedy:
\begin{enumerate}
\item $f(0_P)=0_R$\\\\
D-d. $f(0_P)=f(0_P+0_P)=f(0_P)+f(0_P)$\\
$f(0_P)=f(0_P)+f(0_P) | +(-f(0_P))$\\
$0_R=f(0_P)\qed$
\item $f(-a)=-f(a)$\\\\
D-d. $f(-a)+f(a)=^{hom.} f((-a)+a)=f(0)=^{1}0 | + (-f(a))$\\
$f(-a)=-f(a)$
\item $f(1_P)=1_R$, o ile istnieje
\item $f(a^{-1})=f(a)^{-1}$, o ile istnieje
\end{enumerate}
\end{fakt}

\subsection{Zastosowanie}

\paragraph{Reguła podzielności przez 3}
Notacja. $a,b,c$ - cyfry $0,\dots, 9$, $a|b$ - $a$ dzieli $b$ \\ 
$\overline{abc}=100a + 10b + c$\\
$\overline{933}=933$\\
Przypadek: $3|\overline{abc} \iff 3| (a+b+c)$\\
$3|\overline(abcd) \iff \overline{abcd} (mod (3)) = 0$\\
$\varphi_3(\overline{abcd})=0$\\
$\varphi_3(1000a+100b+10c+d)=^{hom}$\\
$\varphi_3(1000a)+\varphi(100b)+\varphi(10c)+\varphi(d)$\\
$\varphi_3(10)^3+\varphi_3(10)^2+\varphi(10)^1+\varphi(a)+\varphi(b)+\varphi(c)+\varphi(d)=$\\
$\varphi_3(a)+\varphi(b)+\varphi(c)+\varphi(d)=\varphi_3(a+b+c+d)$

\end{document}