\documentclass{article}

\usepackage[english]{babel}
\usepackage[utf8]{inputenc}
\usepackage{polski}
\usepackage[T1]{fontenc}
 
\usepackage[margin=1.5in]{geometry} 

\usepackage{color} 
\usepackage{amsmath}
\usepackage{amsfonts}                                                                   
\usepackage{graphicx}                                                             
\usepackage{booktabs}
\usepackage{amsthm}
\usepackage{pdfpages}
\usepackage{wrapfig}
\usepackage{hyperref}
\usepackage{etoolbox}

\makeatletter

\theoremstyle{definition}
\newtheorem{de}{Defiition}[subsection]

\theoremstyle{definition}
\newtheorem{tw}{Theorem}[subsection]

\theoremstyle{definition}
\newtheorem{pk}{Example}[subsection]

\theoremstyle{definition}
\newtheorem{fakt}{Fact}

\title{Probabilistic Methods and Statistics}  
\author{Rafał Włodarczyk}
\date{INA 3, 2024}

\begin{document}

\maketitle

\begin{pk}
    \begin{itemize}
        \item Ustalmy przestrzeń \(\Omega\)
        \item \(A,B\in \Omega, A\neq B, A,B\neq \empty, A,B\neq \Omega\)
        \item Oznaczmy \(A=A^1,A^C=A^{-1}\)
        \item Mamy 4 \"atomowe\" zbiory postaci \(A^i \cap B^j; i,j\in\{-1,1\}\)
        \item Przy pomocy sumy zbiorów możemy z nich budowac bardziej złożone zbiory
        \item Ile takich zbiorów możemy zbudować? 
    \end{itemize}
\end{pk}

\begin{de}
    Ustalmy przestrzeń \( \Omega \) Rodzinę \(S\in \mathcal{P}(\Omega)\) nazywamy ciałem podzbiorów zbioru \(\Omega\) (ang. field, alebra), jeśli:
    \begin{itemize}
        \item \(S\neq \emptyset\)
        \item \(A\in S \implies A^C\)
        \item \(A,B\in S \implies A\cup B \in S\)
    \end{itemize}

\end{de}

\begin{fakt}
    Weźmy \(A,B\in S \implies A^C,B^C\in S\), wtedy \(A^C\cup B^C\in S\), zatem z Prawa de Morgana \(A\cap B \in S\). Jak widać, wynika to z definicji ciała.
\end{fakt}

\begin{fakt}
    \(\cap_{i\in I} A_i = \left(\cup_{i\in I} A^C\right)^C\in S \)
\end{fakt}

\begin{de}
    Ustalmy przestrzeń \(\Omega\). Rodzinę \(S\in \mathcal{P}(\Omega)\) nazywamy $\sigma$-ciałem podzbiorów zbioru \(\Omega\), jeśli:
    \begin{itemize}
        \item \(S\neq \emptyset\)
        \item \(A\in S \implies A^C \in S\)
        \item \(A_1,A_2,\dots \in S \implies \bigcup_{i\geq 1} A_i \in S\)
    \end{itemize}
\end{de}

\begin{fakt}
    Niech \(S\) będzie $\sigma$-ciałem podzbiorów $\Omega$ i załóżm, że dla pewnego przeliczalnego zbioru indeksów $I$ zachodzi \(\forall i\in I A_i \in S\) wówczas:
    \[\bigcap_{i\in I} A_i \in S\]
\end{fakt}

\begin{fakt}
    Jeśli $S$ jest skończonym ciałem podzbiorów $\Omega$ to $S$ jest $\sigma$-ciałem podzbiorów $\Omega$
\end{fakt}

Istotnie, wówczas przeliczalne sumy "redukują się" do sum skończonych

\begin{pk}
    Trywialne sigma ciała:
    \begin{itemize}
        \item \(\mathcal{P}(\Omega)\) - zbiór potęgowy jest $\sigma$-ciałem
        \item \(\{\emptyset, \omega\}\) - jest $\sigma$-ciałem
    \end{itemize}
\end{pk}

\begin{pk}
    Ustalmy zbiór $\Omega$. Niech $A\subseteq \Omega, A \neq \emptyset$.
    \begin{itemize}
        \item \(S=\{\emptyset, A, A^C, \Omega\}\) jest $\sigma$-ciałem.
        \item Niech $\mathcal{F}$- dowolne $\sigma$-ciało zawierające $A$.\\
        \(A^C \in \mathcal{F}\\
        A^C\cup A = \Omega \in \mathcal{F}\\
        \Omega^C=\emptyset\in\mathcal{F} \implies S\subseteq \mathcal{F}\)
    \end{itemize} 
\end{pk}

\begin{fakt}
    Przekrój $\sigma$-ciał jest $\sigma$-ciałem.
\end{fakt}



\end{document}
