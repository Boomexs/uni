\documentclass{article}

\usepackage[polish]{babel}
\usepackage[utf8]{inputenc}
\usepackage{polski}
\usepackage[T1]{fontenc}
 
\usepackage[margin=1.5in]{geometry} 
\usepackage{color} 
\usepackage{amsmath}
\usepackage{amsfonts}
\usepackage{graphicx}
\usepackage{booktabs}
\usepackage{amsthm}
\usepackage{pdfpages}
\usepackage{wrapfig}
\usepackage{hyperref}
\usepackage{etoolbox}
\AtBeginEnvironment{align}{\setcounter{equation}{0}}

\theoremstyle{definition}
\newtheorem{de}{Definicja}[subsection]

\theoremstyle{definition}
\newtheorem{tw}{Twierdzenie}[subsection]

\theoremstyle{definition}
\newtheorem{pk}{Przykład}[subsection]

\theoremstyle{definition}
\newtheorem*{fakt}{FAKT}

\author{Rafal Wlodarczyk}
\title{Analiza Matematyczna II}  
\date{INA 2 Sem. 2023}

\begin{document}

\maketitle

\section{Wykład I}

\subsection{Iloczyn skalarny}

$\mathbb{R}^n=\{(x_1,x_2,\dots, x_n), x_i \in \mathbb{R}\}$

\begin{de}
Dla $x,y \in \mathbb{R}^n$ definiujemy iloczyn skalarny:
\begin{center}
    $<x,y> = \sum_{i=1}^{n} x_i y_i$
\end{center}
$(x=(x_1,x_2,\dots, x_n) y = (y_1,y_2,\dots, y_n))$\\
$ax = (ax_1, ax_2, \dots, ax_n)$\\
$\sqrt{\langle x,x \rangle} = \sqrt{x_1^2+x_2^2+\dots+x_n^2}$\\
Własności:
\begin{enumerate}
    \item $\langle x,y \rangle = <y,x>$
    \item $<ax,y> = <x,ay> = a<x,y>, a \in \mathbb{R}$
    \item $<x+y,z>=<x,z>+<y,z>$
\end{enumerate}
\end{de}

\begin{de}
    Długość wektora $x\in \mathbb{R}^n$\\
    \begin{center}
        $|x|=\sqrt{\langle x,x \rangle}$
    \end{center}
\end{de}

\begin{pk}
	$\mathbb{R}$ : $|x|=|x_1|$ - oś liczbowa
	$\mathbb{R}^2$ : $|x|=\sqrt{x_1^2+y_2^2}$ - płaszczyzna
\end{pk}

\begin{tw}
	$x\in\mathbb{R}^n$. Wówczas $|\langle x,y \rangle| \leq |x|\cdot |y|$\\
	D-d. $x=(x_1,x_2,\dots,x_n)$, $y=(y_1,y_2,\dots, y_n)$\\
	$|\langle x,y, \rangle| = |\sum_{i=1}^{n} x_i y_i| \leq \sqrt{\sum_{i=1}^{n} x_i^2} + \sqrt{\sum_{i=1}^{n} y_1^2}$\\
	Nierówność Cauchy'ego Schwarza, a zatem dowód.
\end{tw}
Wniosek $x,y\in\mathbb{R}^n$.
\begin{center}
	$|x+y| \leq |x| + |y|$
\end{center}
D-d.\\
$|x+y|^2=\langle x+y, x+y \rangle$
$= \langle x, x+y \rangle + \langle y, x+y \rangle$
$= \langle x, x \rangle + \langle x, y \rangle + \langle y, y \rangle + \langle y, x \rangle$
$=|x|^2+|y|^2 + 2\langle x,y \rangle \leq |x|^2 |y|^2 + 2|x||y|$
$=(|x|+|y|)^2$
$=|x+y|^2 \leq (|x|+|y|)^2$
$\iff$
$|x+y| \leq |x| + |y|\qed$

\subsection{Kąt między wektorami}

$\overrightarrow{x_1}, \overrightarrow{x_2} \in \mathbb{R}^2$,
$\overrightarrow{x_1} = (x_{11}, x_{12})$,
$\overrightarrow{x_2} = (x_{21}, x_{22})$\\
\begin{center}
    $\cos(\overrightarrow{x_1}, \overrightarrow{x_2})=$
    $\frac{\overrightarrow{x_1} \odot \overrightarrow{x_2}}{|\overrightarrow{x_1}||\overrightarrow{x_2}|}$
\end{center}
Rozważmy funkcję:\\
$d_n \mathbb{R}^n \times \mathbb{R}^n \rightarrow \mathbb{R}$\\
$d_n(x,y) = |x-y|$, dla $n=2$:\\\\
$x=(x_1,x_2), y=(y_1,y_2)$
$|x-y| = \sqrt{(x_1-y_1)^2+(x_2-y_2)^2}$
Własności:
\begin{enumerate}
    \item $d_n(x,y)\geq 0$
    \item $d_n(x,y)=0 \iff x=y$
    \item $d_n(x,y)=d_n(y,x)$
    \item $d_n(x,z)\leq d_n(x,y) + d_n(y,z)$ - nierówność trójkąta
\end{enumerate}

\subsection{Przestrzeń metryczna}

\begin{de}
Przestrzenią metryczną nazywamy dowolny zbiór $X$, pewną funkcję $X\times X \rightarrow \mathbb{R}$,
która spełnia następujące aksjomaty:
\begin{enumerate}
    \item $d(x,y)\geq 0$
    \item $d(x,y)=0 \iff x=y$
    \item $d(x,y)=d(y,x)$ dla $x,y\in X$
    \item $d(x,z)\leq d(x,y) + d(y,z)$ dla każdych $x,y,z \in X$
\end{enumerate}
Funkcję $d$ nazywamy metryką, a wartość $d(x,y)$ odległością punktów.

\end{de}
Uwaga: aksjomat $1$ wynika z pozostałych aksjomatów\\
D-d.\\
$d(x,y) = \frac{1}{2} \left(d(x,y)+d(y,x)\right) \geq \frac{1}{2} d(x,x) = 0$, zatem $d(x,y)\geq 0$

\begin{tw}
Stwierdzenie. Niech $(X,d)$ (Corollary) będzie przestrzenią metryczną oraz $x_1,x_2,\dots,x_n\in X$.
Wówczas:
\begin{center}
    $d(x_1,x_n)\leq\sum_{j=1}^{n-1} d(x_j,x_{j+1}), n\geq 2$
\end{center}
Dla $n=2$:\\
    $d(x_1,x_2)\leq d(x_1,x_2)$ - oczywiste\\
Dla $n=3$:\\
    $d(x_1,x_3)\leq d(x_1,x_2) + d(x_2,x_3)$ - nierówność trójkąta\\
Krok indukcyjny:\\
    $d(x_1,x_{n+1})\leq d(x_1,x_2)+d(x_2,x_3)+\dots+d(x_n,x_{n+1})$\\
    $d(x_1,x_{n+1})\leq d(x_1,x_n)+d(x_n,x_{n+1}) \leq_{ind}$
    $d(x_1,x_2) + d(x_2,x_3) + \dots + d(x_{n-1}, x_n) + d(x_{n},x_{n+1})\qed$
\end{tw}

\subsection{Przestrzeń metryczna dyskretna}

$X$ - dowolny zbiór i metryka określona wzorem:
\begin{center}
    $$d(x,y)=\begin{cases}
        0 \text{ dla } x=y\\
        1 \text{ dla } x\neq y
    \end{cases}$$
\end{center}
Aksjomaty $1,2,3,4$ są oczywiste.

\subsection{Metryka Euklidesowa}

$d_n: \mathbb{R}^n \times \mathbb{R}^n \rightarrow \mathbb{R}$\\
$x=(x_1,x_2,\dots, x_n)$\\
$y=(y_1,y_2,\dots, y_n)$\\
$d_n(x,y)=\sqrt{\sum_{i=1}^{n} (x_i-y_i)^2}$

\subsection{Przestrzeń Hilberta}

$x=(x_1,x_2,\dots, x_n)$\\
$y=(y_1,y_2,\dots, y_n)$\\
$\sum_{i=1}^{\infty} x_i^2\leq \infty, \sum_{i=1}^{\infty} y_i^2 < \infty$\\
$x=\left(\frac{1}{1},\frac{1}{2},\frac{1}{3},\dots,\frac{1}{i}, \dots \right)$\\
$\sum_{i=1}^{\infty} \frac{1}{i^2} \leq \infty$\\
$\sum_{i=1}^{\infty} \left(\frac{1}{\sqrt{i}}\right)^2=\sum_{i=1}^{\infty} \frac{1}{i} = \infty$\\
$y=\left(\frac{1}{\sqrt{1}},\frac{1}{\sqrt{2}},\dots \right)$

\subsection{Metryka Manhattan}

$d(x,y)=|x_1-y_1| + |x_2-y_2|$\\
$d((x_1,x_2),(y_1,y_2))=|x_1-y_1|+|x_2-y_2|$

\section{Wykład II}

\subsection{Kula otwarta}

\begin{de}
Kula otwarta w przestrzeni metrycznej $Y$:
\begin{center}
    $K(y_0, r) = \{y\in Y: d(y, y_0) < r\}$
\end{center}
Gdzie:
\begin{itemize}
    \item $y_0$ - środek
    \item $r$ - promień
    \item Wnętrze okręgu w $\mathbb{R}^2$ - metryka Euklidesowa
    \item Wnętrze kuli w $\mathbb{R}^3$
\end{itemize}
\end{de}

\begin{pk}
    Rozważmy następujący przykład: \\
    $K((x_0,y_0),r)$ :: 
    $\sqrt{(x-x_0)^2 + (y-y_0)^2} < r$ zachodzący warunek\\
    (wyobraż sobie rysunek poglądowy)
\end{pk}

\begin{de}
    Niech $(X,d)$ będzie przestrzenią metryczną. Zbiór $U\subseteq X$ jest otwarty, jeśli:
    \begin{center}
        $\forall_{x\in U} \exists_{\varepsilon > 0} \left(K(x,\varepsilon) \subseteq U\right)$
    \end{center}
    (Kula otwarta o środku w punkcie $X$ i promieniu $\varepsilon > 0$)
\end{de}

\begin{pk}
    Przykłady: \\
    $(a,b)$ - jest otwarty\\
    $[a,b)$ - nie jest otwarty
\end{pk}

\begin{de}
    Niech $(X,d)$ będzie przestrzenią metryczną. Zbiór $D\subseteq X$ jest zbiorem domkniętym $\iff$ $X - D$ jest otwarty.\\
    $[a,b] \subset \mathbb{R} \implies \mathbb{R} - [a,b] = (\infty, a) \cup (b,\infty)$ - zbiór otwarty
\end{de}

\begin{de}
    $(X,d_1), (Y, d_2)$ - przestrzenie metryczne\\
    $F: X\rightarrow Y$\\
    $\lim F(x) = b$ :: $x\rightarrow a$ :: $a\in X, b\in Y$\\
    Przykład. Dla $X=\mathbb{N}, Y=\mathbb{R}$ - ciągi, dla obu $\mathbb{R}$ - funkcje\\
    \begin{center}
        $\forall_{\varepsilon>0} \exists_{\delta > 0} \forall_{x} (0 < d_1 (x,a) < \delta \implies d_2(F(x),b) < \varepsilon)$
    \end{center}
\end{de}

\begin{tw}
    Warunki (1), (2) są równoważne:
    \begin{enumerate}
        \item $\lim_{x\rightarrow a} F(x) = b$
        \item dla dowolnego ciągu $(x_n)_{n\geq 0}$ 
        punktów przestrzeni metrycznej $X(x_n \neq a)$\\
        jeśli $\lim_{n\rightarrow a} x_n = a$ w metryce $d_1$ to
        $\lim_{n\rightarrow \infty} F(x_n) = b$
    \end{enumerate}
\end{tw}
$\lim_{n\rightarrow \infty} x_n = a \iff$
$(\forall \varepsilon) (\exists n_0) (\forall n>n_0)$  $d_1(x_n,a)< \varepsilon$\\\\
$(|x_n-a|<\varepsilon), x_n, a \in (X, d_1) \implies \lim_{n\rightarrow \infty} d_1(x_n,a) = 0$

\begin{pk}
    $x_n\in\mathbb{R}, a\in X$\\
    $d_1(x_n,a) \in \mathbb{R}$\\
    $\lim_{n\rightarrow \infty} x_n = a \iff \lim_{n\rightarrow\infty} d_1(x_n,a)=0$ w metryce $d_1$
\end{pk}

\begin{pk}
    $(a_n,b_n,c_n) \in \mathbb{R}^3$\\
    $\lim_{n\rightarrow \infty} (a_n,b_n,c_n) = (g_1,g_2,g_3)$ w metryce Eulidesowej $\mathbb{R}^3 \iff$\\
    $\lim_{n\rightarrow \infty} a_n = g_1 \land \lim_{n\rightarrow \infty} b_n = g_2 \land \lim_{n\rightarrow \infty} c_n = g_3$\\
    Idea:\\
    $\sqrt{(a_n-g_1)^2 + (b_n-g_2)^2 + (c_n-g_3)^2}\rightarrow 0 \iff a_n\rightarrow g_1 \land b_n \rightarrow g_2 \land c_n \rightarrow g_3$\\
    Dla $\mathbb{R}^k$ podane włansości zachodzą analogicznie.
\end{pk}

\begin{de}
    Ciągłość funkcji. $(X,d_1), (Y, d_2), F: X\rightarrow Y$. \\
    Funkcja $F$ jest ciągła w punkcie $a$ jeśli:
    \begin{center}
        $\lim_{x\rightarrow a} F(x) = F(a)$
    \end{center}
    $x\rightarrow a$ w $d_1 \implies F(x)\rightarrow F(a)$ w $d_2$
\end{de}

\begin{pk}
    Weźmy funkcje $f: \mathbb{R}^2 \rightarrow \mathbb{R}$.\\
    $$
    f(x,y)=\begin{cases}
        \frac{xy}{x^2+y^2} \text{ dla } x^2+y^2 > 0\\
        0 \text{ dla } (x,y) = (0,0)
    \end{cases}
    $$\\
    Pokażmy, że $f$ nie jest ciągła w $(0,0)$\\
    $(x_n,y_n) = (\frac{1}{n}, \frac{1}{n}) \rightarrow (0,0)$\\
    $\lim_{n\rightarrow \infty} f(x_n, y_n)$\\
    $f(\frac{1}{n}, \frac{1}{n}) = \frac{1}{2}$ - nie dąży do $0$ nie jest ciągła w $(0,0)$. 
\end{pk}

\begin{pk}
    $f: \mathbb{R}^3 \rightarrow \mathbb{R}^2$\\
    $f(x,y,z) = (x^2,y^2\cdot z)$\\
    Zbadajmy ciąg $a=(x_0, y_0, z_0)$\\
    $f(x_0, y_0, z_0) = (x_0^2, y_0^2\cdot z)$\\
    $(x_n, y_n, z_n) \rightarrow (x_0, y_0, z_0)$\\
    $x_n\rightarrow x_0 \land y_n \rightarrow y_0 \land z_n \rightarrow z_0$\\
    $f(x_n,y_n,z_n) = (x_n^2, y_n z_n)$\\
\end{pk}

\begin{pk}
    $$f(x,y)=\begin{cases}
        \frac{xy^2}{x^2+y^4} \text{ dla } x^2 + y^4 > 0\\
        0 \text { dla } (x,y) = (0,0)
    \end{cases}$$
    $f$ jest ciągła w $(0,0)$ $(\alpha t, \beta t) \rightarrow (0,0)$\\
    $f(\alpha t, \beta t) = \frac{\alpha \beta^2 t^3}{\alpha^2 t^2 + \beta^4 t^4}=$\\
    $\frac{\alpha \beta^2 t}{\alpha^2 + \beta^4 t^2} \rightarrow \frac{0}{\alpha^2} = 0 = f(0,0)$\\
    $\lim_{t\rightarrow 0} f(t^2,t) = \lim_{t\rightarrow 0} \frac{t^2 t^2}{t^4 + t^4} = \frac{1}{2}$\\
    $f$ nie jest ciągła $0=f(0,0)\neq \frac{1}{2}$, czyli nie tylko liniowa ale też dowolna\\\\
    Kolejny przykład obalający dla zdef. funkcji $\left(\frac{1}{n^2} , \frac{1}{n}\right) \rightarrow (0,0)$, ale już\\
    $f\left(\frac{1}{n^2} + \frac{1}{n}\right) = - \frac{1}{2} \neq f(0,0)$
\end{pk}

\begin{pk}
    $f: \mathbb{R} \times \mathbb{R} \rightarrow \mathbb{R}$
    $f(x,y) = x+y$\\
    $g: \mathbb{R} \times \mathbb{R} \rightarrow \mathbb{R}$
    $g(x,y) = xy$\\\\
    $(x_n,y_n) \rightarrow (x_0, y_0) \iff (x_n\rightarrow x_0 \land y_n \rightarrow y_0)$\\
    $\lim_{n\rightarrow \infty} f(x_n, y_n) = \lim_{n\rightarrow \infty} (x_n + y_n) = x_0 + y_0 = f(x_0, y_0)$\\
    $\lim_{n\rightarrow \infty} g(x_n, y_n) = \lim_{n\rightarrow \infty} (x_n, y_n) = x_0 y_0 = g(x_0, y_0)$\\
\end{pk}

\begin{pk}
    $\lim_{x\rightarrow 0} (\lim_{y\rightarrow 0} f(x,y)) =$\\
    $\lim_{x\rightarrow 0} (\lim_{y\rightarrow 0} \frac{xy}{x^2+y^2}) = \lim_{x\rightarrow 0} (0) = 0$
    $\lim_{y\rightarrow 0} (\lim_{x\rightarrow 0} f(x,y)) =$\\
    $\lim_{y\rightarrow 0} (\lim_{x\rightarrow 0} \frac{xy}{x^2+y^2})$\\
    Nie istnieje\\
    $(x_n', y_n') = (\frac{1}{n}, \frac{1}{n}) \rightarrow (0,0)$\\
    $(x_n'', y_n'') = (-\frac{1}{n}, \frac{1}{n}) \rightarrow (0,0)$\\
    $f(x_n', y_n') = \frac{1/n 1/n}{(1/n)^2 + (1/n)^2} = 1/2$\\
    $f(x_n'', y_n'') = \frac{-\frac{1}{n}\frac{1}{n}}{(1/n)^2 + (1/n)^2}=-1/2$\\
    Ergo rozbieżny - granica podwójna nie istnieje.
\end{pk}

\subsection{Granica podwójna}

\begin{de}
    $\lim_{(x,y)\rightarrow(x_0,y_0) f(x,y)}$
\end{de}

\subsection{Granice iterowane}

\begin{de}
    $\lim_{x\rightarrow x_0} (\lim_{y\rightarrow y_n} f(x,y))$\\
    $\lim_{y\rightarrow y_0} (\lim_{x\rightarrow x_n} f(x,y))$
\end{de}

\subsection{Różniczkowanie}

$f: \mathbb{R} \rightarrow \mathbb{R}$:
$f'(x) = a \iff \lim_{h\rightarrow 0} \frac{f(x+h)-f(x)}{h}=a$\\
$\lim_{h\rightarrow 0} \frac{f(x+h)-f(x)}{h} - a = 0$\\
$\lim_{h\rightarrow 0} \frac{f(x+h)-f(x)-ah}{h} = 0$\\
$\lim_{h\rightarrow 0} \left|\frac{f(x+h)-f(x)-ah}{h}\right| = 0$\\
$\lim_{h\rightarrow 0} \frac{|f(x+h)-f(x)-ah|}{|h|} = 0$\\\\
$L: \mathbb{R} \rightarrow \mathbb{R}, L(h)= ah, h\in\mathbb{R}$\\
$L(h_1+h_2) = L(h_1) + L(h_2)$\\
$L(ch)=c\cdot L(h)$ ($L$ jest odwzorowaniem liniowym)\\

\begin{de}
    (Pochodna funkcji) $n, m \in \mathbb{N}-\{0\}$, $f: \mathbb{R}^n \rightarrow \mathbb{R}^n, x \in \mathbb{R}^n$\\
    Mówimy że funkcja $f$ jest różniczkowalna w punkcie $x$\\
    jeśli istnieje odwzorowaniem liniowe $f'(x): \mathbb{R}^n \rightarrow \mathbb{R}^m$\\
    takie że $h\in\mathbb{R}^n 0_n = (0,0,\dots, 0)$

    \begin{center}
        $\lim_{h\rightarrow 0} \frac{||f(x+h)-f(x)-f'(x)(h)||}{||h||} = 0_{\mathbb{R}}$
    \end{center}
\end{de}


\section{Wykład 3}

\subsection{Pochodne cząstkowe}

\begin{de}
$f: \mathbb{R}^2\rightarrow\mathbb{R}$
    \begin{center}
    \[\frac{\partial f}{\partial x} (x_0,y_0) = \lim_{h\rightarrow 0} \frac{f(x_0+h,y_0)-f(x_0,y_0)}{h}\]
    \[\frac{\partial f}{\partial y} (x_0,y_0) = \lim_{h\rightarrow 0} \frac{f(x_0,y_0+h)-f(x_0,y_0)}{h}\]
    \end{center}
\end{de}

\begin{pk}
    Policzmy następującą pochodne cząstkowe dla funkcji:\\
    $f(x,y)=x\cdot y^2, f: \mathbb{R}^2\rightarrow \mathbb{R}$\\
    $\frac{\partial}{\partial x} (xy^2) = y^2$\\
    $\frac{\partial}{\partial y} (xy^2) = 2xy$\\
\end{pk}

\begin{de}
    $f:\mathbb{R}^n \rightarrow \mathbb{R}^m, x \in \mathbb{R}^n$ jest różniczkowalna w $x$ jeśli istnieje odwzorowanie liniowe:
        $f'(x):\mathbb{R}^n\rightarrow\mathbb{R}^m$ taka, że:
    \begin{center}
        \[\lim_{h\rightarrow 0} \frac{f(x+h)-f(x)-f'(x)(h)}{|h|}=0\]
    \end{center}
\end{de}

\begin{tw}
    Zakładamy, że $f:\mathbb{R}^n\rightarrow\mathbb{R}^m$ jest różniczkowalna. 
    Niech: $f=(f_1,f_2,\dots,f_m), f_{i}: \mathbb{R}^n\rightarrow\mathbb{R}, i = 1,2,\dots,m$:
    $a_{ij} = \frac{\partial f_i}{\partial x_j} (x), j = 1,2,\dots,n$. Wówczas macierz pochodnej wynosi:
    \begin{center}
        \[M_{f'(x)}=\begin{bmatrix}
            a_{1,1} & a_{1,2} & \dots & a_{1,n} \\
            a_{2,1} & a_{2,2} & \dots & a_{2,n} \\
            \dots & \dots & \dots & \dots \\
            a_{m,1} & a_{m,2} & \dots & a_{m,n} \\
            \end{bmatrix}\]
    \end{center}
\end{tw}

Uwaga. Istnienie wszystkich pochodnych cząstkowych nie wystarcza aby funkcja była różniczkowalna.
$$
f(x,y)=\begin{cases} 
    \frac{x^2y}{x^2+y^2} \text{ dla } x^2+y^2>0\\
    0 \text{ dla } (x,y)=(0,0) 
\end{cases}
$$
Cel. pokażmy że $f$ nie jest różniczkowalna w $(0,0)$\\
$\frac{\partial f}{\partial x} (0,0) = \lim_{h\rightarrow 0} \frac{f(h,0)-f(0,0)}{h} = \lim_{h\rightarrow 0} \frac{0}{h} = 0$\\
$\frac{\partial f}{\partial y} (0,0) = 0$\\
Kandydat na pochodną:
$$
M_{f'(0,0)} = \left[\frac{\partial f}{\partial x} (0,0),\frac{\partial f}{\partial y} (0,0)\right]
$$
Warunek różniczkowania: $h=(h_1,h_2)$\\
$$
\lim_{(h_1,h_2)\rightarrow(0,0)}
\frac{\left|f(h_1,h_2) - f(0,0) - [0,0]
    \left[\begin{matrix}
    h_1\\h_2\end{matrix}\right]
    \right|}{\sqrt{h_1^2 + h_2^2}} = 0
$$ (??)
$
\lim_{(h_1,h_2)\rightarrow (0,0)} \frac{|h_1^2 h_2|}{(h_1^2+h_2^2)^{\frac{3}{2}}}=
\lim_{n\rightarrow \infty} \frac{\left(\frac{1}{n}\right)^2\frac{1}{n}}{\left((\frac{1}{n})^22\right)^{\frac{3}{2}}}=0
$

\begin{tw}
    Tw. $f: \mathbb{R}^n \rightarrow \mathbb{R}^m, x\in\mathbb{R}^n$. Zakładamy, że pochodne cząstkowe: $\frac{\partial f_i}{\partial x_j} (x)$
    istnieją w otoczeniu punktu $x$ i są ciągłe w punkcie $x$. Wtedy $f$ jest różniczkowalna w punkcie $x$.
\end{tw}

\begin{pk}
    $f:\mathbb{R}^n \rightarrow \mathbb{R}$ $f(x)=\langle x, x \rangle$\\
    $x=(x_1,x_2,\dots,x_n)$\\
    $f(x)=x_1^2+x_2^2+\dots+x_n^2$\\
    $\frac{\partial f}{\partial x_1} = 2x_1, \frac{\partial f}{\partial x_2} = 2x_2, \dots$\\
    $$M_{f'(x)} = \left[2x_1, 2x_2, \dots, 2x_n\right]=2x$$
    $$M_{f'(x)}(h) = 2\langle x,h\rangle$$
\end{pk}

\begin{pk}
    Z definicji $\frac{\left|f(x+h)-f(x)-M_{f'(x)}(h) \right|}{|h|}=\frac{\langle h,h \rangle}{|h|} = \frac{|h||h|}{|h|}$\\
    Jednak algebraicznie $f(x+h)-f(x)-M_{f'(x)}(h)=\langle x+h,x+h \rangle - \langle x, x \rangle - 2 \langle x, h \rangle=\langle h,h \rangle$
\end{pk}

\begin{pk}
    $f:\mathbb{R}\rightarrow \mathbb{R}^2, f(t)=(\sin(t),\cos(t))$\\
    $$
    M_{f'(t)} = \left[
        \begin{matrix}
        sin(t)'\\
        cos(t)'
        \end{matrix}
    \right]=\left[
        \begin{matrix}
            cos(t)\\
            -sin(t)
        \end{matrix}
    \right]
    $$
\end{pk}

\begin{de}
    Pochodne kierunkowe: $f: \mathbb{R}^n\rightarrow \mathbb{R}$. 
    Pochodną kierunkową funkcji $f$ w punkcie $x_0$ 
    w kierunku wektora $\overline{a}$ nazywamy granicę:
    \begin{center}
        \[(D_a f)(x_0)=\lim_{t\rightarrow 0} \frac{f(x_0 + at) - f(x_0)}{t}\]
    \end{center}
    $\varphi(t)=f(x_0+at)$\\
    $\varphi'(0)=\lim_{h\rightarrow 0} \frac{\varphi(h)-\varphi(0)}{h}=\lim_{h\rightarrow 0} \frac{f(x_0+ah)-f(x_0)}{h}=(D_a f)(x_0)$
\end{de}

\begin{pk}
    $f(x,y) = sin(x)\cdot y$\\
    $x_0 = (0,0)$\\
    $a=(\frac{\sqrt{2}}{2},\frac{\sqrt{2}}{2})$\\
    \[D_a f (0,0) = \lim_{t\rightarrow 0} \frac{f\left(\frac{\sqrt{2}}{2}t,\frac{\sqrt{2}}{2}\right)-f(0,0)}{t}=\lim_{t\rightarrow 0} \frac{\sin(\frac{\sqrt{2}}{2})}{t}=\frac{\sqrt{2}}{2}\]
\end{pk}

\begin{tw}
    Jeżeli $f$ jest różniczkowalna w punkcie $x_0$ oraz $a\in\mathbb{R}-\{0\}$, to:
    \[(D_a f)(x_0) = f'(x_0) a^{T}\]
    \[f'(x_0)=\left[\frac{\partial f}{\partial x_1} (x_0),\dots,\frac{\partial f}{\partial x_n} (x_n)\right]\]
    \[f'(x_0)a = \frac{\partial f}{\partial x_1}(x_0)a_1 + \dots + \frac{\partial f}{\partial x_n} (x_0) a_n\]
\end{tw}

\begin{de}
    Gradient funkcji $f: \mathbb{R}^n\rightarrow\mathbb{R}$ w punkcie $x_0\in\mathbb{R}^n$:
    \[\nabla_{x_0} f = \left(\frac{\partial f}{\partial x_1} (x_0), \frac{\partial f}{\partial x_2} (x_0),\dots, \frac{\partial f}{\partial x_n} (x_0)\right)\]
    \[(D_a f) (x_0) = \langle \nabla_{x_0} f, a \rangle\]
\end{de}

\begin{pk}
    Policzmy gradient dla: $f(x,y)=x^2+y^2$.
    \[\nabla_(x_0,y_0) f(x,y) = (2x_0, 2y_0)\]
    \[\nabla f(x,y) = (2x,2y)\]
\end{pk}
D-d. Zakładamy, że $f$ jest różniczkowalna w $x_0$. 
\[0=\lim_{h\rightarrow 0} \frac{|f(x+h)-f(x_0)-f'(x_0)h|}{|h|}=\]
\[=\lim_{h\rightarrow 0} \frac{1}{|a|} \left|\frac{f(x_0+a)-f(x_0)-f'(x_0)t(ta)}{t}\right|=\]
\[=\lim_{t\rightarrow 0} \frac{1}{|a|} \left|\frac{f(x_0+ta)-f(x_0)}{t}-f'(x_0)(a)\right|\]

\subsection{Gradient}

$f:\mathbb{R}^n \rightarrow \mathbb{R}$

\[\nabla_{x_0} f = \left(\frac{\partial f}{\partial x_1} (x_0), \dots, \frac{\partial f}{\partial x_n} (x_0)\right)\]

\subsection{Własności gradientu}

Niech $\alpha \in \mathbb{R}$
\begin{enumerate}
    \item $\nabla_{x_0} (\alpha f) = \alpha \nabla_{x_0} (f)$
    \item $\nabla_{x_0} (f+g) = \nabla{x_0} (f) + \nabla{x_0} (g)$
\end{enumerate}
Wniosek liniowość (dla $\alpha,\beta \in \mathbb{R}$):
\[\nabla_{x_0} (\alpha f + \beta g) = \alpha \nabla_{x_0} (f) + \beta \nabla_{x_0} (g)\]

\subsection{Gradient iloczynu}
\[\nabla_{x_0} (f\cdot g) = \nabla_{x_0} (f) \cdot g(x_0) + f(x_0) \nabla_{x_0} (g)\]
D-d. z liniowości gradientu.

\begin{tw}
Zakładamy, że $f: \mathbb{R}^n \rightarrow \mathbb{R}^m$ jest różniczkowalna w $x\in\mathbb{R}^n$.
Wówczas $f$ jest ciągła w $x\in\mathbb{R}^n$.\\\\
D-d. (metryka w $\mathbb{R}^n$)\\
\[|f(x+h)-f(x)|=\]
\[=|f(x+h)-f(x)-f'(x)h + f'(x)h|\]
\[\leq |f(x+h)-f(x)-f'(x)h|+|f'(x)h|=\]
\[=\frac{|f(x+h)-f(x)-f'(x)h|}{|h|} |h| = |f'(x)h|\]
\end{tw}

\subsection{Minimum lokalne właściwe}

\begin{de}
    $f:\mathbb{R}^n \rightarrow \mathbb{R}, a\in\mathbb{R}^n$. 
    Mówimy, że $f$ ma w punkcie $a$ minimum lokalne (właściwe)
    jeśli istnieje $r>0$:
    \[\left(\forall x\in K(a,r) - \{a\}\right) f(a) \leq_{(<)} f(x)\] 
    Gdzie $K(a,r)$ - kula towarta o środku w $a$ i promieniu $r$.
\end{de}

\begin{tw}
    (Warunek konieczny) $f: \mathbb{R}^n \rightarrow \mathbb{R}, a\in\mathbb{R}^n$.
    Jeżeli $f$ ma w punkcie $a$ ekstremum lokalne to:
    \[\nabla_{x_0} f = (0,0,\dots, 0)\]
    Inaczej:
    $\frac{\partial f}{\partial x_1} (a) = \dots = \frac{\partial f}{\partial x_n} (a) = 0$
\end{tw}

\begin{pk}
    Rozważmy następujące przykłady:
    \begin{enumerate}
        \item $f(x,y)=x^2-y^2$ - sprawdźmy warunek konieczny\\
        $\frac{\partial f}{\partial x} = 2x = 0, \frac{\partial f}{\partial y} = 2y = 0$\\
        $(0,0)$ - punkt podejrzany\\
        $f(\frac{1}{n},0) = \frac{1}{n^2} > 0$
        $f(0, \frac{1}{n}) = -\frac{1}{n^2} < 0$\\
        Nie istnieje taka $K((0,0),r)$, że $f$ ma w tej kuli stały znak.
        \item $f(x)=x^3$, $f\left(\frac{1}{n}\right) = \frac{1}{n^3}$, $f\left(-\frac{1}{n}\right)\dots$
        \item $f(x,y,z) = -x^2 - (y-1)^2 - (z+1)^2$\\
        Sprawdźmy warunek: $\nabla f = (-2x, -2(y-1), -2(z+1))=(0,0,0)$\\
        Zobaczmy $f(0,1,-1)=0$\\
        $f(a,b+1,c-1) - f(0,1,-1) = -a^2 -b^2 -c^2 = -(a^2+b^2+c^2) < 0$ maksimum lokalne
        $f(a,b+1,c-1) < f(0,1,-1)$ (liczby sześcianu opisane na kuli)\\
    \end{enumerate}
\end{pk}

\begin{de}
    Niech $f(x,y): \mathbb{R}^2 \rightarrow \mathbb{R}$ $x,y\in\mathbb{R}$.
    Zakładamy, że $\frac{\partial f(x,y)}{\partial x} = f_x(x,y)$ oraz $\frac{\partial f(x,y)}{\partial y} = f_y(x,y)$.
    Wtedy pochodne cząstkowe pochodnych $f_x(x,y), f_y(x,y)$ nazywamy pochodnymi cząstkowymi drugiego rzędu:

    \begin{enumerate}
        \item $\frac{\partial^2 f}{\partial x^2} = \frac{\partial}{\partial x} \left(\frac{\partial f}{\partial x}\right) = f_{xx} (x,y)$
        \item $\frac{\partial^2 f}{\partial y^2} = \frac{\partial}{\partial y} \left(\frac{\partial f}{\partial y}\right) = f_{yy} (x,y)$
        \item $\frac{\partial^2 f}{\partial x \partial y} = \frac{\partial}{\partial x} \left(\frac{\partial f}{\partial y}\right) = f_{xy} (x,y)$
        \item $\frac{\partial^2 f}{\partial y \partial x} = \frac{\partial}{\partial y} \left(\frac{\partial f}{\partial x}\right) = f_{yx} (x,y)$
    \end{enumerate}
\end{de}

\begin{tw}
  Jeżeli pochodne cząstkowe istnieją w pewnym obszarze i obie są, w pewnym punkcie ciągłe, to w tym punkcie są równe
  \[ f_{xy} = f_{yx} \]
\end{tw}

\[\frac{\partial^{p+q} f}{\partial x^p \partial y^q}\]\\

$\frac{\partial^3 f}{\partial x^2 \partial y}$
$f(x,y) = x^3 y^2$\\

\[\frac{d}{dy} (x^3 y^2) = x^3 2y\]
\[\frac{d}{dx} (x^3 2y) = 3x^2 2y\]
\[\frac{d}{dx} (3x^2 2y) = 6x 2y\]

\[D\left[x^3 \cdot y^2, (x,2), (y,1)\right] \text {(wolframalpha)}\]

\subsection{Różniczkowanie złożenia funkcji}

\begin{tw}
    $f:\mathbb{R}^n \rightarrow \mathbb{R}^m, g: \mathbb{R}^m \rightarrow \mathbb{R}^k$ 
    $a\in\mathbb{R}^n, b=f(a)\in \mathbb{R}^m$. Zakładamy, że $f$ jest różniczkowalna w punkcie $a$
    oraz $g$ jest różniczkowalna w punkcie $b$. Wtedy $g\circ f$ jest różniczkowalna w punkcie $a$ i zachodzi wzór:
    \[(g\circ f)(a) = g'\left(f(a)\right) \circ f'(a)\]
    Złożenie odwzorowań liniowych - mnożenie macierzy.
\end{tw}

\begin{pk}
    $U: \mathbb{R} \rightarrow \mathbb{R}^3, f: \mathbb{R}^3 \rightarrow \mathbb{R}, g=f\circ u : \mathbb{R} \rightarrow \mathbb{R}\\$
    $u(t) = \left(u_1 (t), u_2 (t), u_3 (t)\right)$\\
    $g(t) = (f\circ u) (t) = f(u_1(t),u_2(t),u_3(t))$\\
    Zobaczmy:

    $M_{u'(t)} = \begin{bmatrix}
        u_1'(t) \\
        u_2'(t) \\
        u_3'(t)
    \end{bmatrix}  
    $
    $M_{f'(b)} = \left(\frac{\partial f}{\partial x_1}, \frac{\partial f}{\partial x_2}, \frac{\partial f}{\partial x_3}\right)_{x=b}$\\
    Wykonajmy mnożenie macierzy:
    \[M_{f'(b)}\cdot M_{u'(b)} = \frac{\partial f}{\partial x_1} \frac{d u_1}{dt} + \frac{\partial f}{\partial x_2} \frac{d u_2}{dt} + \frac{\partial f}{\partial x_3} \frac{d u_3}{dt}\]
\end{pk}

\begin{fakt}
    Uogólniona reguła łańcuchowa. ($x_i = u_i(t)$)
    \[\frac{d}{dt} f\left(u_1(t), u_2(t), \dots, u_n(t)\right) = \sum_{i=1}^{n} \frac{\partial f}{\partial x_i} \frac{d u_i}{dt}\]
\end{fakt}

\begin{pk}
    Niech $u(t) = (t^2-t, 2t, 4t), f(x,y,z)=xy+z, g=f\circ u, g(t)=(t^2-t)2t+4t$\\\\
    $g'(t) = (t^2-1)'2t + (t^2-t)(2t)' + (4t)' = (2t-1)(2t) + 2(t^2-t) + 4=$\\
    $= 4t^2 - 2t + 2t^2 - 2t + 4 = 6t^2 - 4t + 4$\\
    Zobaczmy z reguły łańcuchowej:\\
    \[g(t)=(f\circ u)(t)\]
    \[g'(t)=\frac{\partial f}{\partial x_1} \frac{du_1}{dt} + \frac{\partial f}{\partial x_2} \frac{du_2}{dt} + \frac{\partial f}{\partial x_3} \frac{du_3}{dt}\]
    \[y\cdot (2t-1) + 2x + 1\cdot 4 = 2t (2t-1) + 2(t^2-t) + 4 = 4t^2 - 2t + 2t^2 - 2t + 4 = 6t^2 - 4t + 4\]
\end{pk}
W zastosowaniu algorytmu back propagation.

\end{document}
