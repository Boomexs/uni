\documentclass{article}

\usepackage[polish]{babel}
\usepackage[utf8]{inputenc}
\usepackage{polski}
\usepackage[T1]{fontenc}
 
\usepackage[margin=1.5in]{geometry} 
\usepackage{color} 
\usepackage{amsmath}
\usepackage{amsfonts}
\usepackage{graphicx}
\usepackage{booktabs}
\usepackage{amsthm}
\usepackage{pdfpages}
\usepackage{wrapfig}
\usepackage{hyperref}
\usepackage{etoolbox}
\AtBeginEnvironment{align}{\setcounter{equation}{0}}

\theoremstyle{definition}
\newtheorem{de}{Definicja}[subsection]

\theoremstyle{definition}
\newtheorem{tw}{Twierdzenie}[subsection]

\theoremstyle{definition}
\newtheorem{pk}{Przykład}[subsection]

\theoremstyle{definition}
\newtheorem*{fakt}{FAKT}

\author{Rafal Wlodarczyk}
\title{Analiza Matematyczna II}  
\date{INA 2 Sem. 2023}

\begin{document}

\maketitle

\section{Wykład I}

\subsection{Iloczyn skalarny}

$\mathbb{R}^n=\{(x_1,x_2,\dots, x_n), x_i \in \mathbb{R}\}$

\begin{de}
Dla $x,y \in \mathbb{R}^n$ definiujemy iloczyn skalarny:
\begin{center}
    $<x,y> = \sum_{i=1}^{n} x_i y_i$
\end{center}
$(x=(x_1,x_2,\dots, x_n) y = (y_1,y_2,\dots, y_n))$\\
$ax = (ax_1, ax_2, \dots, ax_n)$\\
$\sqrt{\langle x,x \rangle} = \sqrt{x_1^2+x_2^2+\dots+x_n^2}$\\
Własności:
\begin{enumerate}
    \item $\langle x,y \rangle = <y,x>$
    \item $<ax,y> = <x,ay> = a<x,y>, a \in \mathbb{R}$
    \item $<x+y,z>=<x,z>+<y,z>$
\end{enumerate}
\end{de}

\begin{de}
    Długość wektora $x\in \mathbb{R}^n$\\
    \begin{center}
        $|x|=\sqrt{\langle x,x \rangle}$
    \end{center}
\end{de}

\begin{pk}
	$\mathbb{R}$ : $|x|=|x_1|$ - oś liczbowa
	$\mathbb{R}^2$ : $|x|=\sqrt{x_1^2+y_2^2}$ - płaszczyzna
\end{pk}

\begin{tw}
	$x\in\mathbb{R}^n$. Wówczas $|\langle x,y \rangle| \leq |x|\cdot |y|$\\
	D-d. $x=(x_1,x_2,\dots,x_n)$, $y=(y_1,y_2,\dots, y_n)$\\
	$|\langle x,y, \rangle| = |\sum_{i=1}^{n} x_i y_i| \leq \sqrt{\sum_{i=1}^{n} x_i^2} + \sqrt{\sum_{i=1}^{n} y_1^2}$\\
	Nierówność Cauchy'ego Schwarza, a zatem dowód.
\end{tw}
Wniosek $x,y\in\mathbb{R}^n$.
\begin{center}
	$|x+y| \leq |x| + |y|$
\end{center}
D-d.\\
$|x+y|^2=\langle x+y, x+y \rangle$
$= \langle x, x+y \rangle + \langle y, x+y \rangle$
$= \langle x, x \rangle + \langle x, y \rangle + \langle y, y \rangle + \langle y, x \rangle$
$=|x|^2+|y|^2 + 2\langle x,y \rangle \leq |x|^2 |y|^2 + 2|x||y|$
$=(|x|+|y|)^2$
$=|x+y|^2 \leq (|x|+|y|)^2$
$\iff$
$|x+y| \leq |x| + |y|\qed$

\subsection{Kąt między wektorami}

$\overrightarrow{x_1}, \overrightarrow{x_2} \in \mathbb{R}^2$,
$\overrightarrow{x_1} = (x_{11}, x_{12})$,
$\overrightarrow{x_2} = (x_{21}, x_{22})$\\
\begin{center}
    $\cos(\overrightarrow{x_1}, \overrightarrow{x_2})=$
    $\frac{\overrightarrow{x_1} \odot \overrightarrow{x_2}}{|\overrightarrow{x_1}||\overrightarrow{x_2}|}$
\end{center}
Rozważmy funkcję:\\
$d_n \mathbb{R}^n \times \mathbb{R}^n \rightarrow \mathbb{R}$\\
$d_n(x,y) = |x-y|$, dla $n=2$:\\\\
$x=(x_1,x_2), y=(y_1,y_2)$
$|x-y| = \sqrt{(x_1-y_1)^2+(x_2-y_2)^2}$
Własności:
\begin{enumerate}
    \item $d_n(x,y)\geq 0$
    \item $d_n(x,y)=0 \iff x=y$
    \item $d_n(x,y)=d_n(y,x)$
    \item $d_n(x,z)\leq d_n(x,y) + d_n(y,z)$ - nierówność trójkąta
\end{enumerate}

\subsection{Przestrzeń metryczna}

\begin{de}
Przestrzenią metryczną nazywamy dowolny zbiór $X$, pewną funkcję $X\times X \rightarrow \mathbb{R}$,
która spełnia następujące aksjomaty:
\begin{enumerate}
    \item $d(x,y)\geq 0$
    \item $d(x,y)=0 \iff x=y$
    \item $d(x,y)=d(y,x)$ dla $x,y\in X$
    \item $d(x,z)\leq d(x,y) + d(y,z)$ dla każdych $x,y,z \in X$
\end{enumerate}
Funkcję $d$ nazywamy metryką, a wartość $d(x,y)$ odległością punktów.

\end{de}
Uwaga: aksjomat $1$ wynika z pozostałych aksjomatów\\
D-d.\\
$d(x,y) = \frac{1}{2} \left(d(x,y)+d(y,x)\right) \geq \frac{1}{2} d(x,x) = 0$, zatem $d(x,y)\geq 0$

\begin{tw}
Stwierdzenie. Niech $(X,d)$ (Corollary) będzie przestrzenią metryczną oraz $x_1,x_2,\dots,x_n\in X$.
Wówczas:
\begin{center}
    $d(x_1,x_n)\leq\sum_{j=1}^{n-1} d(x_j,x_{j+1}), n\geq 2$
\end{center}
Dla $n=2$:\\
    $d(x_1,x_2)\leq d(x_1,x_2)$ - oczywiste\\
Dla $n=3$:\\
    $d(x_1,x_3)\leq d(x_1,x_2) + d(x_2,x_3)$ - nierówność trójkąta\\
Krok indukcyjny:\\
    $d(x_1,x_{n+1})\leq d(x_1,x_2)+d(x_2,x_3)+\dots+d(x_n,x_{n+1})$\\
    $d(x_1,x_{n+1})\leq d(x_1,x_n)+d(x_n,x_{n+1}) \leq_{ind}$
    $d(x_1,x_2) + d(x_2,x_3) + \dots + d(x_{n-1}, x_n) + d(x_{n},x_{n+1})\qed$
\end{tw}

\subsection{Przestrzeń metryczna dyskretna}

$X$ - dowolny zbiór i metryka określona wzorem:
\begin{center}
    $$d(x,y)=\begin{cases}
        0 \text{ dla } x=y\\
        1 \text{ dla } x\neq y
    \end{cases}$$
\end{center}
Aksjomaty $1,2,3,4$ są oczywiste.

\subsection{Metryka Euklidesowa}

$d_n: \mathbb{R}^n \times \mathbb{R}^n \rightarrow \mathbb{R}$\\
$x=(x_1,x_2,\dots, x_n)$\\
$y=(y_1,y_2,\dots, y_n)$\\
$d_n(x,y)=\sqrt{\sum_{i=1}^{n} (x_i-y_i)^2}$

\subsection{Przestrzeń Hilberta}

$x=(x_1,x_2,\dots, x_n)$\\
$y=(y_1,y_2,\dots, y_n)$\\
$\sum_{i=1}^{\infty} x_i^2\leq \infty, \sum_{i=1}^{\infty} y_i^2 < \infty$\\
$x=\left(\frac{1}{1},\frac{1}{2},\frac{1}{3},\dots,\frac{1}{i}, \dots \right)$\\
$\sum_{i=1}^{\infty} \frac{1}{i^2} \leq \infty$\\
$\sum_{i=1}^{\infty} \left(\frac{1}{\sqrt{i}}\right)^2=\sum_{i=1}^{\infty} \frac{1}{i} = \infty$\\
$y=\left(\frac{1}{\sqrt{1}},\frac{1}{\sqrt{2}},\dots \right)$

\subsection{Metryka Manhattan}

$d(x,y)=|x_1-y_1| + |x_2-y_2|$\\
$d((x_1,x_2),(y_1,y_2))=|x_1-y_1|+|x_2-y_2|$

\end{document}
